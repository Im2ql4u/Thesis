\graphicspath{{../results/figures/theory/}}
This section will lay the foundation for which everything preceding will rely on. First, focusing on quantum mechanics, we will introduce the mathematical framework of quantum mechanics.
 We will then discuss the challenges of many-body systems and how second quantization provides a more tractable approach. 
Then, we will introduce the model systems that will be used throughout this thesis to benchmark and test our methods. Following this, we will present the Hartree-Fock method, a fundamental mean-field approach to approximate many-body wavefunctions.
 Finally, we will discuss Full Configuration Interaction (FCI) theory, which provides an exact solution to the many-body problem within a finite basis set, and Variational Monte Carlo (VMC), a stochastic method for approximating ground-state properties of quantum systems.

 The focus will then shift, and proceeds to introduce the essential concepts of machine learning, covering statistical learning theory, optimization techniques, neural networks and Physics informed neural networks (PINNS). This section will provide the theoretical underpinnings necessary for understanding how machine learning can be applied to quantum many-body problems.

\chapter{Quantum Theory}
\label{sec:theory}

\section{Mathematical Foundations}
At the center of quantum mechanics lies a rich mathematical framework built upon linear algebra and functional analysis. This section introduces the essential mathematical concepts and notations that underpin quantum theory, including Dirac notation, operators, Hilbert spaces, and the representation of quantum states and observables.
\subsection{Hilbert Spaces and Function Representations}
In essence, quantum mechanics is formulated in complex vector spaces, specifically infinite-dimensional Hilbert spaces, where vectors generalize to square-integrable functions. The inner product becomes:  
\[
\braket{\psi_i|\psi_j} = \int \psi_i^*(x)\psi_j(x) dx = \delta_{ij}.
\]
A function \(a(x)\) expands in a basis \(\{\psi_i(x)\}\):  
\[
a(x) = \sum_i \psi_i(x) a_i, \quad a_i = \int \psi_i^*(x)a(x) dx.
\]
The Dirac delta \(\delta(x-y)\) replaces the Kronecker delta in continuous spaces.
A vector \(\ket{a}\) in an \(n\)-dimensional Hilbert space \(\mathcal{H}\) can be decomposed into a complete orthonormal basis \(\{\ket{i}\}\):  
\[
\ket{a} = \sum_i \ket{i}\braket{i|a} = \sum_i a_i \ket{i},
\]
where \(a_i = \braket{i|a}\) are complex coefficients. The completeness relation,  
\[
I = \sum_i \ket{i}\bra{i},
\]
ensures the basis spans \(\mathcal{H}\). Vectors in Dirac notation have dual "bra" counterparts \(\bra{a} = \ket{a}^\dagger\), enabling inner products \(\braket{a|b}\) and outer products \(\ket{a}\bra{b}\). A more complete discussion on hilbert spaces are found in \cite{Rudin1976-Analysis}. 

\subsection{Operators and Matrix Representations}
Linear operators \(\hat{\mathcal{O}}\) map vectors to vectors: \(\hat{\mathcal{O}}\ket{a} = \ket{b}\). Their adjoints \(\hat{\mathcal{O}}^\dagger\) satisfy \(\bra{a}\hat{\mathcal{O}}^\dagger = \bra{b}\). Hermitian operators (\(\hat{\mathcal{O}}^\dagger = \hat{\mathcal{O}}\)) represent observables, while unitary operators (\(\hat{\mathcal{O}}^\dagger = \hat{\mathcal{O}}^{-1}\)) preserve inner products \cite{BereraDelDebbio2021-QuantumMechanics,GriffithsSchroeter2018-IntroQM}.
 In a basis \(\{\ket{i}\}\), operators are represented by matrices:  
\[
O_{ij} = \bra{i}\hat{\mathcal{O}}\ket{j}.
\]
Non-commutativity of operators is quantified by the commutator:  
\[
[\hat{A}, \hat{B}] = \hat{A}\hat{B} - \hat{B}\hat{A}.
\]

\subsection{Hilbert Spaces}

In quantum mechanics a state is a vector $\ket{\psi}$ in a complex Hilbert space $\mathcal H$.
A Hilbert space is a complex inner-product space that is \emph{complete} in the norm induced by
the inner product. Using the physics convention (linear in the second slot),
the inner product $\braket{\phi|\psi}$ obeys
\begin{align*}
\text{conjugate symmetry:}\quad & \braket{\phi|\psi}=\braket{\psi|\phi}^*,\\
\text{linearity in kets:}\quad & \braket{\phi|\alpha\psi_1+\beta\psi_2}
= \alpha\,\braket{\phi|\psi_1}+\beta\,\braket{\phi|\psi_2},\\
\text{positivity:}\quad & \braket{\psi|\psi}\ge 0 \text{ with equality iff } \ket{\psi}=0.
\end{align*}
These allow us to define the norm $\|\psi\|=\sqrt{\braket{\psi|\psi}}$. \emph{Completeness} means every Cauchy sequence in this metric
converges to a vector in $\mathcal H$, ensuring limits and expansions are well defined. For a more detailed explaination, refer to \cite{BereraDelDebbio2021-QuantumMechanics,GriffithsSchroeter2018-IntroQM}.


\subsection{States in Hilbert Space}

A (pure) quantum state is a normalized vector $\ket{\psi}\in\mathcal H$ (global phase is
physically irrelevant). Computations proceed by choosing an orthonormal (ON) basis
$\{\ket{i}\}_{i=1}^\infty$ and expanding
\[
\ket{\psi}=\sum_{i} c_i \ket{i}, \qquad
c_i=\braket{i|\psi}, \qquad \sum_i |c_i|^2=\|\psi\|^2 .
\]
Completeness is the resolution of the identity,
$\sum_i \ket{i}\!\bra{i}=\mathbb I$. For continuous ON bases (e.g.\ position),
$\int \ket{x}\!\bra{x}\,dx=\mathbb I$ and the wavefunction is the coordinate
representation $\psi(x)=\braket{x|\psi}$.

Different ON bases are related by a unitary change of basis. Given
$\{\ket{i}\}$ and $\{\ket{\alpha}\}$,
\[
\ket{\alpha}=\sum_i U_{\alpha i}\ket{i}, \qquad
U_{\alpha i}=\braket{\alpha|i}, \qquad U^\dagger U=\mathbb I .
\]
This freedom is exploited to match the physics (symmetry, locality, boundary conditions),
e.g.\ harmonic-oscillator/Fock–Darwin orbitals for confined charges or plane waves for
periodic systems, leading to sparser representations and faster convergence.

For $N$ particles the Hilbert space factorizes as
$\mathcal H_N=\mathcal H_1^{\otimes N}$. With a single-particle ON set
$\{\ket{\varphi_p}\}$, a convenient fermionic ON basis is the set of Slater determinants
$\{\ket{\Phi_I}\}$ built from $N$ distinct orbitals; any $N$-body state expands as
\[
\ket{\Psi}=\sum_{I} C_I \ket{\Phi_I}, \qquad \sum_I |C_I|^2=\|\Psi\|^2 .
\]
In practice one truncates to a finite active space guided by energy scales and symmetry
quantum numbers, which makes many-body calculations tractable while retaining the relevant
physics.

\section{Quantum Mechanics}
\subsection{The Schrödinger Equation}

In a Hilbert space $\mathcal H$, the time evolution of a normalized state $\ket{\Psi(t)}$ is
postulated to be unitary and generated by the Hamiltonian $H$ \cite{BereraDelDebbio2021-QuantumMechanics,GriffithsSchroeter2018-IntroQM}:
\[
i\,\partial_t \ket{\Psi(t)} = H \ket{\Psi(t)}, \qquad (\hbar=1).
\]
If $H$ is time–independent, solutions separate as
\[
\ket{\Psi(t)}=\sum_n c_n\,e^{-iE_n t}\ket{\psi_n},
\qquad H\ket{\psi_n}=E_n\ket{\psi_n},
\]
so $\{\ket{\psi_n}\}$ forms an orthonormal basis of stationary states.

For an $N$-particle system in $d$ spatial dimensions, with coordinates
$R=(\mathbf r_1,\ldots,\mathbf r_N)$, we use the many-body Hamiltonian
\[
H = -\frac12 \sum_{i=1}^N \nabla_{\mathbf r_i}^2
    \;+\; \sum_{i=1}^N V_{\text{ext}}(\mathbf r_i)
    \;+\; \sum_{i<j} V_{\text{int}}(|\mathbf r_i-\mathbf r_j|),
\]
and the time-independent Schrödinger equation (TISE)
\[
H\,\Psi(R)=E\,\Psi(R), \qquad \int | \Psi(R) |^2\, dR = 1.
\]
Identical particles restrict $\Psi$ to the appropriate symmetry sector (symmetric for bosons,
antisymmetric for fermions).

Throughout we work in atomic units ($\hbar=m_e=e=1$) and focus on stationary problems.
In the quantum-dot applications below we take
$V_{\text{ext}}(\mathbf r)=\tfrac12 \omega^2 r^2$ and $V_{\text{int}}(r)=1/r$ (in 2D this is
understood with the usual Laplacian in two dimensions). Our computations therefore amount to
approximating low-lying eigenpairs $(E,\Psi)$ of the many-body Hamiltonian above using compact
basis expansions and variational ansätze.
% \subsubsection{Rayleigh--Ritz in a Finite Subspace}
% Let $H$ be self-adjoint on a separable Hilbert space $\mathcal H$. For any nonzero $\ket{\psi}\in\mathcal H$,
% the Rayleigh quotient
% \[
% E[\psi]\;=\;\frac{\braket{\psi|H|\psi}}{\braket{\psi|\psi}}
% \]
% satisfies $E[\psi]\ge E_0$, where $E_0$ is the exact ground-state energy. In practice we project
% onto a finite-dimensional subspace
% \[
% \mathcal V_M=\mathrm{span}\{\ket{\varphi_p}\}_{p=1}^M
% \]
% (the \emph{computational basis}). The constrained minimization
% $\min_{\psi\in \mathcal V_M\setminus\{0\}} E[\psi]$ yields the generalized eigenproblem
% \[
% \mathbf H\,\mathbf c \;=\; E\,\mathbf S\,\mathbf c,\qquad
% H_{pq}=\braket{\varphi_p|H|\varphi_q},\quad
% S_{pq}=\braket{\varphi_p|\varphi_q}.
% \]
% If $\{\varphi_p\}$ is orthonormal then $\mathbf S=\mathbf I$. More generally, $\mathbf S$ is
% Hermitian positive definite and can be handled by (i) solving the generalized problem directly,
% or (ii) orthonormalizing the basis via a Cholesky/Löwdin transform:
% $\mathbf S=\mathbf L\mathbf L^\dagger$, $\ket{\tilde\varphi_p}=\sum_q (L^{-1})_{qp}\ket{\varphi_q}$,
% so that in the orthonormal set one solves $\tilde{\mathbf H}\mathbf d=E\mathbf d$ with
% $\tilde{\mathbf H}=\mathbf L^{-1\,\dagger}\mathbf H\,\mathbf L^{-1}$.

% \paragraph{Monotonicity and bounds.}
% Let $E_k^{(M)}$ denote Ritz values in $\mathcal V_M$ ordered increasingly. Then
% $E_0^{(M+1)}\le E_0^{(M)}$ and $E_k^{(M)}\ge E_k$ (Hylleraas--Undheim--MacDonald). Thus,
% increasing $M$ provides a systematic sequence of \emph{variational upper bounds}.
% This justifies investing effort in basis choices that deliver fast convergence (e.g.\ harmonic
% oscillator/Fock--Darwin orbitals for confined charges, plane waves for periodic systems).

% \paragraph{Conditioning.}
% The quality of Ritz estimates is limited not only by truncation but also by numerical
% conditioning of $\mathbf S$ and $\mathbf H$. Near-linear dependencies in $\{\varphi_p\}$ inflate
% $\kappa(\mathbf S)$ and degrade eigenpairs. In practice one removes small-eigenvalue directions
% of $\mathbf S$ (thresholding in the Löwdin step) or re-optimizes the one-body basis to improve
% conditioning before diagonalization.

% \medskip

% \subsubsection{Nonlinear Variational Ansatz}
% Linear subspaces are often too rigid for many-body problems. We therefore adopt a parametric,
% symmetry-respecting ansatz $\Psi_\theta$ (e.g.\ Slater–Jastrow with PINN/Jastrow/backflow factors)
% and minimize the energy functional
% \[
% E(\theta)\;=\;\frac{\braket{\Psi_\theta|H|\Psi_\theta}}{\braket{\Psi_\theta|\Psi_\theta}}.
% \]
% Stationarity requires $\partial_{\theta_k}E(\theta^\star)=0$ for all $k$.

% \paragraph{Normalization and invariances.}
% The energy is scale-invariant under $\Psi_\theta\!\mapsto\!\alpha\,\Psi_\theta$ ($\alpha\!\ne\!0$),
% so we can work with unnormalized wavefunctions and write the ansatz in terms of a log-amplitude:
% $\ln\Psi_\theta(R)=f_\theta(R)$ with $R=(\mathbf r_1,\dots,\mathbf r_N)$. Built-in fermionic
% antisymmetry (e.g.\ Slater determinant), spin structure, short-/long-range constraints
% (e.g.\ electron--electron cusp in Coulomb systems), and simple coordinate scalings (e.g.\
% $r\!\mapsto\! r\sqrt{\omega}$ in harmonic traps) typically reduce variance and bias and
% improve optimizer conditioning.

% \paragraph{Zero-variance principle.}
% Define the \emph{local energy}
% \[
% E_L(R)\;=\;\frac{(H\Psi_\theta)(R)}{\Psi_\theta(R)}.
% \]
% Then $E(\theta)=\mathbb E_{\pi_\theta}[E_L]$ with sampling density
% $\pi_\theta(R)=|\Psi_\theta(R)|^2/\!\int |\Psi_\theta|^2$. Moreover,
% $\mathrm{Var}_{\pi_\theta}(E_L)=0$ if and only if $\Psi_\theta$ is an exact eigenstate.
% This makes the variance of $E_L$ a sensitive diagnostic of ansatz quality.

% \medskip

% \subsubsection{Monte Carlo Estimator and Local Energy}
% For general many-body Hamiltonians
% \[
% H \;=\; -\tfrac12\sum_{i=1}^{N}\nabla_{\mathbf r_i}^2
% \;+\;\sum_{i=1}^{N} V_{\mathrm{ext}}(\mathbf r_i)
% \;+\;\sum_{i<j} V_{\mathrm{int}}(|\mathbf r_i-\mathbf r_j|),
% \]
% the local energy reads
% \[
% E_L(R)\;=\; -\tfrac12 \sum_{i}\frac{\nabla_{\mathbf r_i}^2\Psi_\theta(R)}{\Psi_\theta(R)}
% \;+\;\sum_i V_{\mathrm{ext}}(\mathbf r_i)
% \;+\;\sum_{i<j} V_{\mathrm{int}}(r_{ij}).
% \]
% With samples $\{R^{(s)}\}_{s=1}^{S}$ drawn from $\pi_\theta$, we estimate
% \[
% \widehat E=\frac{1}{S}\sum_{s=1}^S E_L(R^{(s)}),\qquad
% \widehat{\sigma}^2_{E_L}=\frac{1}{S-1}\sum_{s=1}^S\Big(E_L(R^{(s)})-\widehat E\Big)^2.
% \]
% Because Markov-chain samples are correlated, the standard error is
% \[
% \mathrm{SE}(\widehat E)\;\approx\;\sqrt{\frac{\widehat{\sigma}^2_{E_L}}{S_{\mathrm{eff}}}},
% \qquad
% S_{\mathrm{eff}}\;=\;\frac{S}{2\tau_{\mathrm{int}}}\!,
% \]
% where $\tau_{\mathrm{int}}$ is the integrated autocorrelation time (estimated via blocking/batch
% means). Acceptance rates and proposal scales (Metropolis, MALA) are tuned to keep
% $\tau_{\mathrm{int}}$ moderate while maintaining stable exploration of high-probability regions.

% \paragraph{Regularity and variance control.}
% Finite variance requires sufficient smoothness of $\Psi_\theta$ so that $\nabla^2\Psi_\theta/\Psi_\theta$
% remains integrable, particularly near nodal surfaces and coalescence points. Enforcing known
% short-distance behavior (e.g.\ cusp conditions) tames the kinetic-energy singularities and
% reduces $\mathrm{Var}(E_L)$. In practice we also (i) center observables, (ii) use large, well-mixed
% chains, and (iii) report blocking-based confidence intervals.

% \paragraph{Reweighting for small parameter moves.}
% When updating $\theta\mapsto\theta'$, one may reuse samples via importance reweighting
% $w_s \propto |\Psi_{\theta'}(R^{(s)})|^2/|\Psi_{\theta}(R^{(s)})|^2$ to form a low-variance
% line-search along the proposed step. We restrict to small steps to control weight degeneracy.

% \medskip

% \subsubsection{Gradients and Stochastic Reconfiguration (Natural Gradient)}

% Write the log-derivative observables
% \[
% O_k(R)\;=\;\partial_{\theta_k}\ln \Psi_\theta(R),
% \qquad
% \langle \cdot \rangle \equiv \mathbb E_{\pi_\theta}[\cdot].
% \]
% Assuming $H$ does not depend explicitly on $\theta$, the energy gradient admits the covariance form
% \[
% g_k\;=\;\partial_{\theta_k}E
% \;=\;\big\langle \big(E_L-\langle E_L\rangle\big)\,O_k\big\rangle,
% \]
% which avoids differentiating $E_L$ itself. This is the standard VMC ``covariance trick.''

% \paragraph{Derivation of SR from a linearized step.}
% Consider the linearized ansatz
% $\ket{\Psi_{\theta+\Delta\theta}} \approx \ket{\Psi_\theta}
% + \sum_k \Delta\theta_k\,\partial_{\theta_k}\ket{\Psi_\theta}$.
% Minimizing $E(\theta+\Delta\theta)$ to first order under a small-step constraint measured in the
% overlap metric of $\mathcal H$ leads to the normal equations
% \[
% \mathbf S\,\Delta\theta\;=\;-\eta\,\mathbf g,
% \qquad
% S_{kl}\;=\;\langle O_k O_l\rangle - \langle O_k\rangle\langle O_l\rangle,
% \]
% where $\mathbf S$ is the covariance (Fisher) matrix of log-derivatives, $\mathbf g$ the gradient, and
% $\eta>0$ a step scale. This is the \emph{Stochastic Reconfiguration} (SR) or \emph{natural-gradient}
% update; it is equivalent to steepest descent under the Fubini--Study metric and closely related to
% discrete imaginary-time evolution in the variational manifold.

% \paragraph{Practical solver and regularization.}
% We form $\mathbf g$ and apply conjugate gradients (CG) to the linear system using products
% $\mathbf v\mapsto \mathbf S\mathbf v$ estimated from the same samples:
% $\mathbf S\mathbf v \approx \frac{1}{S}\sum_s
% \big(O(R^{(s)})-\bar O\big)\big(O(R^{(s)})-\bar O\big)^\top \mathbf v$.
% To stabilize, we use Tikhonov damping and (optionally) diagonal loading:
% $\mathbf S \leftarrow \mathbf S + \lambda\,\mathbf I + \epsilon\,\mathrm{diag}(\mathbf S)$,
% with a trust-region rule limiting $\|\Delta\theta\|$.
% Centering $O_k\!\leftarrow\!O_k-\langle O_k\rangle$ improves conditioning and reduces estimator bias.

% \paragraph{Diagnostics.}
% Convergence is monitored via (i) decay of $\|\mathbf g\|$, (ii) stabilization of $\widehat E$ and
% its standard error, and (iii) reduction of $\mathrm{Var}(E_L)$. Ill-conditioning (large CG
% iterations, noisy steps) typically indicates either insufficient sampling (large $\tau_{\mathrm{int}}$),
% poorly constrained short-range behavior, or highly correlated parameters; remedies include larger
% batches, stronger damping, parameter tying, or revising the ansatz architecture.



\section{Second Quantization}

\subsection{Motivation: Why first-quantized wavefunctions don’t scale}
In first quantization an $N$-fermion state is an antisymmetric function
$\Psi(x_1,\dots,x_N)$ expanded in Slater determinants of $M$ one-particle orbitals.
The number of configurations grows as $\binom{M}{N}$ (and permanents for bosons grow even faster),
making storage and operator application increasingly intractable. We therefore move to the
occupation-number formalism, which encodes antisymmetry and particle indistinguishability
algebraically.

\subsection{Fock space and occupation representation}
Let $\{\ket{p}\}$ be an orthonormal set of one-particle \emph{spin-orbitals} (spatial orbital $+\,$spin).
The fermionic Fock space is
\[
\mathcal F_- \;=\; \bigoplus_{n=0}^{\infty} \wedge^n \mathcal H_1,
\]
with vacuum $\ket{0}$ and occupation-number basis
$\ket{n_1 n_2 \dots}$, $n_p\in\{0,1\}$. Creation/annihilation operators $a_p^\dagger, a_p$
act by adding/removing a particle in $\ket{p}$ (subject to Pauli) and obey the canonical
anticommutation relations (CAR)
\[
\{a_p, a_q^\dagger\}=\delta_{pq},\qquad \{a_p,a_q\}=\{a_p^\dagger,a_q^\dagger\}=0,
\]
with number operator $\,\hat n_p=a_p^\dagger a_p\,$. A Slater determinant occupying the set
$I=\{p_1<\cdots<p_N\}$ is $\ket{\Phi_I}=a_{p_1}^\dagger\cdots a_{p_N}^\dagger\ket{0}$ and is
represented by the bitstring $\ket{n_1 n_2 \dots}$ with $n_p=\mathbf 1_{p\in I}$.
Unitary changes of one-particle basis $\ket{p'}=\sum_p U_{p'p}\ket{p}$ lift to
$a_{p'}^\dagger=\sum_p U_{p'p} a_p^\dagger$, so physics is basis independent.

\subsection{Operators in second quantization}
Any \emph{one-body} operator $O=\sum_i O(i)$ has matrix elements
$O_{pq}=\bra{p}O\ket{q}$ and becomes
\[
O \;=\; \sum_{pq} O_{pq}\, a_p^\dagger a_q .
\]
A \emph{two-body} interaction $V=\tfrac12\sum_{i\ne j} v(i,j)$ has
\[
V \;=\; \frac12 \sum_{pqrs} \langle pq | v | rs \rangle \, a_p^\dagger a_q^\dagger a_s a_r
\;=\; \frac14 \sum_{pqrs} \langle pq \Vert rs \rangle \, a_p^\dagger a_q^\dagger a_s a_r ,
\]
where $\langle pq|v|rs\rangle$ are two-electron integrals over spin-orbitals and
$\langle pq\Vert rs\rangle=\langle pq|v|rs\rangle-\langle pq|v|sr\rangle$ denotes the
antisymmetrized integral. In this form, operator algebra and particle indistinguishability
are handled by the CAR, independent of $N$.

% \subsection{Many-body Hamiltonian for quantum dots}
% For $N$ electrons in $d$ dimensions with harmonic confinement and Coulomb repulsion,
% \[
% H \;=\; \sum_{i=1}^N \Big(-\tfrac12 \nabla_{\mathbf r_i}^2 + \tfrac12 \omega^2 r_i^2 \Big)
% \;+\; \sum_{i<j} \frac{1}{|\mathbf r_i-\mathbf r_j|}.
% \]
% In a chosen one-particle basis $\{\ket{p}\}$ (e.g.\ Fock–Darwin/H.O.\ orbitals with spin),
% this becomes
% \[
% H \;=\; \sum_{pq} t_{pq}\, a_p^\dagger a_q
% \;+\; \frac14 \sum_{pqrs} \langle pq \Vert rs \rangle \, a_p^\dagger a_q^\dagger a_s a_r,
% \]
% with
% \[
% t_{pq}=\!\int\! \phi_p^*(\mathbf r)\Big(-\tfrac12\nabla^2+\tfrac12\omega^2 r^2\Big)\phi_q(\mathbf r)\,d\mathbf r,
% \quad
% \langle pq|v|rs\rangle=\!\iint\! \frac{\phi_p^*(\mathbf r)\phi_q^*(\mathbf r')\phi_r(\mathbf r)\phi_s(\mathbf r')}
% {|\mathbf r-\mathbf r'|}\,d\mathbf r\,d\mathbf r'.
% \]
% Indices $p,q,\dots$ run over spin-orbitals; spin is included in the labels and integrals.

% \paragraph{Why this helps.}
% Second quantization (i) encodes antisymmetry algebraically, (ii) yields compact operator forms
% that do not change with $N$, (iii) maps Slater determinants to bitstrings amenable to fast
% updates, and (iv) cleanly supports truncations (active spaces) and unitary basis rotations used
% throughout the rest of this thesis.


\section{Model Systems}
% \subsection{One-Dimensional Spinless Fermions}
% For \(N\) fermions in a harmonic trap with Gaussian interactions:  
% \[
% \hat{H} = \sum_{i=1}^N \left(-\frac{1}{2} \nabla_i^2 + \frac{1}{2} x_i^2 \right) + \frac{V_0}{\sigma_0 \sqrt{2\pi}} \sum_{i<j} e^{-(x_i - x_j)^2/(2\sigma_0^2)}.
% \]
% \begin{itemize}
%     \item \textbf{Non-interacting limit:} The ground state is a Slater determinant of Hermite polynomials:  
%     \[
%     \phi_n(x) = \frac{H_n(x)e^{-x^2/2}}{\sqrt{2^n n! \sqrt{\pi}}}.
%     \]
%     Energy: \(E_{\text{gs}} = \frac{N^2}{2}\) (in \(\hbar\omega\) units).
%     \item \textbf{Interacting case:} Attractive (\(V_0 < 0\)) or repulsive (\(V_0 > 0\)) interactions modify spatial correlations and energy spectra (see Fig.~\ref{fig:interactions}).
% \end{itemize}

% \subsection{Two-Dimensional Quantum Dots}
% For fermions in a 2D harmonic trap with Coulomb repulsion:  
% \[
% \hat{H} = \sum_{i=1}^N \left(-\frac{1}{2} \nabla_i^2 + \frac{1}{2} \omega^2 r_i^2 \right) + \sum_{i<j} \frac{1}{|\bm{r}_i - \bm{r}_j|}.
% \]
% \begin{itemize}
%     \item \textbf{Closed-shell configurations:} Magic numbers \(N = 2\binom{n+2}{2}\) arise from filling degenerate orbitals with spin-up/down pairs.
%     \item \textbf{Trap frequency \(\omega\):} Lower \(\omega\) increases spatial spread and enhances correlation effects (Fig.~\ref{fig:omega_dependence}).
% \end{itemize}


%%%%%%%%%%%%%%%%%%%%%%%%%%%%%%%%%%%%%%%%%%%%%%%%%%%%%%%%%%%%%%%%%%%%%
\section{Hartree--Fock (HF)}

Hartree--Fock (HF) approximates the $N$-fermion ground state by a single Slater determinant
built from orthonormal spin–orbitals $\{\ket{p}\}_{p=1}^M$. In second quantization the
Hamiltonian is
\[
H=\sum_{pq} h_{pq}\, a_p^\dagger a_q
  + \frac{1}{4}\sum_{pqrs} (pq\Vert rs)\, a_p^\dagger a_q^\dagger a_s a_r,
\]
with one-body matrix elements $h_{pq}=\langle p|\hat h_0|q\rangle$ and antisymmetrized
two-body integrals $(pq\Vert rs)=(pq|rs)-(pq|sr)$ over spin–orbitals.

\subsection{HF energy functional and the Fock operator}
Let $\ket{\Phi}=\prod_{i=1}^N a_i^\dagger\ket{0}$ be a Slater determinant (occupied
spin–orbitals $i,j,\dots$). Its energy is
\[
E[\Phi]= \sum_{i} h_{ii} + \frac{1}{2}\sum_{i,j} (ij\Vert ij).
\]
Stationarity of $E[\Phi]$ under unitary rotations in the occupied+virtual subspace gives the
HF equations
\[
F\,\ket{\phi_p}=\varepsilon_p\,\ket{\phi_p},
\qquad
F=\hat h_0 + \sum_{j\in\mathrm{occ}}
\Big( J_j - K_j \Big),
\]
where $J_j$ and $K_j$ are the Coulomb and exchange operators generated by orbital $j$.
In a fixed orthonormal one-particle basis $\{\ket{\mu}\}$ these become the matrix
\emph{Roothaan--Hall} equations
\[
\mathbf F\,\mathbf C = \mathbf C\,\boldsymbol\varepsilon,\qquad
F_{\mu\nu}=h_{\mu\nu}+\sum_{\lambda\sigma} P_{\lambda\sigma}
\Big[(\mu\nu|\lambda\sigma)-\tfrac12(\mu\lambda|\nu\sigma)\Big],
\]
with density matrix $P_{\lambda\sigma}=\sum_{i\in\mathrm{occ}} C_{\lambda i} C_{\sigma i}^*$.
(For a non-orthogonal basis $S$, use $\mathbf F\mathbf C=\mathbf S\mathbf C\boldsymbol\varepsilon$.)

\paragraph{Self-consistency (SCF).}
HF is solved iteratively:
initialize $P^{(0)}$, build $\mathbf F[P^{(n)}]$, diagonalize to get $\mathbf C^{(n+1)}$,
update $P^{(n+1)}$, and iterate until the total energy and/or $P$ change is below tolerance.
Damping or DIIS mixing stabilizes convergence. The dominant costs are building two-electron
contributions ($\mathcal O(M^4)$) and diagonalization ($\mathcal O(M^3)$).

\paragraph{Remarks for quantum dots.}
For parabolic confinement $V_\mathrm{ext}=\tfrac12\omega^2 r^2$ we typically choose
Fock--Darwin (2D HO) spin–orbitals as the one-particle basis, which respect angular-momentum
structure and accelerate convergence. Closed shells admit restricted HF (RHF); spin-polarized
cases may require UHF. Koopmans’ theorem (orbital energies as removal/addition estimates)
can provide a rough diagnostic but is not used for final energetics here.

\medskip

\section{Full Configuration Interaction (FCI)}

FCI gives the exact ground state \emph{within the chosen one-particle basis} by expanding in
\emph{all} $N$-electron Slater determinants,
\[
\ket{\Psi}=\sum_{I} C_I \ket{\Phi_I},\qquad
H\mathbf C=E\,\mathbf C.
\]
The determinant space has dimension $\binom{M}{N}$ for fixed spin–orbitals $M$ and particles
$N$ (or smaller with symmetry constraints). FCI thus captures the full correlation energy
$\Delta E=E_\mathrm{FCI}-E_\mathrm{HF}$ in that basis and serves as the gold-standard benchmark.

\subsection{Slater--Condon rules and sparsity}
Matrix elements $\langle \Phi_I|H|\Phi_J\rangle$ are nonzero only if $\Phi_I$ and $\Phi_J$
differ by at most two spin–orbitals:
\begin{itemize}
\item \emph{Diagonal:} $H_{II}=\sum_{i\in I} h_{ii} + \tfrac12\sum_{i,j\in I} (ij\Vert ij)$.
\item \emph{Singles $i\!\to\!a$:} $H_{0,i^a}=F_{ia}$; in canonical HF orbitals, $F_{ia}=0$
(Brillouin’s theorem), so the HF determinant does not couple to singles.
\item \emph{Doubles $ij\!\to\!ab$:} $H_{0,ij^{ab}}=(ij\Vert ab)$.
\end{itemize}
This locality yields a very sparse Hamiltonian in the determinant basis and enables iterative
eigensolvers without forming $\mathbf H$ explicitly.

\subsection{Diagonalization and symmetry reduction}
In practice one targets the lowest eigenpairs with Davidson or Lanczos iterations. Symmetries
($N$, $S_z$, total spin $S$ if spin-adapted CSFs are used, and $L_z$ for Fock--Darwin bases)
block-diagonalize the problem and dramatically shrink the working space and memory. For
few-electron quantum dots (e.g.\ $N=2,6$) this makes high-accuracy FCI feasible for moderate
$M$; for larger $N$ the determinant growth becomes the bottleneck.

\subsection{Role in this thesis}
HF provides physically meaningful orbitals and an upper bound; FCI (in the same one-particle
basis) provides benchmark energies for small systems. We use these as references to assess the
accuracy of our variational Slater–Jastrow/PINN results for 2D quantum dots. For larger $N$
(where FCI is prohibitive) we rely on variational energies plus uncertainty estimates.
\section{Quantum Dots}

\subsection{Model and units}
We consider $N$ spin-$\tfrac12$ fermions in $d{=}2$ with harmonic confinement and Coulomb repulsion,
\[
H \;=\; \sum_{i=1}^N\!\bigg(-\tfrac12\nabla_{\mathbf r_i}^2 + \tfrac12 \omega^2 r_i^2\bigg)
\;+\;\sum_{i<j}\frac{1}{|\mathbf r_i-\mathbf r_j|}, \qquad (\hbar=m_e=e=1).
\]
Spin does not appear in $H$ (spin–independent), so the many-body eigenstates can be chosen as
products of a spatial wavefunction and a spin function with well-defined $S$ and $S_z$.

\subsection{One–particle structure and shell filling}
We use Fock–Darwin (2D harmonic–oscillator) spin–orbitals $\{\ket{n m \sigma}\}$,
with spatial energies $E_{n m}=\omega (2n+|m|+1)$ and spin $\sigma\in\{\uparrow,\downarrow\}$.
For each \emph{principal} index $k=2n+|m|$ there are $(k{+}1)$ spatial states, each
twofold spin-degenerate. Filling from the bottom yields the \emph{closed-shell} electron numbers
\[
N_{\mathrm{closed}}=(K{+}1)(K{+}2), \quad K=0,1,2,\dots \quad (2,6,12,20,30,\dots).
\]
All other $N$ correspond to a \emph{partially filled} top shell (open shell), with a degenerate
noninteracting manifold.

\subsection{Spin structure and Slater form}
Write single-particle spin-orbitals as $\varepsilon_\alpha(x)=\phi_\alpha(\mathbf r)\,\chi_{\sigma}(s)$,
with $x=(\mathbf r,s)$. Because $\chi_\uparrow$ and $\chi_\downarrow$ are orthogonal and $H$ is
spin-independent, the Slater determinant simplifies:

\paragraph{Closed shells (RHF).}
For even $N$ with doubly occupied spatial orbitals
$\{\phi_1,\ldots,\phi_{N/2}\}$, the determinant factorizes into two identical spatial determinants,
one for each spin:
\[
\Psi_{\mathrm{closed}}(X)
=\frac{1}{(N/2)!}\;
\det[\phi_a(\mathbf r_i)]_{i\in\uparrow,a=1..N/2}\;
\det[\phi_a(\mathbf r_j)]_{j\in\downarrow,a=1..N/2}\;\times\; \Xi_{S=0,S_z=0},
\]
where $\Xi$ is the singlet spin function. This is the \emph{restricted} HF (RHF) reference and a
natural starting point for correlated ansätze.

% \paragraph{Open shells.}
% When the top shell is partially filled, the noninteracting ground space is degenerate.
% Physically relevant sectors are labeled by $(N,S,S_z,L_z)$. One may:
% (i) use ROHF (doubly occupied core $+$ singly occupied open-shell orbitals, spin-adapted to
% a chosen $(S,S_z)$), (ii) allow UHF (different spatial orbitals for $\uparrow,\downarrow$), or
% (iii) build a small multi-determinant reference within the top shell to enforce good $S$ and $L_z$.
% In our variational approach we keep a compact Slater reference (RHF for closed shells; ROHF/UHF
% or a minimal multi-determinant for open shells) and capture residual correlation with a Jastrow/PINN
% correlation factor.

% \subsection{Occupation representation (for later use)}
% Let $a^\dagger_{p\sigma}$ create $\ket{\phi_p\sigma}$. A closed shell is the bitstring
% $\prod_{p\le K\text{ shell}} a^\dagger_{p\uparrow} a^\dagger_{p\downarrow}\ket{0}$.
% Open shells occupy a subset of the top-shell spin–orbitals consistent with the target $(S,S_z)$;
% Hund-like choices (maximizing spin within the degenerate shell) often lower the mean-field energy
% via exchange and provide better trial nodes for correlated methods.

% \subsection{Why this split matters}
% Closed shells provide a unique, symmetry-singlet reference with well-conditioned SCF and typically
% smoother local energies. Open shells require an explicit choice of spin and angular-momentum sector
% and (if single-determinant) may have higher variance and more delicate conditioning; this motivates
% either spin-adapted references or adding a minimal set of top-shell configurations before applying
% our Slater–Jastrow/PINN optimization.
\subsection{Kato’s Cusp Conditions (2D Coulomb systems)}

For Coulomb interactions the potential diverges as $1/r$ at coalescence. Finiteness of the
local energy
\[
E_L(R)=\frac{(H\Psi)(R)}{\Psi(R)}
\]
requires that the kinetic singularity cancels the $1/r$ divergence. Kato’s theorem, first showed in \cite{Kato1957-EigenfunctionsManyParticle}, gives the
necessary short–distance behavior of the (spherically averaged) wavefunction. Let $r_{ij}$ be the
interparticle distance, $\mu_{ij}=m_i m_j/(m_i+m_j)$ the reduced mass (in a.u.), $Z_i$ the charge
number (electron $Z=-1$, nucleus $Z=+Z_A$), and $D$ the spatial dimension. Then
\[
\left.\frac{\partial}{\partial r_{ij}}\ln \overline{\Psi}\right|_{r_{ij}\to 0}
\;=\; \frac{2\,\mu_{ij}\,Z_i Z_j}{D-1}
\quad\text{(s-wave / opposite-spin channel)}.
\]
For \emph{identical fermions} the s-wave is excluded; the leading channel is $p$-wave and the
effective cusp is halved:
\[
\left.\frac{\partial}{\partial r_{ij}}\ln \overline{\Psi}\right|_{r_{ij}\to 0}
\;=\; \frac{\mu_{ij}\,Z_i Z_j}{D-1}
\quad\text{(like-spin)}.
\]

\paragraph{Specializing to 2D electrons (atomic units).}
With $m_e=1$ and $\mu_{ee}=1/2$:
\[
\boxed{\ a_{ee}^{\uparrow\downarrow}=\left.\frac{\partial}{\partial r_{ij}}\ln \overline{\Psi}\right|_{0}=1\ ,\qquad
a_{ee}^{\uparrow\uparrow}=\tfrac12\ .\ }
\]
For electron–nucleus coalescence ($\mu_{eA}\!\approx\!1$, $Z_i Z_j=-Z_A$):
\[
\boxed{\ a_{eA}=\left.\frac{\partial}{\partial r_{iA}}\ln \overline{\Psi}\right|_{0}=-2 Z_A\ .\ }
\]

\paragraph{Why this matters.}
These conditions ensure the kinetic energy’s $r^{-1}$ singularity cancels the Coulomb $r^{-1}$
term so that $E_L$ remains finite as particles coalesce. Enforcing the cusp explicitly improves conditioning.
\section{Structural diagnostics and Wigner markers}
\label{sec:analysis-struct}

Unless stated otherwise, curves are accumulated in \emph{trap units} 
$\tilde{\mathbf r}=\sqrt{\omega}\,\mathbf r$ so that length scales align across $\omega$, 
while headline distances are reported in \emph{Bohr} (physical units $\mathbf r$).
Production chains are thinned; we split samples into halves for stability checks 
(acceptance window, JSD between halves, reproducibility under restarts).

\subsection{Pair distribution $g(r)$, $P(r)$, and width markers}
\label{subsec:gr-pr}

We estimate the (spin-summed) pair distribution $g(r)$ via a shell‐area corrected histogram in trap units,
\begin{equation}
g(r_\alpha)\;=\;\frac{A_{\mathrm{eff}}}{N(N-1)}\;
\frac{\big\langle\sum_{i<j}\mathbf{1}\{r_{ij}\!\in[r_\alpha^-,r_\alpha^+)\}\big\rangle}
{2\pi r_\alpha\,\Delta r_\alpha},\qquad
r_{ij}=\|\tilde{\mathbf r}_i-\tilde{\mathbf r}_j\|_2,
\end{equation}
with $A_{\mathrm{eff}}=\pi R_{0.95}^2$ and $R_{0.95}$ the $95\%$ quantile of single‐particle radii (trap units).
We form the normalized radial probability in $d{=}2$,
\(
P(r)\propto r\,g(r),
\)
and report the mode $r_{\mathrm{mode}}$, mean $\bar r$, standard deviation $\sigma_r$, FWHM, and quantiles $q_{10},q_{50},q_{90}$ (in Bohr).
A compact width–to–scale marker is
\begin{equation}
\gamma \;=\; \sigma_r / r_{\mathrm{mode}},
\label{eq:gamma}
\end{equation}
which decreases as correlations stiffen (narrow distribution around a larger typical radius).\footnote{For $N{=}2$, $\gamma$ is complemented by the pair–distance Lindemann $\gamma_{r_{12}}=\mathrm{std}(r_{12})/\mathrm{mode}(r_{12})$, cf.\ Sec.~\ref{subsec:n2-scaling}.}

\paragraph{Sanity anchors at $\boldsymbol{\omega\!=\!10^{-3}}$.}
For $N{=}6$ and $N{=}12$, $P(r)$ exhibits sharp peaks with small $\gamma$; these are the expected Wigner-side shapes.\,
For $N{=}2$ we additionally report the near-origin mass 
$\int_0^{0.25\,r_{\mathrm{mode}}}\!\rho_1(r)\,dr$ and 
$\Pr(r_{12}\!\le 0.25\,\mathrm{mode}(r_{12}))$, both $\ll 1$ as required by the cusp and strong repulsion.

\subsection{Automatic shell detection and topology labeling}
\label{subsec:shells}

For each frame we sort single‐particle radii $\tilde s_1\le\cdots\le \tilde s_N$ (trap units),
compute gaps $g_k=\tilde s_{k+1}-\tilde s_k$, and flag a frame as \emph{two‐shell} if
\begin{equation}
\max_k g_k \;\ge\; \tau\,\mathrm{median}(g_k),
\qquad \tau\in[2.0,3.0] \ \text{(robust)}.
\end{equation}
The cut radius is the midpoint of the maximal gap; the inner multiplicity $n_{\mathrm{in}}$ defines the topology 
$(n_{\mathrm{in}}, N-n_{\mathrm{in}})$.
We report:
(i) the fraction of two‐shell frames,
(ii) the histogram over $n_{\mathrm{in}}$ among two‐shell frames, and
(iii) for selected topologies (e.g.\ $(1,5)$ or $(3,9)$) their fraction among \emph{all} frames.

\paragraph{II/IO/OO decomposition (what is being added).}
On two‐shell frames we decompose pairs into inner–inner (II), inner–outer (IO), and outer–outer (OO), 
accumulate $g_{\mathrm{II}},g_{\mathrm{IO}},g_{\mathrm{OO}}$, and reconstruct
\[
\tilde g(r)=w_{\mathrm{II}}g_{\mathrm{II}}+w_{\mathrm{IO}}g_{\mathrm{IO}}+w_{\mathrm{OO}}g_{\mathrm{OO}},
\]
with empirical weights from pair counts.
We report $\cos\angle(g,\tilde g)$ (cosine similarity); values $\approx 1$ indicate that the observed $g(r)$ is explained by shell geometry rather than sampling artifacts.

\paragraph{Sanity anchors at $\boldsymbol{\omega\!=\!10^{-3}}$.}
For $N{=}6$, $w_{\mathrm{OO}}\!:\!w_{\mathrm{IO}}\!\approx\!2\!:\!1$ and $\cos\angle(g,\tilde g)\!=\!1.000$ within numerical precision.
For $N{=}12$, the two‐shell fraction is $0.906$–$0.990$ across $\tau\in\{3.0,2.5,2.0\}$ and $\cos\angle(g,\tilde g)\!\in[0.989,1.000]$ depending mildly on $\tau$/seed.\,
The \emph{modal} inner multiplicity at $\omega\!=\!10^{-3}$ is $(3,9)$, while $(2,10)$ and $(1,11)$ retain substantial weight—consistent with a rotating two-shell Wigner molecule. 
For classical references to shellings and near-degeneracies in parabolic/cluster confinements, see
Schweigert–Peeters and Kong–Partoens–Peeters.\cite{schweigert1994spectralpropertieschargedparticles,Kong_2002}

\subsection{Orientational order and angular “Lindemann”}
\label{subsec:orient}

After recentring each detected ring (inner/outer) to remove global rotations, with angles $\{\phi_k\}_{k=1}^n$, we use the bond–orientational order
\begin{equation}
\Phi_m \;=\; \frac{1}{n}\sum_{k=1}^n e^{\,\mathrm{i} m \phi_k},
\qquad m=n,
\end{equation}
so that $|\Phi_m|\in[0,1]$ measures $n$‐fold orientational order (e.g.\ $m{=}5$ for $(1,5)$; $m{=}9$ for the outer ring of $(3,9)$).\,
As a complementary, dimensionless \emph{angular Lindemann} number we use
\begin{equation}
\lambda_\phi \;=\; \frac{\mathrm{std}(\Delta\phi)}{\mathbb{E}[\Delta\phi]},
\qquad 
\Delta\phi\;=\;\text{nearest‐neighbour angular spacings on the ring}.
\end{equation}
Smaller $\lambda_\phi$ indicates a stiffer (more Wigner‐like) ring.
This follows the spirit of bond–orientational order used in 2D Coulomb/Yukawa systems,\cite{Mazars_2008,Filinov_2001}
with the Lindemann idea adapted from melting criteria to angular spacings on a fixed-$n$ ring.

\paragraph{Sanity anchors at $\boldsymbol{\omega\!=\!10^{-3}}$.}
For $N{=}6$, phase-aligned $(1,5)$ and $(0,6)$ sectors show comparable order 
($|\Phi_5|\!\approx\!0.48$, $|\Phi_6|\!\approx\!0.52$) with $\lambda_\phi\!\approx\!0.29$—a single-ring rotating Wigner molecule.
For $N{=}12$, $(3,9)$ frames have strong order 
($|\Phi_3|\!\approx\!0.67$, $|\Phi_9|\!\approx\!0.50$) and small $\lambda_\phi$ ($\sim\!0.24$–$0.26$).
$(2,10)$ exhibits very tight inner order ($|\Phi_2|\!\approx\!0.73$) with a softer outer ring ($|\Phi_{10}|\!\approx\!0.42$), and $(1,11)$ has weak outer order and a softly vibrating center—precisely the \emph{rotating Wigner molecule} (RWM) picture of fluctuating orientations across near-degenerate shellings.\cite{Egger_1999,Manninen2007metalclustersquantumdots}

\subsection{Scaling check and two-body benchmarks ($N=2$)}
\label{subsec:n2-scaling}

To connect with the classical strong‐coupling limit we fit 
\begin{equation}
r_{\mathrm{mode}}(\omega) \;=\; C\,\omega^{\alpha}
\end{equation}
by linear regression in $\big(\log\omega,\log r_{\mathrm{mode}}\big)$ and compare with the two-body classical scaling $\alpha_{\mathrm{cl}}=-2/3$.
At $\omega\!=\!10^{-3}$ the $r_{12}$ histogram is sharply peaked and the near-origin mass in $\rho_1$ and $p(r_{12})$ is $\ll 1$, validating the cusp implementation while confirming deep Wigner localization (see also Ref.~\cite{Egger_1999} for the crossover criterion in dots).

\subsection{Reporting and stability checks}
\label{subsec:analysis-sanity}

For all scalars and histograms we report standard errors from block averages over thinned frames.
We verify: 
(i) acceptance in the target window,
(ii) Jensen–Shannon divergence and $\ell_2$ distance between split halves (stationarity),
(iii) robustness of topology fractions and $|\Phi_m|$ under $\tau\in[2.0,3.0]$ and mild pre‐smoothing of radii, and
(iv) reproducibility of $g(r)$, $P(r)$, and bond-order metrics under independent sampler restarts.
All diagnostics operate directly on saved bundles 
$\{\mathbf{X}_{\mathrm{Bohr}},\,r,\,g(r)\}$.

\paragraph{Interpretation guide (link to Results).}
Across $\omega\in\{1.0,0.5,0.1,0.01,10^{-3}\}$ these diagnostics map the fluid$\to$Wigner crossover: 
$\gamma$ and $\lambda_\phi$ shrink, shell gaps sharpen, and shell-resolved $|\Phi_m|$ becomes nonzero after phase alignment. 
For $N{=}6$ this manifests as a single-ring RWM with $(1,5)$–$(0,6)$ near-degeneracy; 
for $N{=}12$ as a two-shell RWM with modal $(3,9)$ but sizable $(2,10)$ and $(1,11)$ weights.
Classical shelling and near-degeneracies in finite Coulomb clusters\cite{schweigert1994spectralpropertieschargedparticles,Kong_2002}
and quantum Wigner markers in parabolic dots\cite{Egger_1999,Manninen2007metalclustersquantumdots,Filinov_2001,Mazars_2008}
provide the theoretical backdrop for reading these plots.

% ------------------------------------------------------------


%%%%%%%%%%%%%%%%%%%%%%%%%%%%%%%%%%%%%%%%%%%%%%%%%%%%%%%%%%%%%%%%%%%%%
\section{Variational Methods}

Quantum Monte Carlo (QMC) methods offer powerful tools for solving the many-body Schrödinger equation through stochastic sampling. These methods yield accurate ground-state properties by estimating expectation values of observables with controlled approximations. Among the various QMC techniques, Variational Monte Carlo (VMC) and Diffusion Monte Carlo (DMC) are particularly notable; VMC optimizes a trial wavefunction based on the variational principle, while DMC projects out the ground state through imaginary-time evolution \cite{Kroese2014-WhyMonteCarlo, KosztinFaberSchulten1996-IntroDMC}.

\subsection{Variational Monte Carlo (VMC)}
\subsubsection{The Variational Principle}
The variational principle guarantees that for any normalized trial wavefunction $\Psi_T(\mathbf{R}; \boldsymbol{\alpha})$, the expectation value
\[
E[\Psi_T] = \frac{\langle \Psi_T | \hat{H} | \Psi_T \rangle}{\langle \Psi_T | \Psi_T \rangle}
\]
provides an upper bound to the true ground-state energy $E_0$. Here, $\mathbf{R} = (\mathbf{r}_1,\dots,\mathbf{r}_N)$ and $\boldsymbol{\alpha}$ represents the set of variational parameters. Optimizing $\boldsymbol{\alpha}$ minimizes $E[\Psi_T]$.

\subsubsection{Sampling}
In VMC, the probability density is defined as
\[
P(\mathbf{R}) = \frac{|\Psi_T(\mathbf{R})|^2}{\int |\Psi_T(\mathbf{R})|^2\, d\mathbf{R}},
\]
and configurations are sampled using the Metropolis-Hastings algorithm, initially described in \cite{Metropolis1953}, with acceptance probability
\[
A(\mathbf{R} \to \mathbf{R}') = \min\!\left(1, \frac{|\Psi_T(\mathbf{R}')|^2}{|\Psi_T(\mathbf{R})|^2}\, \frac{T(\mathbf{R} \to \mathbf{R}')}{T(\mathbf{R}' \to \mathbf{R})}\right).
\]
A \emph{quantum force},
\[
\mathbf{F}(\mathbf{R}) = 2\, \nabla \ln \Psi_T(\mathbf{R}),
\]
further guides the sampling toward regions of high probability. This leads to what is known as importance sampling. A more precise discussion of this is found in \cite{Chin1990-QuadraticDMC,VanKampen1992-StochasticProcesses}.

\subsubsection{Local Energy and Parameter Optimization}
The local energy is defined as
\[
E_L(\mathbf{R}) = \frac{\hat{H}\Psi_T(\mathbf{R})}{\Psi_T(\mathbf{R})}.
\]
If $\Psi_T$ were exact, $E_L(\mathbf{R})$ would be constant. In practice, the variational energy is approximated by averaging $E_L$ over many configurations:
\[
E[\Psi_T] \approx \frac{1}{N_{\text{MC}}} \sum_{i=1}^{N_{\text{MC}}} E_L(\mathbf{R}_i).
\]
Variational parameters are then optimized (using gradient descent or stochastic reconfiguration) to reduce both the energy and its variance.

\subsection{Diffusion Monte Carlo (DMC)}
\subsubsection{Imaginary-Time Evolution and Projection}
DMC refines the trial wavefunction by projecting out the ground state through imaginary-time evolution. The Schrödinger equation becomes
\[
-\frac{\partial \Psi(\mathbf{R}, \tau)}{\partial \tau} = \Bigl(\hat{H} - E_T\Bigr)\Psi(\mathbf{R}, \tau),
\]
where $\tau = it$ and $E_T$ is a trial energy. As $\tau \to \infty$, $\Psi(\mathbf{R}, \tau)$ converges to the ground-state wavefunction provided the initial state has nonzero overlap with it.

\subsubsection{Stochastic Implementation}
DMC simulates the imaginary-time evolution using a combination of diffusion, drift, and branching:
\begin{enumerate}
    \item \textbf{Diffusion and Drift:} Particles execute random walks, with drift guided by the quantum force $\mathbf{F}(\mathbf{R})$, steering them toward regions where $\Psi_T$ is large.
    \item \textbf{Branching/Death:} Walkers are assigned a weight,
    \[
    w(\mathbf{R}, \Delta\tau) = \exp\!\Bigl[-\Bigl(E_L(\mathbf{R}) - E_T\Bigr)\Delta\tau\Bigr],
    \]
    which determines whether they are replicated or removed, ensuring that the walker distribution evolves toward the ground-state probability density.
\end{enumerate}

\subsection{Applications and Numerical Considerations}
\subsubsection{Model Systems}
QMC methods have been successfully applied to various quantum systems:
\begin{itemize}
    \item \textbf{Hydrogen Atom:} A simple trial wavefunction $\Psi_T(r)=e^{-\alpha r}$ yields the exact ground state for $\alpha=1$, illustrating zero variance.
    \item \textbf{Harmonic Traps:} In systems confined by harmonic potentials, Jastrow factors like $\exp[-ar_{ij}/(1+\beta r_{ij})]$ model inter-particle repulsion while Slater determinants maintain the proper fermionic symmetry.
\end{itemize}

\subsubsection{Computational Efficiency}
Key factors in achieving efficient QMC simulations include:
\begin{itemize}
    \item \textbf{Efficient Determinant Updates:} Algorithms for rapid updates of Slater determinants and their inverses can reduce computational cost from $\mathcal{O}(N^3)$.
    \item \textbf{Optimized Gradient and Laplacian Calculations:} Recursive methods and efficient numerical routines for evaluating $\nabla\Psi_T/\Psi_T$ and $\nabla^2\Psi_T/\Psi_T$ are essential.
\end{itemize}

\subsection{Conclusion}
Variational and Diffusion Monte Carlo methods together provide a robust framework for studying quantum systems. VMC offers a flexible approach for optimizing trial wavefunctions, while DMC projects out the exact ground state through imaginary-time evolution. These methods, working in concert, enable accurate and efficient investigations of systems from simple atoms to strongly correlated materials. Ongoing improvements in trial wavefunction design and stochastic algorithms continue to extend the reach of QMC in modern computational physics.

\chapter{Machine Learning}
\label{ch:ml_theory}

\section{Machine Learning as Statistical Learning}

Machine learning (ML) is best understood as the study of algorithms that approximate 
functions from data. At its core, it is not about ``intelligent machines,'' but about developing 
systematic ways to extract patterns from observations and to generalize these patterns to 
unseen situations. From a mathematical perspective, ML is a branch of 
\emph{statistical learning theory}, which views the learning process as the minimization 
of an expected error (or \emph{risk}) under uncertainty.  

Formally, consider an input space $\mathcal{X}\subseteq\mathbb{R}^d$ and an output 
space $\mathcal{Y}$, together with an unknown joint distribution $P(X,Y)$ over 
pairs $(x,y)$. The aim of learning is to construct a function 
$f_\theta:\mathcal{X}\to\mathcal{Y}$, parameterized by $\theta\in\Theta$, 
such that $f_\theta(x)$ predicts $y$ with small error. A loss function
\[
\ell:\mathcal{Y}\times\mathcal{Y}\to\mathbb{R}_{\geq 0}
\]
quantifies the discrepancy between prediction $f_\theta(x)$ and truth $y$. 
The central object of learning is then the \textbf{population risk}
\begin{equation}
  \mathcal{R}(\theta) = \mathbb{E}_{(x,y)\sim P}\big[\ell(f_\theta(x),y)\big].
\end{equation}
Since the distribution $P$ is unknown, we only have access to a finite dataset 
$D=\{(x_i,y_i)\}_{i=1}^n$. The corresponding \textbf{empirical risk} is
\begin{equation}
  \hat{\mathcal{R}}(\theta) = \frac{1}{n}\sum_{i=1}^n \ell(f_\theta(x_i),y_i).
\end{equation}
This setup is known as \emph{empirical risk minimization (ERM)}, and it underlies 
virtually all modern machine learning methods \cite{Berner_2022}.  

\subsection{Types of learning}
Depending on the information provided in the dataset, learning problems are 
traditionally categorized as:
\begin{itemize}
  \item \textbf{Supervised learning.} We are given labeled pairs $(x_i,y_i)$, and the goal 
  is to approximate the conditional expectation $\mathbb{E}[Y|X=x]$. Regression 
  (continuous $Y$) and classification (discrete $Y$) are canonical examples.  

  \item \textbf{Unsupervised learning.} Only input data $\{x_i\}$ is observed, without labels. 
  The aim is to uncover structure, such as clustering, dimensionality reduction, 
  or latent-variable representations.  

  \item \textbf{Reinforcement learning.} An agent interacts with an environment, receiving 
  observations and rewards. The task is to learn a policy that maximizes expected 
  cumulative reward.  
\end{itemize}
Although reinforcement learning plays a minor role in this thesis, supervised and 
unsupervised perspectives provide the theoretical foundation for the 
physics-informed approaches discussed later.  

\subsection{Generalization and the bias--variance tradeoff}
The essential challenge in ML is not just minimizing the empirical risk 
$\hat{\mathcal{R}}(\theta)$, but ensuring that this leads to a low population risk 
$\mathcal{R}(\theta)$ on unseen data. A model that simply memorizes the training set 
achieves low empirical risk but poor generalization --- a phenomenon known as 
\emph{overfitting}.  

Statistical learning theory formalizes this challenge via the 
\emph{bias--variance tradeoff}. Given a data-generating process and a learning algorithm, 
the expected squared error can be decomposed into
\begin{equation}
  \mathbb{E}\big[(f_\theta(x)-y)^2\big] 
  = \underbrace{(\mathbb{E}[f_\theta(x)]-f^\ast(x))^2}_{\text{bias}^2} 
  + \underbrace{\mathbb{E}[(f_\theta(x)-\mathbb{E}[f_\theta(x)])^2]}_{\text{variance}} 
  + \underbrace{\sigma^2}_{\text{irreducible noise}},
\end{equation}
where $f^\ast(x)$ is the true regression function and $\sigma^2$ the intrinsic 
noise of the data. Complex models tend to have low bias but high variance, while 
simple models have the opposite. Successful learning algorithms balance these two 
contributions. This is discussed in more detail in \cite{HastieTibshiraniFriedman2009-ESL}.

% \subsection{Statistical learning in the context of neural networks}
% For neural networks, the hypothesis space $\{f_\theta:\theta\in\Theta\}$ is vast 
% and highly expressive, often capable of interpolating the data exactly. Yet, 
% empirical evidence shows that heavily overparameterized networks can still generalize 
% well. This paradox has motivated intensive research into implicit regularization, 
% inductive biases, and the dynamics of optimization --- topics that will be central 
% in the subsequent sections.  

\section{Optimization}
\label{sec:ml_optimization}

The minimization of empirical risk,
\[
  \hat{\mathcal{R}}(\theta) = \frac{1}{n}\sum_{i=1}^n \ell(f_\theta(x_i),y_i),
\]
is an \emph{optimization problem}. The hypothesis class $\{f_\theta\}$ is usually 
nonlinear and high-dimensional, so closed-form solutions are not available. 
Instead, optimization is performed iteratively by updating parameters 
$\theta \in \mathbb{R}^p$ according to the geometry of the loss landscape.  

\subsection{Gradient descent}
The most basic algorithm is \emph{gradient descent}. Starting from an initial guess 
$\theta^{(0)}$, parameters are updated by
\begin{equation}
  \theta^{(t+1)} = \theta^{(t)} - \eta \,\nabla_\theta \hat{\mathcal{R}}(\theta^{(t)}),
\end{equation}
where $\eta>0$ is the \emph{learning rate}. Intuitively, the gradient points in the 
direction of steepest increase of the loss, so subtracting it decreases the loss most 
rapidly in a local sense.  

The choice of learning rate is crucial: if too small, convergence is slow; 
if too large, the algorithm may diverge.  

\subsection{Stochastic gradient descent (SGD)}
For large datasets, evaluating the full gradient at every step is prohibitively 
expensive. Instead, one uses \emph{stochastic gradient descent (SGD)}, which 
approximates the gradient on a randomly sampled mini-batch $B \subset D$:  
\begin{equation}
  \nabla_\theta \hat{\mathcal{R}}_B(\theta) 
  = \frac{1}{|B|}\sum_{(x_i,y_i)\in B} \nabla_\theta \ell(f_\theta(x_i),y_i).
\end{equation}
The update rule is then
\begin{equation}
  \theta^{(t+1)} = \theta^{(t)} - \eta \,\nabla_\theta \hat{\mathcal{R}}_B(\theta^{(t)}).
\end{equation}
SGD introduces randomness into the optimization trajectory. 
Paradoxically, this noise often \emph{helps}: it prevents the algorithm 
from getting trapped in poor local minima and can improve generalization 
through an implicit regularization effect \cite{wu2022alignmentpropertysgdnoise,Berner_2022}.  

\subsection{Momentum and adaptive methods}
Plain gradient descent treats all directions in parameter space equally.  
To accelerate convergence, one may use \emph{momentum}, maintaining a velocity 
vector $v_t$:
\begin{align}
  v_{t+1} &= \beta v_t + (1-\beta)\,\nabla_\theta \hat{\mathcal{R}}_B(\theta^{(t)}), \\
  \theta^{(t+1)} &= \theta^{(t)} - \eta v_{t+1},
\end{align}
where $\beta \in (0,1)$ controls the contribution of past gradients.  
Momentum smooths oscillations and allows the optimization to traverse 
valleys in the loss landscape more effectively.  

Beyond momentum, \emph{adaptive methods} such as RMSProp and Adam 
rescale learning rates per parameter based on historical gradient magnitudes \cite{KingmaBa2015-Adam}. 
For instance, Adam combines momentum with an adaptive normalization:  
\begin{align}
  m_t &= \beta_1 m_{t-1} + (1-\beta_1) g_t, \\
  v_t &= \beta_2 v_{t-1} + (1-\beta_2) g_t^2, \\
  \theta^{(t+1)} &= \theta^{(t)} - \eta \frac{\hat{m}_t}{\sqrt{\hat{v}_t} + \epsilon},
\end{align}
where $g_t=\nabla_\theta \hat{\mathcal{R}}_B(\theta^{(t)})$ and 
$\hat{m}_t, \hat{v}_t$ are bias-corrected versions of $m_t, v_t$.  
Adam is widely used in practice for its robustness across diverse problems \cite{ZhangEtAl2022-AdamConverges}.  

\subsection{Second-order methods and natural gradient}
First-order methods only use gradient information. In principle, convergence can be 
accelerated by incorporating curvature via the Hessian 
$H(\theta)=\nabla^2_\theta \hat{\mathcal{R}}(\theta)$.  
Newton’s method updates parameters as
\begin{equation}
  \theta^{(t+1)} = \theta^{(t)} - H(\theta^{(t)})^{-1}\,\nabla_\theta \hat{\mathcal{R}}(\theta^{(t)}).
\end{equation}
However, computing and inverting the Hessian is usually infeasible for large models.  

An attractive alternative is the \emph{natural gradient} \cite{Amari1998-NaturalGradient,AmariDouglas1998-WhyNaturalGradient}, which replaces the Euclidean 
metric with the geometry of probability distributions. The key object is the 
\emph{Fisher information matrix} (FIM),
\begin{equation}
  F(\theta) = \mathbb{E}\big[ \nabla_\theta \log p_\theta(x)\,\nabla_\theta \log p_\theta(x)^\top \big],
\end{equation}
where $p_\theta$ denotes the model distribution. The natural gradient update is
\begin{equation}
  \theta^{(t+1)} = \theta^{(t)} - \eta\, F(\theta^{(t)})^{-1}\,\nabla_\theta \hat{\mathcal{R}}(\theta^{(t)}).
\end{equation}
This preconditioning makes updates invariant to reparameterizations of $\theta$ 
and often improves conditioning of the optimization. In practice, computing 
$F^{-1}$ exactly is intractable, but approximations (e.g.\ stochastic reconfiguration 
in variational Monte Carlo, Kronecker-factored approximations in deep learning) 
make the method viable \cite{Pfau2020-FermiNet}. 

\subsection{Optimization challenges in deep learning}
Optimization in deep neural networks poses unique challenges:
\begin{itemize}
  \item \textbf{Non-convexity.} Loss landscapes are highly non-convex, with many local minima 
  and saddle points. Empirically, most local minima reached by SGD perform similarly well.  

  \item \textbf{Vanishing and exploding gradients.} In deep networks, gradients can shrink 
  or blow up during backpropagation. Proper initialization and activation choice are 
  essential to mitigate this problem.  

  \item \textbf{Flat vs.\ sharp minima.} Empirical evidence suggests that flat minima 
  (regions where the loss remains low under perturbations of $\theta$) lead to 
  better generalization. SGD’s noise implicitly biases optimization toward flatter minima.  
\end{itemize}

Optimization thus provides not only the means of fitting a model, but also shapes the 
inductive biases and generalization behavior of neural networks. In the following sections, 
we build on this foundation by examining neural network expressivity, activation functions, 
and initialization strategies in detail.

\section{Feed forward Neural Networks}
\label{sec:FFNN}

Activation functions are the nonlinear components of neural networks that 
enable them to approximate complex functions. Without nonlinearities, 
a stack of affine transformations would collapse to a single linear map, 
regardless of depth. The choice of activation function therefore plays a 
central role in the expressivity, optimization, and stability of a network.  

\subsection{From linear models to nonlinear function classes}
Classical regression models are linear in their parameters: 
\[
  f_\theta(x) = \theta^\top x.
\]
While simple and interpretable, linear models are limited to representing 
affine functions. Introducing basis expansions (e.g.\ polynomials, Fourier series, 
kernels) enlarges the hypothesis space, but requires manual design.  

Neural networks automate this process. A feed-forward neural network with $L$ layers is 
a composition of affine maps and nonlinear activation functions:
\begin{align}
  h^{(0)} &= x, \\
  h^{(\ell)} &= \sigma\!\left(W^{(\ell)} h^{(\ell-1)} + b^{(\ell)}\right), 
  \quad \ell=1,\dots,L-1, \\
  f_\theta(x) &= W^{(L)} h^{(L-1)} + b^{(L)}.
\end{align}
Here $\sigma:\mathbb{R}\to\mathbb{R}$ is a nonlinear activation. Without nonlinearity, 
the network collapses to a single linear transformation; with it, the space of representable 
functions expands dramatically.  

\subsection{Activation functions}
\subsubsection{Classical Activation Functions}

The earliest activation functions were smooth, saturating nonlinearities:
\begin{align}
  \sigma_{\text{sigmoid}}(x) &= \frac{1}{1+e^{-x}}, \\
  \sigma_{\tanh}(x) &= \tanh(x).
\end{align}
Both map $\mathbb{R}$ to a bounded range and are differentiable. 
The hyperbolic tangent is zero-centered, which often improves optimization.  

Despite their smoothness, these functions suffer from the 
\emph{vanishing gradient problem}. For large $|x|$, their derivatives 
approach zero, causing gradients to vanish during backpropagation in deep 
networks. As a result, learning becomes prohibitively slow.  


To address vanishing gradients, the \emph{Rectified Linear Unit} (ReLU) 
was introduced:
\begin{equation}
  \sigma_{\text{ReLU}}(x) = \max(0,x).
\end{equation}
ReLU is unbounded above and has derivative $1$ for positive inputs, 
avoiding gradient saturation. This makes deep networks with ReLU 
trainable even with many layers.  

However, ReLU introduces its own drawbacks:
\begin{itemize}
  \item Non-smoothness at $x=0$, which can lead to optimization instability 
  in contexts where derivatives or Laplacians are central (e.g.\ physics-informed models).  
  \item The \emph{dying ReLU} problem, where neurons with negative inputs 
  become permanently inactive.  
\end{itemize}

\subsubsection{Modern activations}
To combine the benefits of ReLU with smoothness, a number of 
\emph{smoothed rectifiers} have been proposed:
\begin{align}
  \sigma_{\text{SiLU}}(x) &= x \cdot \sigma_{\text{sigmoid}}(x), 
  &&\text{(Sigmoid Linear Unit)} \\
  \sigma_{\text{Swish}}(x) &= x \cdot \frac{1}{1+e^{-\beta x}}, 
  &&\text{(generalized SiLU with $\beta>0$)} \\
  \sigma_{\text{GELU}}(x) &= x \cdot \Phi(x), 
  &&\text{(Gaussian Error Linear Unit)} \\
  \sigma_{\text{Mish}}(x) &= x \cdot \tanh(\log(1+e^x)).
\end{align}
Here $\Phi(x)$ is the standard Gaussian cumulative distribution function.  

These activations are smooth ($C^\infty$) and non-saturating, 
preserving gradient flow while avoiding non-differentiable kinks.  
Empirically, GELU, SiLU/Swish, and Mish improve optimization stability 
and generalization across a wide range of tasks.  

% \subsection{Spectral bias and smoothness}
% Recent theory has highlighted the \emph{spectral bias} of gradient-based 
% training: neural networks tend to learn low-frequency components of a 
% function first, and high-frequency details later. The choice of activation 
% function influences this inductive bias.  

% Smooth activations (e.g.\ tanh, GELU, SiLU) yield better approximation 
% in Sobolev spaces, where not only function values but also derivatives 
% must be accurate. In contrast, piecewise-linear activations (ReLU) 
% are less suited when derivatives appear explicitly in the loss.  

% This distinction is crucial for physics-informed neural networks (PINNs), 
% which require stable computation of $\nabla f_\theta$ and $\Delta f_\theta$.  
% Using smooth activations ensures that gradients and Hessians are 
% well-defined and numerically stable, reducing variance in physical 
% losses involving differential operators.  

% \subsection{Choosing an activation in practice}
% Theoretical and empirical considerations suggest:
% \begin{itemize}
%   \item For generic tasks such as image classification, ReLU and its variants 
%   remain popular due to simplicity and efficiency.  
%   \item For scientific computing, PINNs, and variational wavefunction models, 
%   smooth activations such as SiLU, GELU, or Mish are strongly preferable.  
%   \item Tanh remains a robust choice when boundedness is desirable or when 
%   inputs are normalized to small ranges.  
% \end{itemize}

% The activation function thus encodes an implicit inductive bias, influencing 
% both optimization dynamics and the type of functions that can be efficiently 
% represented. In the next section, we turn to the equally critical issue of 
% \emph{weight initialization}, which determines how activations and gradients 
% propagate at the start of training.  


\subsection{Initialization}
\label{sec:initialization}

The initialization of neural network weights has a critical impact on 
optimization. At the start of training, parameter values determine how 
signals propagate forward and backward through the network. Poor initialization 
can lead to exploding or vanishing activations and gradients, making 
optimization unstable or ineffective. Proper schemes balance signal magnitudes 
across layers and preserve gradient flow, especially in deep networks.  

\subsubsection{The role of initialization}
Consider a feed-forward network with layers
\[
  h^{(\ell)} = \sigma\!\left(W^{(\ell)} h^{(\ell-1)} + b^{(\ell)}\right),
\]
where $W^{(\ell)}$ has entries drawn from some distribution.  
If weights are too large, activations grow exponentially with depth, 
leading to exploding gradients. If too small, activations shrink toward zero, 
causing vanishing gradients. Both phenomena impede learning.  

Initialization therefore aims to preserve the variance of activations 
and gradients across layers, such that
\[
  \mathrm{Var}[h^{(\ell)}] \approx \mathrm{Var}[h^{(\ell-1)}].
\]

\subsubsection{Xavier/Glorot initialization}
Glorot and Bengio (2010) proposed initializing weights with variance scaled 
by the number of input and output units:
\begin{equation}
  W_{ij}^{(\ell)} \sim \mathcal{U}\!\left(-\sqrt{\frac{6}{n_\text{in}+n_\text{out}}}, \;
  \sqrt{\frac{6}{n_\text{in}+n_\text{out}}}\right).
\end{equation}
Here $n_\text{in}$ and $n_\text{out}$ denote the fan-in and fan-out of a layer.  
This choice balances variance across layers for activations with 
approximately unit derivative (e.g.\ tanh).  

\subsubsection{Kaiming/He initialization}
For rectified activations such as ReLU, He et al.\ (2015) proposed 
\emph{Kaiming initialization}, scaling variance by $2/n_\text{in}$:
\begin{equation}
  W_{ij}^{(\ell)} \sim \mathcal{N}\!\left(0, \; \frac{2}{n_\text{in}}\right).
\end{equation}
This accounts for the fact that ReLU activations output zero half of the time, 
requiring larger variance to maintain signal propagation.  

\subsubsection{Gain and activation dependence}
Both Glorot and Kaiming schemes can be generalized with a \emph{gain} factor 
$g$ that depends on the activation function:
\[
  \mathrm{Var}[W_{ij}^{(\ell)}] = \frac{g^2}{n_\text{in}}.
\]
For example, $g=\sqrt{2}$ is recommended for ReLU, while $g=1$ is typical for 
tanh and smooth activations. Choosing the correct gain ensures that variance 
neither explodes nor decays through the network.  

\subsubsection{Orthogonal and random normal initialization}
An alternative approach is to initialize each weight matrix as an orthogonal 
matrix scaled by a variance factor. Orthogonal initialization preserves the 
norm of activations across layers, which can stabilize very deep networks.  

Simpler schemes such as sampling from a standard normal distribution without 
scaling are generally inferior: they do not respect the scaling needs of 
different architectures or activation functions, and often result in 
unstable training.  


% For physics-informed networks, stable gradient propagation is crucial: 
% losses often involve higher-order derivatives, and exploding Hessians 
% can render training unstable. Empirically, Glorot initialization with 
% a tuned gain has been found robust for smooth activations, while Kaiming 
% initialization is well-suited for ReLU-like functions.  

\subsection{Summary}
Initialization is particularly delicate in deep networks. Even small 
imbalances in variance can compound exponentially with depth. Smooth 
activations (SiLU, GELU, Mish) alleviate gradient instability but 
require careful scaling, as their effective slope differs from ReLU or tanh.  

Key takeaways include:
\begin{itemize}
  \item Proper scaling of variance prevents vanishing/exploding signals.  
  \item Initialization must match the activation function (Glorot for tanh/smooth, 
  Kaiming for ReLU).  
  \item Orthogonal schemes can improve stability in deep networks.  
  \item For PINNs and scientific ML, initialization is not merely a convenience 
  but essential for stable computation of derivatives.  
\end{itemize}

Initialization and activation choice thus form an inseparable pair: together 
they determine whether information propagates stably through the network. This will be discussed further in the context of physics-informed models in Chapter~\ref{ch:pinns}.


\section{Modern mathematics}
\label{sec:nn_approximation}

The central reason neural networks have become the dominant model class in machine 
learning is their \emph{expressivity}. However, in contrast with what classical statistical theory suggests, neural networks have been found to 
mitigate the curse of dimensionality, generalize well despite overparameterization, and
optimize effectively in highly non-convex landscapes. Understanding these phenomena
has been a major focus of recent theoretical research. In this section, we review key
results on the approximation capabilities of neural networks, with an emphasis on
implications for physical modeling.

\subsection{Universal approximation theorems}
Foundational results established that neural networks are \emph{universal approximators}. 
It has been proved that a feed-forward network with at least one 
hidden layer and a suitable nonlinear activation (e.g.\ sigmoid, ReLU, tanh) can approximate 
any continuous function on a compact set to arbitrary accuracy \cite{Cybenko1989,HORNIK1991251}. Formally:

\begin{theorem}[Universal Approximation, informal]
Let $f^\ast$ be continuous on a compact $K\subset\mathbb{R}^d$. 
Then for any $\epsilon>0$, there exists a neural network $f_\theta$ with a single hidden 
layer such that
\[
  \sup_{x\in K} |f_\theta(x) - f^\ast(x)| < \epsilon.
\]
\end{theorem}

This result guarantees existence but says nothing about how many neurons are required, 
how they should be trained, or how generalization occurs.  

\subsection{Depth versus width}

Cybenko's theorem establishes that a \emph{single} hidden layer with a
nonpolynomial sigmoidal activation is a \emph{universal approximator}:
for every continuous $f$ on a compact domain and every $\varepsilon>0$
there exists a one–hidden–layer network whose output $g$ satisfies
$\|f-g\|_\infty<\varepsilon$ \cite{Cybenko1989}. The result concerns
\emph{existence} of an approximant and does not provide constructive
bounds on the number of units required, rates of approximation, or
generalization guarantees.

In practice, depth is introduced for efficiency, not for universality. By composing intermediate features, additional layers often reduce the number of units required to attain a target accuracy. For certain target classes, there are known separations where a one-hidden-layer network needs exponentially many neurons to match the accuracy achievable with a deep architecture of polynomial size. Hence modern models leverage depth primarily for parameter efficiency and a stronger inductive bias (via hierarchical feature reuse), while width remains the lever for raw capacity \cite{Berner_2022}.
\subsection{Curse of dimensionality}

Classical approximation theory predicts that uniformly approximating a generic
$f:\mathbb{R}^d\!\to\!\mathbb{R}$ requires resources that grow exponentially in $d$—the
\emph{curse of dimensionality} (CoD). This perspective cannot explain the strong
empirical performance of deep networks on very high–dimensional tasks, prompting
a search for structural conditions under which the CoD does \emph{not} apply.:contentReference[oaicite:0]{index=0}

A first resolution is \emph{low effective dimension}: if the data (and loss) live on a
$d'\!\ll\!d$ manifold $M$ and error is measured in $L^p(\mu)$ with $\mathrm{supp}\,\mu\subset M$,
then neural networks can learn local charts and partitions of unity and achieve rates
that depend on $d'$ rather than the ambient $d$. This replaces ambient complexity
by the intrinsic geometry of $M$, yielding polynomial (rather than exponential)
scaling.

A second route is \emph{spectral/Barron regularity}: for targets with finite first Fourier
moment (Barron class), two–layer networks attain a \emph{dimension–independent}
$O(n^{-1/2})$ $L^2$–approximation error with $n$ hidden units. Deep and
compositional Barron spaces extend this beyond smooth settings, providing broad
families where the CoD provably disappears.

A third line exploits \emph{PDE structure}. For many high–dimensional linear and
semilinear parabolic/elliptic problems, stochastic representations (e.g., Feynman–Kac)
and NN emulations of time–stepping yield approximation bounds whose complexity
scales polynomially in $d$, again sidestepping the CoD under standard conditions.

Depth then clarifies \emph{why} these escapes are efficient. Certain functions admit
polynomial–size deep representations but require exponentially wide shallow ones
(Eldan–Shamir separations), and deep ReLU networks generate exponentially more
affine regions or oscillations per parameter than single–hidden–layer models. Thus,
depth supplies the compositional mechanism that turns structure (manifolds,
spectral regularity, PDE priors) into dimension–tolerant approximation.
\subsection{Generalization in overparameterized networks}
\label{sec:generalization}

Modern networks often have far more parameters than training samples, yet they fit the data \emph{and} perform well on new inputs. This looks paradoxical if we judge models only by parameter count. A better lens is to ask which interpolating solution the training dynamics actually pick: test error is governed less by how large the hypothesis class is and more by the \emph{kind} of function the algorithm converges to.

Two ideas make this concrete. First, stability: if small changes in the dataset produce only small changes in the learned predictor, the train–test gap remains small even at scale,
\[
\bigl|\mathcal{R}(\hat\theta)-\hat{\mathcal{R}}(\hat\theta)\bigr| \ \text{is controlled}.
\]
Second, implicit regularization: stochastic gradient methods (with finite batches, step sizes, and momentum) do not search the space uniformly; they are biased toward predictors with lower effective complexity—e.g., larger margins, smaller norms, or flatter basins in parameter space. In very wide nets, the dynamics are well-approximated by a linearized (kernel-like) model that interpolates while implicitly minimizing a function-space norm; at realistic widths, the same bias persists in practice. This also explains \emph{double descent}: near the interpolation threshold test error can worsen, then improves again as capacity grows and the dynamics have more low-complexity interpolants to select from.

\subsection{Optimization in deep networks}
\label{sec:optimization}

Training is a large-scale, nonconvex problem, yet simple first-order methods reliably drive the loss to (near) zero. Two features help. Overparameterization and modern architectures (residual connections, normalization) shape the landscape into broad, connected valleys of low loss and maintain usable descent directions through depth. As a result, poor isolated minima are less of a practical obstacle.

Stochastic gradient descent can be viewed as gradient flow with noise whose effective “temperature’’ is set mainly by the learning-rate–to–batch-size ratio. A higher temperature encourages escape from sharp minima and biases the search toward wider, flatter regions that tend to generalize better; annealing the temperature over time refines the final solution. Momentum accelerates along coherent directions and damps high-frequency noise; normalization keeps activations well-scaled so larger, stable steps are possible. In short, the optimizer does more than find zero training error—it determines \emph{which} zero-error solution you reach, and that choice largely sets generalization quality.

\subsection{Implications for physical modeling}
In physical sciences, target functions such as wavefunctions are continuous, smooth 
(except at singularities like cusps), and often structured by symmetries. 
Neural networks thus provide a flexible ansatz space:
\begin{enumerate}
  \item They can approximate smooth functions and their derivatives, which is crucial 
  when differential operators appear in the loss (as in PINNs).  
  \item Depth allows efficient capture of multi-scale and compositional structure, 
  relevant for many-body interactions.  
  \item Constraints such as permutation invariance or antisymmetry can be incorporated 
  to encode physics explicitly.  
\end{enumerate}

These theoretical guarantees justify the use of neural networks as universal, symmetry-aware 
function approximators, laying the groundwork for our discussions of activation functions, 
initialization strategies, and physics-informed extensions.  

\section{Physics-Informed Neural Networks}
\label{sec:pinns}

Traditional machine learning focuses on fitting models to labeled data. 
In scientific applications, however, governing equations such as differential 
equations encode a wealth of prior information about the system. 
\emph{Physics-Informed Neural Networks (PINNs)} integrate these constraints 
directly into the learning process, combining data-driven approximation 
with analytical structure.  Figure \ref{fig:activation_derivatives} illustrates how the behaviour of neural networks changes when working with higher order derivatives, as is common in PINNs. 
%this is a plot of the different activation functions for first, second and third order derivative showing
%how behaviour changes with PINNs, the file is runder all_activations_shared_legend.pdf

\begin{figure}[htbp]
  \centering
  \includegraphics[width=0.8\textwidth]{all_activations_shared_legend.pdf}
  \caption{Comparison of common activation functions and their first, second, 
  and third derivatives. Smooth activations (GELU, Mish) yield stable 
  derivatives, while Tanh and Sigmoid have less stable higher derivatives.}
  \label{fig:activation_derivatives}
\end{figure}

\subsection{Formulation of PINNs}
Let $u:\Omega \subset \mathbb{R}^d \to \mathbb{R}$ denote the unknown solution 
of a partial differential equation (PDE) of the form
\begin{equation}
  \mathcal{N}[u](x) = 0, \qquad x \in \Omega,
\end{equation}
subject to boundary conditions $u(x)=g(x)$ for $x\in\partial\Omega$.  
Here $\mathcal{N}$ is a (possibly nonlinear) differential operator.  

In a PINN, $u_\theta(x)$ is represented by a neural network, 
and the loss function enforces both data fitting and PDE residual minimization:
\begin{equation}
  \mathcal{L}(\theta) = 
  \lambda_\text{data} \sum_{i=1}^{n_d} |u_\theta(x_i)-y_i|^2
  + \lambda_\text{PDE} \sum_{j=1}^{n_r} |\mathcal{N}[u_\theta](\xi_j)|^2
  + \lambda_\text{BC} \sum_{k=1}^{n_b} |u_\theta(\zeta_k)-g(\zeta_k)|^2.
\end{equation}
Here $\{x_i,y_i\}$ are observed data, $\{\xi_j\}$ are collocation points in 
the domain for enforcing the PDE, and $\{\zeta_k\}$ are boundary points.  
The weights $\lambda_\text{data}, \lambda_\text{PDE}, \lambda_\text{BC}$ 
balance the different contributions.  

\subsection{Automatic differentiation}
A key ingredient of PINNs is the use of \emph{automatic differentiation} (AD) 
to evaluate $\nabla u_\theta$ and $\Delta u_\theta$. Unlike finite-difference 
approximations, AD computes derivatives exactly (up to floating-point error), 
making it feasible to incorporate higher-order differential operators directly 
into the loss. This enables end-to-end training of networks constrained by 
physical laws.  

\subsection{Theoretical difficulties}
Although PINNs are conceptually straightforward, making them reliable and scalable raises several non-trivial challenges that go beyond classical universal approximation results.

\paragraph{Approximation in Sobolev spaces.}
Universal approximation theorems control function error in $L^2$/$L^\infty$, but strong-form PINN losses depend on \emph{derivatives} through $L[u_\theta]$. 
To make residuals small one must approximate in matching Sobolev norms $W^{\ell,p}$ (with $\ell$ the PDE order), which imposes smooth activations (e.g., $\tanh$/SiLU for $\ell\!\ge\!2$) and architecture choices that control higher derivatives. 
Low-regularity targets (discontinuities, cusps) break global $W^{\ell,p}$ control and favor weak/variational or entropy-based formulations.

\paragraph{Stability (residual-to-error).}
A small residual does not imply a small solution error unless the forward problem is (conditionally) stable. 
For well-posed elliptic/parabolic problems one can bound $\|u-u_\theta\|$ by $\|L[u_\theta]-f\|$, but inverse/ill-posed settings may require regularization, priors, or restricted hypothesis classes to prevent ``residual overfitting'' with large state error.

\paragraph{Sampling and generalization.}
Collocation-based training minimizes a sampled surrogate of a continuum loss. 
Generalization requires capacity control (e.g., bounded weights/Lipschitz loss) and sufficient sampling of interior vs.\ boundary/IC points; otherwise the empirical residual can be small while the population residual remains large. 

\paragraph{Optimization and conditioning.}
PINN losses couple multiple physics terms (interior, boundary, data) with disparate scales, producing ill-conditioned gradient dynamics. 
In the linearized/NTK view, convergence speed is governed by the spectrum of a problem-dependent kernel; large condition numbers lead to slow or stalled training. 

\paragraph{Spectral bias.}
Gradient-based training tends to fit low-frequency components first, delaying convergence for solutions with sharp layers or oscillations unless the sampling, features, or architecture emphasize high-frequency content (e.g., Fourier features, windowed bases, curriculum).

\medskip
Mitigating these challenges requires careful design of architectures, activations, initialization, loss weighting, and sampling strategies. These topics are explained and discussed in detail in \cite{De_Ryck_2024}

\subsection{Relevance to this thesis}
For scientific machine learning, PINNs provide a principled framework 
for embedding physical structure into neural networks. In this thesis, 
they motivate the choice of smooth activations (for stable derivatives), 
careful initialization (to prevent exploding Hessians), and architectural 
constraints (to encode symmetry and physical laws).  
Thus, PINNs serve both as a methodological tool and as a theoretical lens 
for understanding the interplay between learning and physics.  

\subsection{Implications for scientific machine learning}
For physics-informed models, generalization takes on a slightly different meaning: 
the goal is not merely prediction on unseen data, but consistency with 
underlying physical laws. Inductive biases (symmetry, smoothness, conservation laws) 
are therefore not optional but essential. Implicit regularization from 
gradient-based optimization complements these biases by favoring solutions 
that are both smooth and physically plausible. Generalization in neural networks arises from an interplay of:
\begin{enumerate}
  \item Algorithmic stability of gradient-based methods,  
  \item Implicit bias toward simple, smooth, low-complexity solutions, and  
  \item Architectural inductive biases encoding domain knowledge.  
\end{enumerate}
In scientific computing, these mechanisms explain why neural networks, 
despite their size, can approximate physically meaningful solutions.  
They also justify the careful design choices of activations, initialization, 
and constraints emphasized throughout this chapter.  

\medskip
\noindent\textbf{Mitigation in practice.}
The challenges discussed for PINNs do not resolve themselves; they must be engineered away. 
In brief: use weak/variational formulations (or energy principles) when higher-order derivatives make strong forms unstable; 
hard–enforce essential boundary conditions and adapt weights for the rest; 
non-dimensionalize and balance loss terms to improve conditioning; 
prefer smooth activations and architectures that control derivatives; 
exploit symmetry, equivariance, and locality to reduce effective complexity; 
precondition optimization with second-order or natural-gradient methods (e.g., L\,-BFGS, Stochastic Reconfiguration); 
and design sampling (importance/stratified, curricula in time/space) to close the generalization gap. 
Together, these measures turn inductive bias into concrete numerical stability and reliable, law-consistent generalization.

\section{Summary and Outlook}
\label{sec:ml_summary}

In this chapter we have developed the theoretical foundations of modern 
machine learning, with particular emphasis on the aspects most relevant 
to scientific applications and physics-informed modeling.  

We began by framing machine learning as \emph{statistical learning}, 
where the goal is to minimize population risk via empirical risk minimization. 
This perspective highlights the central challenge of \emph{generalization}, 
formalized through the bias--variance tradeoff.  

We then reviewed the role of \emph{optimization}, from basic gradient descent 
to stochastic and adaptive methods, as well as the natural gradient. 
Optimization not only fits models to data but also imposes an implicit bias 
toward smooth and low-complexity solutions.  

The expressivity of neural networks was established through universal 
approximation theorems and the depth--width tradeoff, showing that deep, 
nonlinear architectures efficiently capture compositional and structured 
functions. Building on this, we examined two design choices that critically 
affect stability: \emph{activation functions} and \emph{weight initialization}. 
Smooth modern activations (e.g.\ SiLU, GELU, Mish) and principled 
initializations (Glorot, Kaiming, orthogonal) ensure stable signal and 
gradient propagation, which is essential for scientific applications 
requiring accurate derivatives.  

We introduced \emph{physics-informed neural networks (PINNs)} as a framework 
for embedding differential equations and physical constraints into learning. 
PINNs exemplify how universal approximation, smooth activations, and stable 
initialization combine to produce function approximators that are not only 
flexible but also consistent with governing laws.  

Finally, we discussed \emph{generalization mechanisms}, highlighting the 
importance of algorithmic stability, implicit regularization from gradient-based 
optimization, and inductive biases encoded by architecture. For physics, 
such biases often correspond to symmetries, conservation laws, or invariances 
that dramatically reduce hypothesis complexity.  

\paragraph{Outlook.} 
The theoretical principles presented here provide the foundation for the 
architectures and diagnostics developed in this thesis. As we turn to 
applications in quantum systems, these ideas will guide the design of 
networks that balance expressivity with stability, respect physical 
constraints, and generalize beyond the training domain.  
