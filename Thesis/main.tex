\documentclass{report}
\usepackage{graphicx}
\usepackage{caption}
\usepackage{subcaption}
\usepackage{float}  % Enables [H]
\usepackage{amsmath}
\usepackage{amssymb} % For advanced mathematical symbols
\usepackage{fancyhdr}
\usepackage{graphicx}
\usepackage[utf8]{inputenc}
\usepackage{booktabs}
\usepackage{bm}
\usepackage[numbers,sort&compress]{natbib} % in the preamble
%\usepackage{newpxtext,newpxmath} 
\usepackage[a4paper,margin=2.5cm]{geometry}
\usepackage{microtype}
\usepackage{titlesec}
\usepackage{xcolor}
\usepackage{physics}
\usepackage{braket}
\usepackage{tikz}

\usetikzlibrary{calc}
\usetikzlibrary{arrows.meta,positioning,fit}
\usepackage[flushleft]{threeparttable}  % provides the threeparttable and tablenotes environments

\usetikzlibrary{positioning, arrows.meta, shapes.geometric}

% Define your professional hex colors:
% ================== Colors (use your palette) ==================
% Base hex colors
\definecolor{mutedblue}{HTML}{1F77B4}
\definecolor{deepwine}{HTML}{A13333}
\definecolor{earthybrown}{HTML}{8c564b}
\definecolor{mutedpurple}{HTML}{9467bd}

% Lighter (50% white) versions for reduced saturation
\colorlet{mutedblueLight}{mutedblue!20!white}
\colorlet{deepwineLight}{deepwine!30!white}
\colorlet{lightearthybrown}{earthybrown!30!white}
\colorlet{lightmutedpurple}{mutedpurple!20!white}

% Optional aliases to keep earlier code working (map my soft names to yours)
\colorlet{softblue}{mutedblueLight}
\colorlet{softpurple}{lightmutedpurple}
\colorlet{softrose}{deepwineLight}
\colorlet{softsand}{lightearthybrown}
\colorlet{softteal}{mutedblueLight} % reuse light blue for pooling
% ================================================================
\definecolor{myaccent}{HTML}{a13333}

% put this in your preamble, after amsmath/amssymb
\newcommand{\RR}{\mathbb{R}}        % expectation
\newcommand{\EE}{\mathbb{E}}        % expectation

% expectation & probability

\newcommand{\PP}{\mathbb{P}}
\newcommand{\KL}{\mathrm{D}_{\mathrm{KL}}}

% variance & covariance
\newcommand{\Var}{\mathrm{Var}}
\newcommand{\Cov}{\mathrm{Cov}}

% bold vectors (uses bm package you already load)
\newcommand{\bx}{\bm{x}}
\newcommand{\bz}{\bm{z}}
\newcommand{\bw}{\bm{w}}
\newcommand{\btheta}{\bm{\theta}}

% distributions
\newcommand{\calN}{\mathcal{N}}   % normal distribution symbol

% convenience
\usepackage{amsthm}
\numberwithin{equation}{chapter}
\newtheorem{theorem}{Theorem}[chapter]
\newtheorem{lemma}[theorem]{Lemma}
\newtheorem{proposition}[theorem]{Proposition}



\titleformat{\chapter}
  [display]  % display = title is on its own line
  {\normalfont\huge\bfseries\color{myaccent}}  % Style
  {\chaptername\ \thechapter}  % "Chapter 1"
  {20pt}  % Space between number and title
  {}  % Code before the title (none)


\titleformat{\subsection}
  {\normalfont\large\bfseries\color{black}}{\thesubsection}{1em}{}
\titleformat{\section}
  {\normalfont\Large\bfseries}  % Style of the title
  {\thesection}{1em}            % Number formatting
  {\vspace{0.0em}}              % Space before title
  [\vspace{0.03em}{\color{myaccent}\rule{0.62\textwidth}{0.5pt}}]
    % Small rule below the title

\pagestyle{fancy}

\title{On the state of Many body quantum mechanics, and on PINNs}
\author{Aleksander Sekkelsten}

\date{\today}

\begin{document}

\maketitle
\tableofcontents
\part{Theory}

\graphicspath{{../results/figures/theory/}}
\chapter{Quantum Theory}
\label{sec:theory}

\section{Mathematical Foundations}
\subsection{Linear Algebra and Dirac Notation}
Quantum mechanics is formulated in complex vector spaces. A vector \(\ket{a}\) in an \(n\)-dimensional Hilbert space \(\mathcal{H}\) can be decomposed into a complete orthonormal basis \(\{\ket{i}\}\):  
\[
\ket{a} = \sum_i \ket{i}\braket{i|a} = \sum_i a_i \ket{i},
\]
where \(a_i = \braket{i|a}\) are complex coefficients. The completeness relation,  
\[
I = \sum_i \ket{i}\bra{i},
\]
ensures the basis spans \(\mathcal{H}\). Vectors in Dirac notation have dual "bra" counterparts \(\bra{a} = \ket{a}^\dagger\), enabling inner products \(\braket{a|b}\) and outer products \(\ket{a}\bra{b}\).

\subsection{Operators and Matrix Representations}
Linear operators \(\hat{\mathcal{O}}\) map vectors to vectors: \(\hat{\mathcal{O}}\ket{a} = \ket{b}\). Their adjoints \(\hat{\mathcal{O}}^\dagger\) satisfy \(\bra{a}\hat{\mathcal{O}}^\dagger = \bra{b}\). Hermitian operators (\(\hat{\mathcal{O}}^\dagger = \hat{\mathcal{O}}\)) represent observables, while unitary operators (\(\hat{\mathcal{O}}^\dagger = \hat{\mathcal{O}}^{-1}\)) preserve inner products. In a basis \(\{\ket{i}\}\), operators are represented by matrices:  
\[
O_{ij} = \bra{i}\hat{\mathcal{O}}\ket{j}.
\]
Non-commutativity of operators is quantified by the commutator:  
\[
[\hat{A}, \hat{B}] = \hat{A}\hat{B} - \hat{B}\hat{A}.
\]

\subsection{Hilbert Spaces and Function Representations}
Quantum states reside in infinite-dimensional Hilbert spaces, where vectors generalize to square-integrable functions. The inner product becomes:  
\[
\braket{\psi_i|\psi_j} = \int \psi_i^*(x)\psi_j(x) dx = \delta_{ij}.
\]
A function \(a(x)\) expands in a basis \(\{\psi_i(x)\}\):  
\[
a(x) = \sum_i \psi_i(x) a_i, \quad a_i = \int \psi_i^*(x)a(x) dx.
\]
The Dirac delta \(\delta(x-y)\) replaces the Kronecker delta in continuous spaces.

\section{Postulates of Quantum Mechanics}
\subsection{States and Observables}
A quantum state is a ray in Hilbert space, represented by a normalized ket \(\ket{\Psi}\). Observables are Hermitian operators with real eigenvalues. The time-independent Schrödinger equation,  
\[
\hat{H}\ket{\Psi} = E\ket{\Psi},
\]
determines stationary states, where \(\hat{H}\) is the Hamiltonian. Expectation values are computed via:  
\[
\braket{\hat{\mathcal{O}}} = \bra{\Psi}\hat{\mathcal{O}}\ket{\Psi}.
\]

\subsection{Composite Systems and Entanglement}
For \(N\) particles, the total Hilbert space is the tensor product of individual spaces:  
\[
\mathcal{H} = \bigotimes_{i=1}^N \mathcal{H}_i.
\]
Entangled states cannot be written as a single tensor product, e.g., \(\ket{\Psi} \neq \ket{\psi_1} \otimes \ket{\psi_2}\). Fermionic systems require antisymmetric wavefunctions under particle exchange:  
\[
\Psi(\mathbf{x}_1, \dots, \mathbf{x}_N) = \frac{1}{\sqrt{N!}} \det[\phi_i(\mathbf{x}_j)],
\]
where \(\mathbf{x}_i = (\bm{r}_i, \sigma_i)\) includes spatial (\(\bm{r}_i\)) and spin (\(\sigma_i\)) coordinates.

\section{Many-Body Systems and Second Quantization}
\subsection{Challenges of First Quantization}
Solving the Schrödinger equation for interacting particles becomes intractable due to the factorial growth of Slater determinants. For \(N\) particles in \(M\) orbitals, the Hilbert space dimension scales as \(\binom{M}{N}\).

\subsection{Fock Space and Occupation Representation}
Second quantization simplifies this by using Fock space \(\mathcal{F} = \bigoplus_{n=0}^\infty \mathcal{H}^{\otimes n}\), where states are labeled by occupation numbers:  
\[
\ket{n_1, n_2, \dots} \quad (n_i = 0 \text{ or } 1 \text{ for fermions}).
\]
Creation (\(a_i^\dagger\)) and annihilation (\(a_i\)) operators enforce (anti)commutation relations:  
\[
\{a_i, a_j^\dagger\} = \delta_{ij}, \quad \{a_i, a_j\} = 0 \quad \text{(fermions)}.
\]
The Hamiltonian in second quantization is:  
\[
\hat{H} = \sum_{ij} t_{ij} a_i^\dagger a_j + \frac{1}{4} \sum_{ijkl} V_{ijkl} a_i^\dagger a_j^\dagger a_k a_l,
\]
where \(t_{ij}\) and \(V_{ijkl}\) encode kinetic and interaction terms.

\section{Model Systems}
\subsection{One-Dimensional Spinless Fermions}
For \(N\) fermions in a harmonic trap with Gaussian interactions:  
\[
\hat{H} = \sum_{i=1}^N \left(-\frac{1}{2} \nabla_i^2 + \frac{1}{2} x_i^2 \right) + \frac{V_0}{\sigma_0 \sqrt{2\pi}} \sum_{i<j} e^{-(x_i - x_j)^2/(2\sigma_0^2)}.
\]
\begin{itemize}
    \item \textbf{Non-interacting limit:} The ground state is a Slater determinant of Hermite polynomials:  
    \[
    \phi_n(x) = \frac{H_n(x)e^{-x^2/2}}{\sqrt{2^n n! \sqrt{\pi}}}.
    \]
    Energy: \(E_{\text{gs}} = \frac{N^2}{2}\) (in \(\hbar\omega\) units).
    \item \textbf{Interacting case:} Attractive (\(V_0 < 0\)) or repulsive (\(V_0 > 0\)) interactions modify spatial correlations and energy spectra (see Fig.~\ref{fig:interactions}).
\end{itemize}

\subsection{Two-Dimensional Quantum Dots}
For fermions in a 2D harmonic trap with Coulomb repulsion:  
\[
\hat{H} = \sum_{i=1}^N \left(-\frac{1}{2} \nabla_i^2 + \frac{1}{2} \omega^2 r_i^2 \right) + \sum_{i<j} \frac{1}{|\bm{r}_i - \bm{r}_j|}.
\]
\begin{itemize}
    \item \textbf{Closed-shell configurations:} Magic numbers \(N = 2\binom{n+2}{2}\) arise from filling degenerate orbitals with spin-up/down pairs.
    \item \textbf{Trap frequency \(\omega\):} Lower \(\omega\) increases spatial spread and enhances correlation effects (Fig.~\ref{fig:omega_dependence}).
\end{itemize}


%%%%%%%%%%%%%%%%%%%%%%%%%%%%%%%%%%%%%%%%%%%%%%%%%%%%%%%%%%%%%%%%%%%%%

\section{Hartree Fock}
Hartree-Fock (HF) theory is a fundamental \emph{ab initio} method in quantum many-body physics. It approximates the true many-body wavefunction by a single Slater determinant, ensuring the correct antisymmetry under particle exchange. By applying the variational principle to minimize the energy, HF produces a self-consistent mean-field solution that serves as the basis for more sophisticated post-HF methods such as Configuration Interaction and Coupled Cluster theory.

\subsection{Second Quantization Primer}
In second quantization, the many-body state is described using creation and annihilation operators. Let $\{\ket{\alpha}\}$ be an orthonormal single-particle basis. The fermionic operators $a_\alpha^\dagger$ and $a_\alpha$ satisfy
\[
\{a_\alpha, a_\beta^\dagger\} = \delta_{\alpha\beta}, \quad \{a_\alpha, a_\beta\} = \{a_\alpha^\dagger, a_\beta^\dagger\} = 0.
\]
These operators act on the vacuum state $\ket{0}$, allowing one to build Slater determinants that automatically obey the Pauli exclusion principle.

\subsection{Many-Fermion Hamiltonian}
In the second-quantized formalism, the Hamiltonian for interacting fermions is expressed as
\[
\hat{H} = \sum_{\alpha\beta} \langle \alpha | \hat{h}_0 | \beta \rangle\, a_\alpha^\dagger a_\beta + \frac{1}{4} \sum_{\alpha\beta\gamma\delta} \langle \alpha \beta | \hat{v} | \gamma \delta \rangle_{\!AS}\, a_\alpha^\dagger a_\beta^\dagger a_\delta a_\gamma,
\]
where $\hat{h}_0$ includes kinetic and external potential energy, and $\langle \alpha \beta | \hat{v} | \gamma \delta \rangle_{\!AS}$ represents the antisymmetrized two-body matrix elements.

\subsection{Variational Principle and HF Energy Functional}
In HF theory, one minimizes the energy functional
\[
E[\Phi] = \langle \Phi | \hat{H} | \Phi \rangle,
\]
where $\ket{\Phi}$ is a single Slater determinant:
\[
\ket{\Phi} = \prod_{i=1}^N a_i^\dagger \ket{0}.
\]
For an $N$-particle system, the energy can be written as
\[
E[\Phi] = \sum_{i=1}^N \langle i | \hat{h}_0 | i \rangle + \frac{1}{2}\sum_{i,j=1}^N \langle ij | \hat{v} | ij \rangle_{\!AS}.
\]

\subsection{Derivation of the Fock Matrix}
To optimize the single-particle orbitals, one performs a unitary transformation among the basis states:
\[
a_p^\dagger = \sum_\lambda C_{p\lambda}\, c_\lambda^\dagger,
\]
with the coefficients $C_{p\lambda}$ forming a unitary matrix. This leads to the definition of the \emph{Fock operator}:
\[
\hat{h}^{HF}_{\alpha\beta} = \langle \alpha | \hat{h}_0 | \beta \rangle + \sum_{\gamma\delta} \rho_{\gamma\delta}\, \langle \alpha \gamma | \hat{v} | \beta \delta \rangle_{\!AS},
\]
where the one-particle density matrix is
\[
\rho_{\gamma\delta} = \sum_{i=1}^N C_{\gamma i}\, C_{\delta i}^*.
\]
The HF equations are then obtained by solving the eigenvalue problem:
\[
\sum_\beta \hat{h}^{HF}_{\alpha\beta} C_{p\beta} = \epsilon_p^{HF} C_{p\alpha}.
\]

\subsection{Self-Consistent Field Procedure}
Since the Fock operator depends on the density matrix (which in turn depends on the orbital coefficients), the HF equations must be solved iteratively:
\begin{enumerate}
    \item \textbf{Initialization:} Start with an initial guess for the density matrix $\rho^{(0)}$.
    \item \textbf{Construct the Fock Matrix:} Use $\rho^{(n)}$ to build $\hat{h}^{HF}$.
    \item \textbf{Diagonalize:} Solve $\hat{h}^{HF} C = \epsilon^{HF} C$ to obtain new orbital coefficients.
    \item \textbf{Update:} Form a new density matrix $\rho^{(n+1)}$ from the occupied orbitals.
    \item \textbf{Convergence Check:} Compare the HF energy or density matrix between iterations and repeat until convergence.
\end{enumerate}


\section{Full Configuration Interaction Theory}

Full Configuration Interaction (FCI) theory provides the exact solution to the many-body Schrödinger equation within a finite single-particle basis by incorporating all possible excitations of $N$ fermions among $M$ orbitals. This method captures every correlation effect, making it the benchmark against which approximate methods are measured. However, the combinatorial growth of the determinant space limits FCI to small systems or necessitates the use of symmetry reductions.

\subsection{Slater Determinants in Second Quantization}
A single Slater determinant for $N$ fermions is written as
\[
\ket{\Phi_0} = \prod_{i=1}^N a_i^\dagger \ket{0}.
\]
The antisymmetry is guaranteed by the fermionic anticommutation relations:
\[
\{a_p^\dagger, a_q^\dagger\} = 0, \quad \{a_p, a_q\} = 0, \quad \{a_p^\dagger, a_q\} = \delta_{pq}.
\]

\subsection{Exact Wavefunction Expansion}
Unlike HF, FCI expands the ground state in a complete basis of Slater determinants:
\[
\ket{\Psi_0} = \sum_{PH} C_H^P \ket{\Phi_H^P} = \Bigl(1 + \hat{C}\Bigr) \ket{\Phi_0},
\]
where $\ket{\Phi_H^P}$ represents determinants with particle-hole excitations relative to the reference $\ket{\Phi_0}$, and $\hat{C}$ is the correlation operator:
\[
\hat{C} = \sum_{ai} C_i^a\, a_a^\dagger a_i + \sum_{abij} C_{ij}^{ab}\, a_a^\dagger a_b^\dagger a_j a_i + \dots.
\]

\subsection{Variational Principle and the FCI Matrix}
Applying the variational principle,
\[
\langle \delta\Psi_0 | (\hat{H} - E) | \Psi_0 \rangle = 0,
\]
leads to the eigenvalue problem
\[
\sum_{P'H'} \langle \Phi_H^P | \hat{H} | \Phi_{H'}^{P'} \rangle C_{H'}^{P'} = E\, C_H^P.
\]
The Hamiltonian matrix is block-structured according to excitation levels (e.g., $0p0h$, $1p1h$, $2p2h$, \dots). Table~\ref{tab:fci_matrix} summarizes the schematic structure.

\begin{table}[h]
\centering
\begin{tabular}{c|cccc}
 & $0p0h$ & $1p1h$ & $2p2h$ & $\dots$ \\
\hline
$0p0h$ & $\langle 0|\hat{H}|0 \rangle$ & $\langle 0|\hat{H}|1p1h \rangle$ & $\langle 0|\hat{H}|2p2h \rangle$ & $\cdots$ \\
$1p1h$ & $\langle 1p1h|\hat{H}|0 \rangle$ & $\langle 1p1h|\hat{H}|1p1h \rangle$ & $\langle 1p1h|\hat{H}|2p2h \rangle$ & $\cdots$ \\
$2p2h$ & $\langle 2p2h|\hat{H}|0 \rangle$ & $\langle 2p2h|\hat{H}|1p1h \rangle$ & $\langle 2p2h|\hat{H}|2p2h \rangle$ & $\cdots$ \\
$\vdots$ & $\vdots$ & $\vdots$ & $\vdots$ & $\ddots$ \\
\end{tabular}
\caption{Schematic structure of the FCI Hamiltonian matrix in the determinant basis.}
\label{tab:fci_matrix}
\end{table}

\subsection{Exponential Growth and Computational Limits}
The dimension of the determinant space is given by
\[
\text{Dim} = \binom{M}{N},
\]
which increases combinatorially with system size. For example, for ${}^{16}\mathrm{O}$ with $N=8$ valence nucleons in a basis of $M=40$ states, $\binom{40}{8} \approx 10^9$. This rapid growth limits the direct application of FCI to small systems or requires symmetry constraints to reduce the effective space.

\subsection{Correlation Energy}
The correlation energy, defined as
\[
\Delta E = E_{\mathrm{FCI}} - E_{\mathrm{HF}},
\]
quantifies the many-body effects missing in the HF approximation. Although small compared to the total energy, $\Delta E$ is essential for accurate predictions and is expressed in terms of the excitation amplitudes (e.g., $C_{ij}^{ab}$).

\subsection{Practical Considerations and Approximations}
Due to the exponential scaling, approximate methods are often employed:
\begin{itemize}
    \item \textbf{Truncated CI:} Restrict excitations to singles and doubles (CISD), etc.
    \item \textbf{Coupled Cluster (CC):} Use an exponential ansatz $\exp(\hat{T})$ to incorporate higher excitations efficiently.
    \item \textbf{Monte Carlo Methods:} Techniques like FCI Quantum Monte Carlo (FCIQMC) sample the determinant space stochastically.
\end{itemize}

\subsection{Conclusion}
FCI provides an exact solution (within a finite basis) to the many-body Schrödinger equation by accounting for all possible correlations. Although computationally demanding, it serves as a critical benchmark for approximate methods, underpinning modern approaches in quantum chemistry and many-body physics.

%%%%%%%%%%%%%%%%%%%%%%%%%%%%%%%%%%%%%%%%%%%%%%%%%%%%%%%%%%%%%%%%%%%%%
\section{Variational Methods}

Quantum Monte Carlo (QMC) methods offer powerful tools for solving the many-body Schrödinger equation through stochastic sampling. These methods yield accurate ground-state properties by estimating expectation values of observables with controlled approximations. Among the various QMC techniques, Variational Monte Carlo (VMC) and Diffusion Monte Carlo (DMC) are particularly notable; VMC optimizes a trial wavefunction based on the variational principle, while DMC projects out the ground state through imaginary-time evolution.

\subsection{Variational Monte Carlo (VMC)}
\subsubsection{The Variational Principle}
The variational principle guarantees that for any normalized trial wavefunction $\Psi_T(\mathbf{R}; \boldsymbol{\alpha})$, the expectation value
\[
E[\Psi_T] = \frac{\langle \Psi_T | \hat{H} | \Psi_T \rangle}{\langle \Psi_T | \Psi_T \rangle}
\]
provides an upper bound to the true ground-state energy $E_0$. Here, $\mathbf{R} = (\mathbf{r}_1,\dots,\mathbf{r}_N)$ and $\boldsymbol{\alpha}$ represents the set of variational parameters. Optimizing $\boldsymbol{\alpha}$ minimizes $E[\Psi_T]$.

\subsubsection{Trial Wave Function and Importance Sampling}
A successful trial wavefunction for fermions combines a Slater determinant with a Jastrow factor:
\[
\Psi_T(\mathbf{R}) = \mathcal{D}^\uparrow(\mathbf{R})\, \mathcal{D}^\downarrow(\mathbf{R}) \prod_{i<j} \exp\!\bigl[f(r_{ij})\bigr],
\]
where $\mathcal{D}^\sigma$ ensures antisymmetry and $f(r_{ij})$ captures dynamical correlations. In VMC, the probability density is defined as
\[
P(\mathbf{R}) = \frac{|\Psi_T(\mathbf{R})|^2}{\int |\Psi_T(\mathbf{R})|^2\, d\mathbf{R}},
\]
and configurations are sampled using the Metropolis-Hastings algorithm with acceptance probability
\[
A(\mathbf{R} \to \mathbf{R}') = \min\!\left(1, \frac{|\Psi_T(\mathbf{R}')|^2}{|\Psi_T(\mathbf{R})|^2}\, \frac{T(\mathbf{R} \to \mathbf{R}')}{T(\mathbf{R}' \to \mathbf{R})}\right).
\]
A \emph{quantum force},
\[
\mathbf{F}(\mathbf{R}) = 2\, \nabla \ln \Psi_T(\mathbf{R}),
\]
further guides the sampling toward regions of high probability.

\subsubsection{Local Energy and Parameter Optimization}
The local energy is defined as
\[
E_L(\mathbf{R}) = \frac{\hat{H}\Psi_T(\mathbf{R})}{\Psi_T(\mathbf{R})}.
\]
If $\Psi_T$ were exact, $E_L(\mathbf{R})$ would be constant. In practice, the variational energy is approximated by averaging $E_L$ over many configurations:
\[
E[\Psi_T] \approx \frac{1}{N_{\text{MC}}} \sum_{i=1}^{N_{\text{MC}}} E_L(\mathbf{R}_i).
\]
Variational parameters are then optimized (using gradient descent or stochastic reconfiguration) to reduce both the energy and its variance.

\subsection{Diffusion Monte Carlo (DMC)}
\subsubsection{Imaginary-Time Evolution and Projection}
DMC refines the trial wavefunction by projecting out the ground state through imaginary-time evolution. The Schrödinger equation becomes
\[
-\frac{\partial \Psi(\mathbf{R}, \tau)}{\partial \tau} = \Bigl(\hat{H} - E_T\Bigr)\Psi(\mathbf{R}, \tau),
\]
where $\tau = it$ and $E_T$ is a trial energy. As $\tau \to \infty$, $\Psi(\mathbf{R}, \tau)$ converges to the ground-state wavefunction provided the initial state has nonzero overlap with it.

\subsubsection{Stochastic Implementation}
DMC simulates the imaginary-time evolution using a combination of diffusion, drift, and branching:
\begin{enumerate}
    \item \textbf{Diffusion and Drift:} Particles execute random walks, with drift guided by the quantum force $\mathbf{F}(\mathbf{R})$, steering them toward regions where $\Psi_T$ is large.
    \item \textbf{Branching/Death:} Walkers are assigned a weight,
    \[
    w(\mathbf{R}, \Delta\tau) = \exp\!\Bigl[-\Bigl(E_L(\mathbf{R}) - E_T\Bigr)\Delta\tau\Bigr],
    \]
    which determines whether they are replicated or removed, ensuring that the walker distribution evolves toward the ground-state probability density.
\end{enumerate}

\subsection{Applications and Numerical Considerations}
\subsubsection{Model Systems}
QMC methods have been successfully applied to various quantum systems:
\begin{itemize}
    \item \textbf{Hydrogen Atom:} A simple trial wavefunction $\Psi_T(r)=e^{-\alpha r}$ yields the exact ground state for $\alpha=1$, illustrating zero variance.
    \item \textbf{Harmonic Traps:} In systems confined by harmonic potentials, Jastrow factors like $\exp[-ar_{ij}/(1+\beta r_{ij})]$ model inter-particle repulsion while Slater determinants maintain the proper fermionic symmetry.
\end{itemize}

\subsubsection{Computational Efficiency}
Key factors in achieving efficient QMC simulations include:
\begin{itemize}
    \item \textbf{Efficient Determinant Updates:} Algorithms for rapid updates of Slater determinants and their inverses can reduce computational cost from $\mathcal{O}(N^3)$.
    \item \textbf{Optimized Gradient and Laplacian Calculations:} Recursive methods and efficient numerical routines for evaluating $\nabla\Psi_T/\Psi_T$ and $\nabla^2\Psi_T/\Psi_T$ are essential.
\end{itemize}

\subsection{Conclusion}
Variational and Diffusion Monte Carlo methods together provide a robust framework for studying quantum systems. VMC offers a flexible approach for optimizing trial wavefunctions, while DMC projects out the exact ground state through imaginary-time evolution. These methods, working in concert, enable accurate and efficient investigations of systems from simple atoms to strongly correlated materials. Ongoing improvements in trial wavefunction design and stochastic algorithms continue to extend the reach of QMC in modern computational physics.

\chapter{Machine Learning}
\label{sec:theory}

\section{Elementary Theory}
\label{sec:ml_concepts}

In this chapter, we systematically synthesize the core concepts in three interconnected pillars of machine learning: statistical learning theory, optimization methods, and neural network architectures. This chapter progresses from fundamental principles to advanced techniques, maintaining mathematical rigor while enhancing pedagogical clarity through explicit connections between theoretical concepts and their practical implementations.

\subsection{Statistical Learning and Regression Analysis}
\label{subsec:stat_learning}

Statistical learning theory establishes the formal framework for constructing predictive models through data-driven inference. Consider a supervised learning scenario with a dataset containing $N$ independent observations:

\[
\mathcal{D} = \{(x_i, y_i)\}_{i=1}^{N},
\]

where feature vectors $x_i \in \mathbb{R}^{d}$ are associated with targets $y_i \in \mathbb{R}$ (or $\mathbb{R}^k$ for multi-output problems). The data-generating process is modeled as:

\begin{equation}
y_i = f_\theta(x_i) + \epsilon_i, \quad \epsilon_i \sim \mathcal{N}(0, \sigma^2), \quad i=1,\ldots,N,
\label{eq:regression_model}
\end{equation}

where $f_\theta:\mathbb{R}^{d}\to\mathbb{R}$ represents a parameterized hypothesis function and $\epsilon_i$ captures irreducible measurement noise. The fundamental objective is to estimate parameters $\theta$ that minimize the expected prediction risk:

\begin{equation}
\mathcal{R}(\theta) = \mathbb{E}_{(x,y)\sim P_{\text{data}}} \left[\mathcal{L}\left(y, f_\theta(x)\right)\right],
\label{eq:true_risk}
\end{equation}

where $P_{\text{data}}$ is the unknown data distribution. In practice, we employ empirical risk minimization (ERM) using the available samples:

\begin{equation}
\hat{\mathcal{R}}(\theta) = \frac{1}{N}\sum_{i=1}^{N} \mathcal{L}\left(y_i, f_\theta(x_i)\right),
\label{eq:empirical_risk}
\end{equation}

where $\mathcal{L}:\mathbb{R}\times\mathbb{R}\to\mathbb{R}_{\ge 0}$ is typically chosen as the Mean Squared Error (MSE) for regression tasks:

\[
\mathcal{L}_{\text{MSE}}(y, \hat{y}) = \frac{1}{2}(y - \hat{y})^2.
\]

The optimization landscape defined by $\hat{\mathcal{R}}(\theta)$ forms the basis for model training through gradient-based methods, which we subsequently analyze.

\subsection{Gradient-Based Optimization Methods}
\label{subsec:optimization}

The minimization of empirical risk requires efficient navigation of high-dimensional parameter spaces. We analyze three fundamental optimization paradigms with increasing complexity.

\section{Optimization Schemes in Machine Learning}

Optimization is central to training machine learning models, where the goal is to minimize a loss function \( \mathcal{L}(\theta) \) with respect to the model parameters \(\theta\). Different optimization algorithms offer various trade-offs between computational efficiency, convergence speed, and robustness to the loss landscape’s geometry. In this section, we discuss several key methods: basic Gradient Descent (GD), Stochastic Gradient Descent (SGD), and enhancements such as momentum, RMSProp, and ADAM. We also briefly mention the intuition behind second-order methods like Newton's method, even though we do not use Hessians directly in our work.

\subsection{Gradient Descent (GD)}
The simplest optimization method is full-batch Gradient Descent, where the parameters are updated in the direction of the negative gradient:
\begin{equation}
\theta_{t+1} = \theta_t - \eta \nabla_\theta \mathcal{L}(\theta_t),
\label{eq:gd_update}
\end{equation}
where \(\eta > 0\) is the learning rate. GD uses the entire dataset to compute the gradient, ensuring a stable but computationally expensive update, especially for large datasets.

\subsection{Stochastic Gradient Descent (SGD)}
SGD mitigates the computational cost of GD by approximating the full gradient using a mini-batch of data. This introduces stochasticity, which can help escape shallow local minima:
\begin{equation}
\theta_{t+1} = \theta_t - \eta \frac{1}{|B_t|} \sum_{i \in B_t} \nabla_\theta \mathcal{L}_i(\theta_t),
\label{eq:sgd_update}
\end{equation}
where \(B_t\) is a randomly sampled mini-batch at iteration \(t\). Although SGD’s updates are noisier, they allow for faster iterations and can improve generalization by avoiding overfitting to the training data.

\subsection{Momentum-Based Methods}
Momentum methods improve upon SGD by accumulating a velocity vector \(v_t\) that smooths out the updates and accelerates convergence, especially in regions with oscillatory gradients:
\begin{align}
v_t &= \gamma v_{t-1} + \eta \nabla_\theta \mathcal{L}(\theta_t), \\
\theta_{t+1} &= \theta_t - v_t,
\label{eq:momentum_update}
\end{align}
where \(\gamma \in [0,1)\) is the momentum coefficient. This approach helps in dampening oscillations and can navigate ravines more effectively.

\subsection{Adaptive Learning Rate Methods: RMSProp and ADAM}
While momentum methods focus on accumulating gradient information, adaptive learning rate methods adjust the update step for each parameter individually.

\subsubsection{RMSProp}
RMSProp scales the learning rate by a running average of squared gradients, thereby normalizing the updates:
\begin{align}
s_t &= \beta s_{t-1} + (1-\beta) \left(\nabla_\theta \mathcal{L}(\theta_t)\right)^2, \\
\theta_{t+1} &= \theta_t - \frac{\eta}{\sqrt{s_t} + \epsilon} \nabla_\theta \mathcal{L}(\theta_t),
\label{eq:rmsprop_update}
\end{align}
where \(\beta\) is the decay rate and \(\epsilon\) is a small constant to prevent division by zero. This adaptive scaling is particularly effective when different parameters have gradients of varying magnitudes.

\subsubsection{ADAM Optimizer}
ADAM combines momentum and RMSProp by maintaining both first and second moment estimates of the gradients:
\begin{align}
m_t &= \beta_1 m_{t-1} + (1-\beta_1) \nabla_\theta \mathcal{L}(\theta_t), \\
v_t &= \beta_2 v_{t-1} + (1-\beta_2) \left(\nabla_\theta \mathcal{L}(\theta_t)\right)^2, \\
\hat{m}_t &= \frac{m_t}{1-\beta_1^t}, \quad \hat{v}_t = \frac{v_t}{1-\beta_2^t}, \\
\theta_{t+1} &= \theta_t - \frac{\eta \, \hat{m}_t}{\sqrt{\hat{v}_t} + \epsilon},
\label{eq:adam_update}
\end{align}
where \(\beta_1\) and \(\beta_2\) control the exponential decay rates for the moment estimates. ADAM's parameter-wise adaptation makes it highly effective for optimizing complex, high-dimensional models with sparse or noisy gradients.

\subsection{Parallels to Second-Order Methods}
Second-order methods, such as Newton's method, incorporate curvature information by using the Hessian matrix. Although computing the Hessian is impractical in high dimensions, the intuition behind Newton’s method is valuable:
\begin{equation}
\theta_{t+1} = \theta_t - \bigl(\nabla^2_\theta \mathcal{L}(\theta_t)\bigr)^{-1} \nabla_\theta \mathcal{L}(\theta_t).
\label{eq:newton_update}
\end{equation}
This update rescales the gradient by the curvature of the loss landscape, potentially leading to faster convergence when the curvature varies significantly across dimensions. Adaptive methods like RMSProp and ADAM can be seen as approximating this behavior by normalizing the gradients using running estimates of gradient magnitudes, thus offering some benefits of second-order methods without the computational overhead.

\subsection{Summary}
In summary, the progression from GD to SGD, momentum methods, RMSProp, and ADAM reflects a continuous effort to balance computational efficiency with robust convergence in complex, high-dimensional optimization problems. While second-order methods provide a theoretical ideal by incorporating curvature information, first-order methods enhanced with momentum and adaptive learning rates have become the practical standard in modern machine learning.



\section{Feed-Forward Neural Network Architectures}
\label{sec:ffnn}

Feed-Forward Neural Networks (FFNNs) implement hierarchical feature transformations through layered composition of nonlinear functions. For an $L$-layer network, the $l$-th layer's computation is:

\begin{equation}
h^{(l)} = \sigma\left(W^{(l)} h^{(l-1)} + b^{(l)}\right), \quad l = 1, \dots, L,
\label{eq:ffnn_layer}
\end{equation}

where $h^{(0)} = x$ is the input, $W^{(l)} \in \mathbb{R}^{d_l \times d_{l-1}}$ are weight matrices, $b^{(l)} \in \mathbb{R}^{d_l}$ are bias vectors, and $\sigma$ denotes element-wise activation functions. The final output is:

\[
f_\theta(x) = W^{(L)} h^{(L-1)} + b^{(L)},
\]

with $\theta = \{W^{(l)}, b^{(l)}\}_{l=1}^L$. The choice of activation functions critically determines the network's expressive power.

\subsection{Activation Function Analysis}

In neural networks, the activation function is essential to introduce non-linearity into the model. Without any activation function, each layer would perform a linear transformation, and the composition of multiple linear layers would still be linear. This collapse to a purely linear model severely limits the network’s capacity to capture complex, non-linear relationships in the data.

Moreover, while ReLU (Rectified Linear Unit) is popular in many architectures, it is not suitable for our purposes. In our framework, where we differentiate the network's output, ReLU's derivative is zero for negative inputs, which can lead to vanishing gradients and inactive neurons. Therefore, we use alternative activation functions that maintain non-linearity and allow for effective gradient propagation.

Below, we briefly analyze the activation functions employed in our work:

\textbf{Sigmoid:}
\[
\sigma(x) = \frac{1}{1+e^{-x}}, \quad x \in \mathbb{R}
\]
Maps inputs to \((0,1)\). Although it provides smooth gating, it suffers from vanishing gradients for large \(|x|\).

\textbf{Hyperbolic Tangent (tanh):}
\[
\tanh(x) = \frac{e^x - e^{-x}}{e^x + e^{-x}}, \quad x \in \mathbb{R}
\]
Maps inputs to \((-1,1)\) and is centered at zero, which can improve gradient flow compared to the sigmoid.

\textbf{Swish:}
\[
\text{Swish}(x) = x \cdot \sigma(x) = \frac{x}{1+e^{-x}}
\]
A smooth, non-monotonic activation that often outperforms conventional functions by allowing better gradient propagation and learning dynamics.

\textbf{Mish:}
\[
\text{Mish}(x) = x \cdot \tanh\left(\ln(1+e^x)\right)
\]
A self-regularizing activation function that enhances information flow through non-monotonicity, often resulting in improved performance and robustness in deep networks.

\textbf{Gaussian Error Linear Unit (GELU):}
\[
\text{GELU}(x) \approx x \cdot \frac{1}{2}\left[1 + \tanh\left(\sqrt{\frac{2}{\pi}}(x + 0.044715x^3)\right)\right]
\]
A smoother alternative that incorporates a probabilistic view of neuron activation. GELU can yield benefits in deeper models where smooth gradient behavior is critical.

These activation functions ensure that our network maintains non-linear transformations, enabling it to model complex phenomena while also supporting effective differentiation and gradient-based optimization.


\section{Restricted Boltzmann Machines}

Restricted Boltzmann Machines (RBMs) are generative stochastic neural networks that offer a powerful framework for modeling complex probability distributions. An RBM is composed of two layers: a visible layer $\mathbf{v} \in \mathbb{R}^{N}$, which represents the observed data or degrees of freedom, and a hidden layer $\mathbf{h} \in \{0,1\}^{M}$, which captures higher-order correlations among the visible units. The term ``restricted'' indicates that there are no intra-layer connections within the visible or hidden units, a feature that simplifies both inference and training.

\subsection{RBM Energy Function and Probability Distribution}

The joint configuration of the visible and hidden units is characterized by an energy function:
\begin{equation}
    E(\mathbf{v}, \mathbf{h}) = -\mathbf{a}^T \mathbf{v} - \mathbf{b}^T \mathbf{h} - \mathbf{v}^T W \mathbf{h},
\end{equation}
where:
\begin{itemize}
    \item $\mathbf{a} \in \mathbb{R}^{N}$ and $\mathbf{b} \in \mathbb{R}^{M}$ are bias vectors for the visible and hidden layers, respectively,
    \item $W \in \mathbb{R}^{N \times M}$ is the weight matrix that connects visible and hidden units.
\end{itemize}

The probability of a given configuration $(\mathbf{v}, \mathbf{h})$ is defined by the Boltzmann distribution:
\begin{equation}
    p(\mathbf{v}, \mathbf{h}) = \frac{1}{Z} \exp\bigl(-E(\mathbf{v}, \mathbf{h})\bigr),
\end{equation}
with the partition function
\begin{equation}
    Z = \sum_{\mathbf{v}, \mathbf{h}} \exp\bigl(-E(\mathbf{v}, \mathbf{h})\bigr)
\end{equation}
ensuring normalization. Marginalizing over the hidden variables, the distribution over the visible units becomes:
\begin{equation}
    p(\mathbf{v}) = \frac{1}{Z} \sum_{\mathbf{h}} \exp\bigl(-E(\mathbf{v}, \mathbf{h})\bigr).
\end{equation}

A key property of RBMs is the conditional independence of the hidden units given the visible layer, which leads to:
\begin{equation}
    p(h_j = 1 | \mathbf{v}) = \sigma\Bigl(b_j + \sum_{i} W_{ij} v_i\Bigr),
\end{equation}
where $\sigma(x) = 1/(1+\exp(-x))$ is the logistic sigmoid function.

\subsection{RBMs as Variational Wavefunction Ans\"atze}

In the context of quantum many-body systems, RBMs can be utilized as variational ans\"atze for wavefunctions. Here, the wavefunction is represented as:
\begin{equation}
    \Psi(\mathbf{x}) = \exp\Biggl( \mathbf{a}\cdot\mathbf{x} + \sum_{j=1}^{M} \log\Bigl[1+\exp\Bigl(b_j + \sum_{i=1}^{N} W_{ij} x_i\Bigr)\Bigr] \Biggr),
\end{equation}
where $\mathbf{x}$ represents the configuration (e.g., positions of particles) and $\{ \mathbf{a}, \mathbf{b}, W \}$ are variational parameters. This ansatz is particularly attractive because the hidden layer introduces non-linear correlations that can capture complex entanglement and interactions in the system.

Training the RBM wavefunction involves variational Monte Carlo (VMC) methods, where the variational parameters are optimized to minimize the expectation value of the Hamiltonian:
\begin{equation}
    E = \frac{\langle \Psi | \hat{H} | \Psi \rangle}{\langle \Psi | \Psi \rangle}.
\end{equation}
During the VMC procedure, configurations are sampled according to $|\Psi(\mathbf{x})|^2$, and the parameters are updated via gradient-based optimization methods.

\subsection{Advantages and Challenges}

The RBM ansatz is highly flexible and can approximate a wide range of quantum states. However, a standard RBM is symmetric under particle exchange and does not inherently incorporate fermionic antisymmetry, which is essential for systems of electrons. 




\section{Normalizing Flow}
Sampling from complex probability distributions is a core challenge in fields such as statistics, machine learning, and computational sciences. Typically, computers generate pseudo-random numbers uniformly over the interval \([0,1]\) using deterministic algorithms. These uniform samples are then transformed into samples from well-known distributions—like the standard normal distribution—using methods such as the Box-Muller transform or the Ziggurat algorithm. The standard normal distribution, with its mathematical convenience, serves as a natural \emph{base distribution}.

Normalizing flows build upon this idea by defining a sequence of invertible, differentiable transformations that map samples from a simple base distribution (e.g., \(\mathcal{N}(0,I)\)) to those from a complex target distribution. Essentially, the procedure is analogous to the transformation used to generate normal variates, but it is extended to tackle far more intricate distributions that may exhibit multimodality and high-dimensional correlations.

\subsection{Theoretical Foundations of Normalizing Flows}
At the heart of normalizing flows is an invertible mapping \(T\) that transforms a random variable \(x\) from a base distribution \(p_X(x)\) into a new variable \(y = T(x)\) that follows a target distribution \(p_Y(y)\). This relationship is formalized by the change-of-variables formula:
\[
p_Y(y) = p_X(x) \left| \det \left( \frac{\partial T^{-1}(y)}{\partial y} \right) \right|.
\]
Instead of modeling \(p_Y(y)\) directly, the flow model parameterizes \(T\) as a composition of simpler invertible functions, ensuring that both \(T\) and its Jacobian determinant are tractable.

In our framework, we employ a variant known as \emph{conditional flow matching}. Here, the transformation from the base to the target distribution is conceptualized via a displacement field. Given a base sample \(x\) and a corresponding target sample \(y\), the ideal displacement is defined as:
\[
v_{\text{true}} = y - x.
\]
We consider a continuous interpolation between \(x\) and \(y\) using an artificial time parameter \(t \in [0,1]\):
\[
\psi_t = (1-t)x + t\, y.
\]
The objective is to learn a function \(v(\psi_t, t)\) that approximates this displacement field. Integrating the displacement field over \(t\) effectively reconstructs the overall transformation \(T\) that maps the simple base distribution to the complex target distribution.

This theoretical perspective encapsulates the philosophy of normalizing flows: leveraging simple base distributions and learnable, invertible mappings to model complex, high-dimensional densities. By focusing on the displacement field through conditional flow matching, we provide a clear and generalizable framework that extends the basic idea of transforming uniform or normal samples to a wide array of challenging distributions.

\section{Conclusion}
Normalizing flows offer a powerful and general approach to modeling complex probability distributions by transforming samples from simple base distributions through invertible mappings. The conditional flow matching method highlights this concept by framing the transformation as a continuous displacement field, thereby generalizing the fundamental process of random number generation and transformation. This theoretical foundation is crucial for addressing high-dimensional and intricate sampling problems across many scientific and engineering domains.
\section{Physics Informed Neural Network}

\subsection{Overview}

In our approach, the variational wavefunction is written in a form inspired by the Slater–Jastrow ansatz:
\begin{equation}
  \Psi(\mathbf{x}) = \mathrm{SD}(\mathbf{x}) \, \exp\Big[ f(\mathbf{x}) \Big],
\end{equation}
where \(\mathrm{SD}(\mathbf{x})\) is the Slater determinant ensuring the required antisymmetry, and the correlation factor
\[
  \exp\Big[ f(\mathbf{x}) \Big]
\]
is built from a neural network architecture designed to capture both one-body and two-body correlation effects.

\subsection{Architecture Details}

Our architecture decomposes the function \(f(\mathbf{x})\) into two main contributions:
\begin{equation}
  f(\mathbf{x}) = \sum_{i=1}^{N} \phi(x_i) + \sum_{i<j} \psi(x_i, x_j),
\end{equation}
where \(x_i\) denotes the coordinates of electron \(i\) (with \(x_i \in \mathbb{R}^d\) and typically \(d=2\) in our examples).

\paragraph{One-Body Network \(\phi\):}  
The function \(\phi(x_i)\) is evaluated independently for each electron. This network is designed to capture single-particle effects (such as orbital relaxation) and scales linearly with the number of electrons \(O(N)\). Thanks to the universal approximation property of neural networks, \(\phi\) can, in principle, approximate any continuous function of a single electron coordinate.

\paragraph{Two-Body Network \(\psi\):}  
The function \(\psi(x_i, x_j)\) handles the explicit electron-electron interactions. It is designed to take as input either the pairwise coordinates (or a derived measure such as the relative distance) and outputs a scalar that reflects the interaction correction. Because there are \(O(N^2)\) unique pairs, this two-body network is crucial to capture correlations beyond the mean-field approximation. Importantly, \(\psi\) is constructed to be \emph{symmetric}, i.e., 
\[
\psi(x_i, x_j) = \psi(x_j, x_i),
\]
which ensures that the overall Jastrow factor, \(\exp\left[\sum_{i} \phi(x_i) + \sum_{i<j} \psi(x_i,x_j)\right]\), remains symmetric as required. (The overall antisymmetry of \(\Psi(\mathbf{x})\) is provided solely by the Slater determinant.)

\paragraph{Exponential Form:}  
We incorporate the outputs of the one-body and two-body networks inside an exponential. This multiplication rule is common in variational Monte Carlo (VMC) approaches because:
\begin{itemize}
  \item It guarantees positivity for the correlation factor.
  \item Small additive corrections in \(f(\mathbf{x})\) translate to multiplicative modifications of the wavefunction amplitude, which is key to adjusting the nodal surface in a controlled manner.
\end{itemize}
In fact, the variational principle ensures that if the ansatz has sufficient expressive power, the optimized wavefunction will yield an energy that is an upper bound to the exact ground state energy. The combination of \(\phi\) and \(\psi\) networks in the exponent, by virtue of the universal approximation theorem, can in principle approximate any symmetric correlation function and therefore capture the full configuration interaction (FCI) limit.

\subsection{Advantages Over a Brute-Force \(N\)-Particle Input}

A direct, brute-force neural network that takes the full \(N\)-particle coordinate \(\mathbf{x} \in \mathbb{R}^{Nd}\) as input and outputs a scalar correction factor suffers from several drawbacks:
\begin{enumerate}
  \item \textbf{Scalability}: The combinatorial complexity of the full \(N\)-particle space makes it extremely challenging to model correlations when \(N\) increases. In contrast, our architecture scales as \(O(N) + O(N^2)\), a substantial improvement over factorial or exponential scaling.
  \item \textbf{Physical Interpretability}: By explicitly separating one-body and two-body contributions, the model more naturally mirrors the underlying physical processes (mean-field behavior plus electron-electron interactions), thus offering better insight and more robust training.
  \item \textbf{Symmetry Enforcement}: In a full \(N\)-particle network, one must explicitly enforce permutation invariance among electrons. In our architecture, this invariance is naturally built into the summation over individual and pairwise contributions.
\end{enumerate}

\subsection{In Principle Achievement of FCI Quality}

\paragraph{Expressivity:}  
Both the one-body network \(\phi\) and the two-body network \(\psi\) are universal approximators. Provided they have sufficient capacity (depth and width), they can approximate the optimal correlation functions that would reproduce the FCI wavefunction when combined with the Slater determinant. Therefore, even though we only include one-body and two-body terms, this ansatz is capable of capturing the majority of electron correlations in many systems.

\paragraph{Combined Network:}  
In our implementation, the final output \(f(\mathbf{x})\) is obtained by an additive combination of the \(\phi\) and \(\psi\) outputs, which is then exponentiated. This unification is critical:
\begin{itemize}
  \item It allows the network to use the multiplicative nature of the exponential to fine-tune the weight of correlations.
  \item It ensures that contributions from both one-body and two-body networks are coherently combined.
  \item Being in the exponent, small changes in \(f(\mathbf{x})\) can lead to exponential corrections in the wavefunction, thus providing a sensitive control mechanism to lower the variational energy.
\end{itemize}

\subsection{Summary}

In summary, our architecture adopts a Slater–Jastrow ansatz with two separate neural networks, one for one-body (individual electron) contributions and one for two-body (electron-electron) interactions. These contributions are summed and placed in an exponential to form the correlation factor multiplying a Slater determinant. This design offers:

\begin{itemize}
  \item Efficient scaling compared to brute-force \(N\)-particle input by focusing on one-body and two-body components.
  \item The potential to achieve FCI-quality results by virtue of the universal approximation capabilities of neural networks.
  \item Inherent symmetry due to the explicit symmetrization of the two-body term and the additive combination, ensuring that the total wavefunction respects particle indistinguishability.
  \item Physical interpretability by mirroring the separation between mean-field behavior and electron-electron correlation.
\end{itemize}

This modular and physically motivated architecture not only enhances computational efficiency but also provides a systematic pathway to improve accuracy by gradually increasing the network capacity or incorporating additional many-body contributions if necessary.

\part{Methods}
\chapter{Trial Wavefunction}
\label{ch:trial_wavefunction}

In this chapter we specify the concrete modeling and algorithmic choices used in our computations.
We proceed from the physical setup and trial states to objectives, with an emphasis on stability,
symmetry, and differentiability---all of which are critical for Laplacian-based local-energy
evaluations and stochastic reconfiguration (SR). Unless otherwise stated, we work in atomic units.

\section{Preliminaries}
\label{sec:setup}

\paragraph{Hamiltonian and units.}
We study $N$ spin-$\tfrac12$ fermions in two spatial dimensions confined by a harmonic trap
and interacting via Coulomb repulsion,
\begin{equation}
\label{eq:dot-h}
H \;=\; \sum_{i=1}^N \Big( -\tfrac12 \nabla_{\mathbf r_i}^2 + \tfrac12 \,\omega^2 r_i^2 \Big)
\;+\; \sum_{1\le i<j\le N} \frac{1}{|\mathbf r_i-\mathbf r_j|},
\qquad (\hbar=m_e=e=1).
\end{equation}
Spin does not appear in $H$; eigenstates factorize into a spatial function with good $L_z$ and a
spin function with good $(S,S_z)$.

\paragraph{Coordinates and scaling.}
We collect positions as $R=(\mathbf r_1,\ldots,\mathbf r_N)\in\mathbb R^{2N}$ and use trap scaling
$\tilde{\mathbf r}_i=\sqrt{\omega}\,\mathbf r_i$, $\tilde{R}=\sqrt{\omega}\,R$. This aligns typical
length scales to $\mathcal O(1)$, improves conditioning of second derivatives, and makes the
noninteracting ground state $\omega$-independent. When useful, we also use pair distances
$r_{ij}=|\mathbf r_i-\mathbf r_j|$ and their scaled versions $\tilde r_{ij}=|\tilde{\mathbf r}_i-\tilde{\mathbf r}_j|$.

\paragraph{Sectors and focus.}
We primarily report \emph{closed-shell} cases (unique spin singlet at the noninteracting level).
Open shells introduce a degenerate manifold and additional choices for the reference; we keep this
to brief remarks only where needed.

\paragraph{Spin–orbitals and basis.}
One-particle spatial orbitals are Fock–Darwin functions $\{\phi_{nm}(\mathbf r)\}$ with
$E_{nm}=\omega(2n+|m|+1)$, $n\in\mathbb N_0$, $m\in\mathbb Z$. Spin–orbitals are
$\varepsilon_\alpha(x)=\phi_{nm}(\mathbf r)\chi_\sigma(s)$, $\sigma\in\{\uparrow,\downarrow\}$.

% ==========================================================

\section{Overview of the ansatz}
\label{sec:ansatz}
Our trial state combines exact antisymmetry, an analytic short-range cusp, and a compact
neural residual, while optionally evaluating the Slater determinant on backflowed coordinates:
\begin{align}
  \Psi_{\theta,\beta}(R)
  &= \mathrm{SD}\big(\tilde{R}'\big)\;
     \exp\!\Big(
       \sum_{i<j} u_{\sigma_i \sigma_j}(\tilde r_{ij})
       + W_\theta(\tilde{R})
     \Big),\label{eq:ansatz-overall} \\
  \tilde{R}' &= \tilde{R} + \Delta_\beta(\tilde{R}).\label{eq:bf-coords}
\end{align}
Here $\mathrm{SD}$ enforces antisymmetry; $u$ hard-wires the Kato cusp; $W_\theta$ (the NQS correlator) learns smooth
medium/long-range structure; and $\Delta_\beta$ (backflow) adjusts the nodes by smoothly warping the coordinates used
inside the determinants. We deliberately feed the NQS with $\tilde{R}$ (not $\tilde{R}'$) to avoid self-reinforcing feedback
and keep local-energy variance low.

\paragraph{Design principles.}
(i) \emph{Separate guaranteed physics from learned details:} antisymmetry and the cusp are hard-wired; the network learns the rest.
(ii) \emph{Condition the geometry:} work in trap units and use soft-core pair features so that first/second derivatives are bounded.
(iii) \emph{Keep the learned object small and smooth:} once the cusp is explicit, a compact NQS suffices across $(N,\omega)$.

\section{Slater reference}
\label{subsec:slater}
For closed shells we use a restricted product of spin-block determinants of the lowest $N/2$ harmonic oscillator orbitals (trap units):
\begin{equation}
  \mathrm{SD}(\tilde{R})
  =
  \det\!\big[\phi_a(\tilde{\mathbf r}_i)\big]_{i\in\uparrow,\;a\le N/2}\;
  \det\!\big[\phi_a(\tilde{\mathbf r}_j)\big]_{j\in\downarrow,\;a\le N/2}.
  \label{eq:slater}
\end{equation}
The correlator reweights amplitudes \emph{within} nodal pockets; without backflow, the nodes are those of~\eqref{eq:slater}.

\section{Neural correlator $W_\theta$}
\label{subsec:nqs}
The NQS aggregates \emph{particlewise} and \emph{pairwise} embeddings with permutation-invariant pooling and appends two global
statistics, all in $\tilde{R}$. A core safety requirement is that pair channels satisfy $d s_k/d\tilde r \to 0$ as $\tilde r\to 0$
so that $W_\theta$ never injects spurious $1/\tilde r$ terms into $\nabla\log\Psi$ or $\Delta\log\Psi$.

\subsection{Particle branch (DeepSets).}
This part of the network is typically referred to as Deep Sets [kilde]. The permutation invariance of quantum many-body mechanics is naturally captured by this architecture. 
A neural network $\phi$ creates a latent embedding for each particle individually. All embeddings are pooled together to form a global representation, which is used later in the readout. Formally, we have
\[
  \mathbf h_i^\phi=\phi(\tilde{\mathbf x}_i)\in\mathbb R^{d_L},\qquad
  \overline{\mathbf h}^{\phi}=N^{-1}\sum_i \mathbf h_i^\phi\in\mathbb R^{d_L}.
\]

\subsection{Pair branch (soft-core, even channels).}
Although the particle branch could theoretically produce pair correlations, in practice it struggles to represent sharp near-field structure. We therefore add a dedicated pair branch that builds features from interparticle distances.
This network processes all pairs $(i,j), i\ne j$ individually and pools the resulting latent embeddings. 
The problem is that raw distances $\tilde r_{ij}$ have unbounded derivatives at coalescence, which destabilizes Laplacian evaluations. We therefore build \emph{safe} pair features with bounded first and second derivatives as $\tilde r\to 0$.
With $\tilde r^{\rm soft}=\sqrt{\tilde r^2+\varepsilon^2}$ ($\varepsilon\!\in[0.15,0.30]$ in trap units), we build six “safe” features
\begin{align}
  s_1 &= \log\!\big(1+(\tilde r^{\rm soft}/\varepsilon)^2\big),\quad
  s_2 = \tfrac{\tilde r^{2}}{\tilde r^{2}+\varepsilon^2},\quad
  s_3 = (\tilde r^{\rm soft}/\varepsilon)^{2}\exp[-(\tilde r^{\rm soft}/\varepsilon)^{2}], \notag\\
  \mathrm{rbf}_g &= \exp(-g\,s_1),\quad g\in\{0.25,1,4\},\qquad
  \mathbf u_{ij}=[s_1,s_2,s_3,\mathrm{rbf}_{0.25},\mathrm{rbf}_{1},\mathrm{rbf}_4]\in\mathbb R^{6}.
  \label{eq:safe-feats}
\end{align}
An MLP $\psi$ maps $\mathbf u_{ij}\mapsto\mathbf h^{\psi}_{ij}\in\mathbb R^{d_L}$. A short-range gate
\[
  \chi(\tilde r)=\frac{\tilde r^{2}}{\tilde r^{2}+r_g^{2}},\quad r_g\approx 0.3,\quad \chi(0)=\chi'(0)=0,
\]
suppresses learned responses \emph{at} coalescence, we use $\chi\,\mathbf h^\psi_{ij}$ downstream.

\subsection{Analytic short-range cusp}
\label{subsec:cusp}
Due to the nesseccity of the safe features in the pair branch, the NQS cannot represent the Kato cusp condition exactly. 
We therefore add an analytic cusp term to the exponent in~\eqref{eq:ansatz-overall}. 
We enforce the electron–electron cusp in trap units via
\begin{equation}
  u_{\sigma\sigma'}(\tilde r)
  =
  \gamma_{\sigma\sigma'}\,\tilde r\,e^{-\tilde r},
  \qquad
  \gamma_{\uparrow\downarrow}=\tfrac{1}{d-1},\quad
  \gamma_{\uparrow\uparrow}=\tfrac{1}{d+1}.
  \label{eq:cusp}
\end{equation}
Equivalently, in physical units $r$, $u(r)=\gamma_{\sigma\sigma'}\,(r/a_{\rm ho})\,\exp[-(r/a_{\rm ho})]$.
The linear coefficient yields the Kato slope at contact in general $d$; the exponential taper suppresses long-range interference, which means the NQS need not unlearn cusp structure at medium/long range.


\subsection{Global readout}
With contributions from the particle and pair branches, we form a global representation for the final readout. In addition to the pooled embeddings $\overline{\mathbf h}^\phi$ and $\overline{\mathbf h}^\psi$, we also include two global statistics that help the network gauge overall system size and density.
We average pairs by default or apply degree-normalized radial attention
\[
  w_{ij}=\exp[-(\tilde r_{ij}/r_c)^p],\quad
  \hat w_{ij}=w_{ij}\Big/\sum_{k\ne i}w_{ik},\quad
  (r_c,p)=(0.4,6),
\]
then form $g_i=\sum_{j\ne i}\hat w_{ij}\,\mathbf h^\psi_{ij}$ and $\overline{\mathbf h}^\psi=N^{-1}\sum_i g_i$.
Two system-level scalars in trap units,
\[
  \mathbf g(\tilde{R})=
  \Big[\tfrac{1}{Nd}\sum_i\|\tilde{\mathbf x}_i\|^2,\;
       \tfrac{2}{N(N-1)}\sum_{i<j}s_1(\tilde r_{ij})\Big]\in\mathbb R^2,
\]
are concatenated with $\overline{\mathbf h}^{\phi}$ and $\overline{\mathbf h}^{\psi}$ and mapped to a scalar by a tiny MLP $\rho$:
\begin{equation}
  W_\theta(\tilde{R})
  =
  \rho\!\Big(\overline{\mathbf h}^{\phi}\ \|\ \overline{\mathbf h}^{\psi}\ \|\ \mathbf g(\tilde{R})\Big).
  \label{eq:nqs-readout}
\end{equation}

\section{Backflow}
\label{subsec:backflow-method}

Backflow deforms the coordinates \emph{used only inside the Slater reference} so that the nodal surface can adapt to interparticle correlations:
\begin{equation}
  \tilde{R}' \;=\; \tilde{R} + \Delta_\beta(\tilde{R}) ,
  \qquad
  \Psi_{\theta,\beta}(R) \;=\; \mathrm{SD}(\tilde R')\;
  \exp\!\Big(\sum_{i<j} u_{\sigma_i\sigma_j}(\tilde r_{ij}) + W_\theta(\tilde R)\Big).
  \label{eq:bf-def}
\end{equation}
The correlator $W_\theta$ and cusp $u$ are evaluated on $\tilde R$ to avoid feedback loops in the local energy; only the determinant ``sees'' $\tilde R'$.
This choice cleanly separates (i) amplitude reweighting within nodal pockets (handled by $u+W_\theta$) from (ii) nodal \emph{geometry} (handled by backflow).

\paragraph{Design goals.}
(i) \emph{Stability near coalescence:} messages use mollified radii and short-range gates so that $\partial \Delta_\beta/\partial \tilde r$ remains bounded; \\
(ii) \emph{Symmetries:} permutation invariance by construction; near translation invariance by centering/projection; rotation equivariance via vector messages; \\
(iii) \emph{Close to identity at initialization:} zero-initialized last layer and a positive, learnable scale keep $\Delta_\beta\!\approx\!0$ at start.

\subsection{Message–passing construction}
We use a minimal two-stage message–passing update. For each ordered pair $(i,j)$ with $i\neq j$, form a message
\begin{align}
  \mathbf m_{ij}
  &= \phi_\beta\!\Big(
      \underbrace{\tilde{\mathbf x}_i}_{\text{node $i$}},\;
      \underbrace{\tilde{\mathbf x}_j}_{\text{neighbor $j$}},\;
      \underbrace{\mathbf r_{ij}}_{\tilde{\mathbf x}_i-\tilde{\mathbf x}_j},\;
      \underbrace{\tilde r_{ij}^{\rm soft}}_{\sqrt{\tilde r_{ij}^2+\varepsilon^2}},\;
      \underbrace{(\tilde r_{ij}^{\rm soft})^2}_{\text{scale}}
    \Big)\cdot w_{ij}^{\rm spin},
  \label{eq:bf-message}
\end{align}
where $\phi_\beta$ is a tiny MLP and $\varepsilon\!\ll\!1$ (trap units) mollifies the norm.\footnote{In code: \texttt{BackflowNet.phi} takes $(3d+2)$ inputs: $(\tilde{\mathbf x}_i,\tilde{\mathbf x}_j,\mathbf r_{ij},\tilde r_{ij}^{\rm soft},(\tilde r_{ij}^{\rm soft})^2)$.}
Spin enters through a light‐weight mask $w_{ij}^{\rm spin}\in\{0,1\}$ (same‐spin only or all pairs).

Messages are aggregated into a node state
\begin{equation}
  \mathbf m_i \;=\; \mathrm{Agg}_{j\neq i}\,\mathbf m_{ij},\quad
  \mathrm{Agg}\in\{\mathrm{sum},\,\mathrm{mean},\,\mathrm{max}\},
\end{equation}
and a node/update network $\psi_\beta$ returns a displacement
\begin{equation}
  \delta\tilde{\mathbf x}_i \;=\; \psi_\beta\!\big(\tilde{\mathbf x}_i \ \| \ \mathbf m_i\big)
  \;\in\;\mathbb{R}^d.
\end{equation}
To keep displacements bounded and start near identity we apply
\begin{equation}
  \Delta_\beta(\tilde R)\;=\;\tanh\!\big(\delta\tilde X\big)\cdot s_\beta,
  \qquad s_\beta=\mathrm{softplus}(s_\beta^{\rm raw})>0,
  \label{eq:bf-bound-scale}
\end{equation}
with the last linear layer of $\psi_\beta$ zero‐initialized.
Empirically, this parameterization stabilizes SR/energy tails and protects Laplacians.

\subsection{Short-range behavior and gates}
As $\tilde r_{ij}\!\to\!0$, raw $1/\tilde r$ factors can make $\nabla\!\cdot\!\Delta_\beta$ spiky.
We therefore (i) mollify $\tilde r_{ij}$ as in~\eqref{eq:bf-message}, and optionally (ii) gate the message with
\begin{equation}
  \chi_{\rm bf}(\tilde r_{ij})=\frac{\tilde r_{ij}^2}{\tilde r_{ij}^2+r_g^2},
  \qquad \chi_{\rm bf}(0)=\chi'_{\rm bf}(0)=0,\quad r_g\approx 0.3,
  \label{eq:bf-gate}
\end{equation}
or with degree-normalized radial attention
\begin{equation}
  w_{ij}=\exp\!\big[-(\tilde r_{ij}/r_c)^p\big],\qquad
  \hat w_{ij}=\frac{w_{ij}}{\sum_{k\neq i} w_{ik}}.
\end{equation}
Either choice keeps the learned response smooth at coalescence and limits the BF contribution to the local energy variance.

\subsection{Symmetries and COM neutrality}
Permutation symmetry is automatic (pairwise messages and symmetric aggregation).
To avoid a spurious center-of-mass (COM) drift, we project the flow to zero mean:
\begin{equation}
  \Delta_\beta \;\leftarrow\; \Delta_\beta \;-\; \frac{1}{N}\sum_{i=1}^N \Delta_{\beta,i}.
  \label{eq:bf-com}
\end{equation}
With inputs expressed in trap units $\tilde R$ (already centered in practice) and with messages built from $\mathbf r_{ij}$, the update is effectively translation-invariant and rotation-equivariant.%
\footnote{Implementation note: subtracting the batchwise mean of $\Delta_\beta$ is a one-liner and removes the small residual COM drift seen in ablations.}

\subsection{Effect on the nodes and variational safety}
Because only $\mathrm{SD}$ is evaluated at $\tilde R'$, backflow changes the \emph{nodes} but leaves the correlator exponent on $\tilde R$. No Jacobian term is introduced (we are not performing a change of variables of the full wavefunction).
This ``Feynman--Cohen style'' design targets fixed-node error directly while keeping the amplitude model simple and smooth.

\subsection{Regularization, initialization, and cost}
We found the following defaults robust across $(N,\omega)$:
(i) $\varepsilon\in[10^{-6},10^{-4}]$ (trap units) for the soft norm;
(ii) bounded output $\tanh$ with a learnable scale $s_\beta$ initialized in $[0.02,0.10]$;
(iii) optional gate $\chi_{\rm bf}$ with $r_g\approx 0.3$ (trap units).
Zero-initializing the last layer of $\psi_\beta$ makes the ansatz start exactly at the Slater nodes and improves optimizer stability.
The compute/memory cost scales as $O\!\big(BN^2(d+H)\big)$ for batch size $B$ and hidden width $H$.

\subsection{Mapping to implementation}
The networks $\phi_\beta$ and $\psi_\beta$ correspond to \texttt{BackflowNet.phi} (message MLP) and \texttt{BackflowNet.psi} (node/update MLP).
Aggregation is selectable: \texttt{sum|mean|max}. Spin handling (\texttt{use\_spin}, \texttt{same\_spin\_only}) produces the mask $w_{ij}^{\rm spin}$.
Output control matches~\eqref{eq:bf-bound-scale} via \texttt{out\_bound="tanh"} and $s_\beta=\mathrm{softplus}(\texttt{bf\_scale\_raw})$.
To enforce~\eqref{eq:bf-com} in practice, subtract the batchwise mean of the predicted $\Delta_\beta$ before forming $\tilde R'$.

\paragraph{Summary.}
Backflow provides a compact, symmetry-respecting way to move the Slater nodes toward the true correlated nodes. With mollified distances, optional short-range gates/attention, zero-mean projection, and bounded outputs, it integrates cleanly with the analytic cusp and smooth NQS correlator, reducing fixed-node error without inflating local-energy variance.


\chapter{Optimization}
\label{ch:optimization}

\section{Primary objective and two–stage training}
\label{sec:opt-overview}

Given a parameterized wavefunction $\Psi_\theta$, the variational ground–state energy is
\begin{equation}
E(\theta)
=\frac{\langle\Psi_\theta|H|\Psi_\theta\rangle}{\langle\Psi_\theta|\Psi_\theta\rangle}
=\mathbb{E}_{\tilde{R}\sim \pi_\theta}\!\big[E_L(\tilde{R})\big],
\qquad
\pi_\theta(\tilde{R})=\frac{|\Psi_\theta(\tilde{R})|^2}{\int |\Psi_\theta|^2}.
\end{equation}
With scaled coordinates $\tilde{\mathbf r}=\sqrt{\omega}\,\mathbf r$ the local energy reads
\begin{equation}
\label{eq:local-energy}
E_L(\tilde{R})
= -\tfrac12\sum_{i=1}^N\!\Big[\Delta_{\tilde{\mathbf r}_i}\ln\Psi_\theta + \|\nabla_{\tilde{\mathbf r}_i}\ln\Psi_\theta\|^2\Big]
+\tfrac{1}{2}\sum_{i=1}^N \tilde r_i^2
+\sum_{i<j}\frac{\sqrt{\omega}}{\tilde r_{ij}},
\end{equation}
and the zero–variance principle holds: $\mathrm{Var}_{\pi_\theta}[E_L]=0$ iff $\Psi_\theta$ is an eigenstate.

Our optimization follows two stages that mirror the implementation:
\begin{enumerate}
  \item \textbf{Residual–based pretraining (stage I).} We fit the residual of $E_L$ to a moving scalar target $E_{\rm eff}$ using the loss
  \(
    \mathcal{L}_{\rm res}(\theta)=\mathbb{E}\!\big[(E_L(\tilde R)-E_{\rm eff})^2\big].
  \)
  We begin with a \emph{self–target} (stabilizer) $E_{\rm eff}=\mu$ where $\mu=\mathbb{E}[E_L]$ over the current batch. This choice is equivalent to variance minimization and prevents the network from chasing a potentially inaccurate external target before it can represent low–variance pockets. We then anneal toward a trusted reference $E_{\rm DMC}$ via
  \begin{equation}
    E_{\rm eff}(\alpha) \;=\; \alpha\,E_{\rm DMC} + (1-\alpha)\,\mu,
    \qquad \alpha \in [0,1],
  \end{equation}
  with a smooth cosine schedule $\alpha=\alpha(t)$ that ramps from $\alpha_{\rm start}$ to $\alpha_{\rm end}$ over a prescribed fraction of epochs (code: \texttt{alpha\_start}, \texttt{alpha\_end}, \texttt{alpha\_decay\_frac}). Setting the \texttt{objective} to \texttt{"residual"} uses $E_{\rm eff}=\mu$; \texttt{"energy"} uses $E_{\rm eff}=E_{\rm DMC}$; and \texttt{"energy\_var"} uses the annealed blend above.
  \item \textbf{Stochastic Reconfiguration (stage II).} After residual pretraining, we refine with SR (natural–gradient VMC). Writing $O(\tilde R)=\partial_\theta \log\Psi_\theta(\tilde R)$,
  \begin{equation}
    S \;=\; \mathbb{E}_\pi\!\big[O^\top O\big],\qquad
    g \;=\; 2\,\mathbb{E}_\pi\!\big[(E_L-\mathbb{E}_\pi[E_L])\,O\big],
  \end{equation}
  we solve the damped normal equations
  \begin{equation}
    (S+\lambda I)\,\Delta\theta \;=\; -\,g,
  \end{equation}
  using (preconditioned) conjugate gradients with restarts, a trust–region scale on the final step, and batchwise centering/whitening of $O$ (code: \texttt{damping}, \texttt{max\_param\_step}, CG options).
\end{enumerate}
This pipeline exploits the zero–variance principle for stable shaping of $W_\theta$ (and backflow) before applying the geometry–aware SR update.

\section{Configuration–space sampling}
\label{sec:sampling}

\paragraph{Guiding principle (permutation invariance).}
In a many–body fermionic system, \emph{which} particle indices are close is irrelevant; what matters is \emph{how many} close pairs and at which length scales. Consequently, we can design samplers that (i) emphasize \emph{configurational motifs} (clusters, shells, dimers, long tails) rather than specific labels, and (ii) deliberately permute particles and randomly rotate configurations to explore the space efficiently without biasing towards any ordering.

\subsection{Stratified capped–simplex mixture}
We draw collocation batches $X\in\mathbb{R}^{B\times N\times d}$ from a five–component mixture,
\[
\text{\small(center, tails, mixed, shells, dimers)},
\]
with nonnegative weights $w\in\Delta^4$ projected to a \emph{capped simplex} $\{\,w:\, \sum_k w_k=1,\ 0\le w_k\le 0.3\,\}$ to enforce coverage (code: \texttt{\_project\_simplex\_with\_caps}).
Each component proposes $x$ as follows (all in physical units; the code internally scales by $a_{\rm ho}=1/\sqrt{\omega}$):
\begin{itemize}
  \item \textbf{Center:} narrow isotropic Gaussian around the trap center (samples near the density core).
  \item \textbf{Tails:} wide Gaussian (probes low–density outskirts and large–$r$ behavior).
  \item \textbf{Mixed:} hybrid batches mixing narrow and moderate spreads per configuration (tests robustness to inhomogeneous spreads).
  \item \textbf{Shells:} points placed on $K$ hyperspherical shells with trap–scaled radii $\{r_k\}$ and occupancy $\{q_k\}$, with small radial/tangential jitter (captures ring/shell order at weak traps).
  \item \textbf{Dimers:} we induce a few close pairs per configuration by sampling short interparticle offsets with random directions and log–uniform distances (enforces near–coalescence coverage without singularities).
\end{itemize}
We then apply a random in–plane rotation (for $d=2$), and a random permutation of particle indices; both preserve the target expectations by symmetry.

\paragraph{Adaptive mixture and shell occupancy.}
Per epoch we compute simple difficulty scores and update $w$ and $q$ by exponentiated–gradient (EG) steps with momentum, temperature, and an exploration term; $w$ is then projected back to the capped simplex (code: \texttt{eg\_eta}, \texttt{eg\_temp}, \texttt{eg\_momentum}, \texttt{explore\_gamma}, \texttt{prob\_floor}). Mixture difficulty uses the residual $(E_L-E_{\rm eff})^2$ per component; shell difficulty uses a proxy
\[
\underbrace{V_i}_{\text{harm.+Coulomb per particle}}
\;+\; \underbrace{\gamma_{g^2}\,\|\nabla_{\mathbf x_i}\log\Psi\|^2}_{\text{stiff gradients}}
\;+\; \underbrace{\gamma_{\rm cusp}\,r_{\min}^{-1}}_{\text{coalescence pressure}},
\]
aggregated by nearest shell index (code: \texttt{g2\_weight}, \texttt{cusp\_gamma}, safe clip \texttt{rmin\_clip}). This shifts probability mass toward underfit regions \emph{without} collapsing coverage thanks to the caps.

\paragraph{Hard–example injection.}
We maintain a small buffer of previously challenging configurations (based on residuals) and stochastically inject them, with mild multiplicative and additive jitter, into the next batch (code: \texttt{sampler\_hard\_*}). This provides a curriculum–like pressure while avoiding overfitting to outliers.

\paragraph{Robustness tweaks.}
We trim extreme local–energy outliers by quantiles (default $3\%$) \emph{before} forming losses/updates; we also micro–batch the evaluation to keep memory bounded, and apply gradient clipping to $f_\theta$ (and backflow) parameters.

\section{Residual–based training (stage I)}
\label{sec:residual-stage}

For a batch $\{\tilde R_b\}_{b=1}^B$ we compute $E_L(\tilde R_b)$ and the batch mean $\mu=\frac{1}{B}\sum_b E_L(\tilde R_b)$.
The residual loss is
\begin{equation}
\mathcal{L}_{\rm res}(\theta; \alpha)
= \frac{1}{B}\sum_{b=1}^B \big(E_L(\tilde R_b)-E_{\rm eff}(\alpha)\big)^2,
\qquad
E_{\rm eff}(\alpha)=\alpha\,E_{\rm DMC}+(1-\alpha)\,\mu.
\end{equation}
\begin{itemize}
  \item \textbf{Self–target warmup ($\alpha=0$).} Minimizing $\mathbb{E}[(E_L-\mu)^2]$ is exactly variance minimization; it teaches the network to produce smooth, low–variance amplitudes inside nodal pockets and reduces sensitivity to early sampler noise.
  \item \textbf{Anneal to reference ($\alpha\uparrow$).} As capacity grows, we blend in the external target, gently steering the mean energy while retaining the variance pressure. The cosine ramp (code: \texttt{alpha\_start}, \texttt{alpha\_end}, \texttt{alpha\_decay\_frac}) avoids abrupt objective switches.
  \item \textbf{Pure energy fit ($\alpha=1$).} If desired, we can end stage I with $E_{\rm eff}=E_{\rm DMC}$.
\end{itemize}
In practice we compute $\nabla_\theta \mathcal{L}_{\rm res}$ by autodiff, use micro–batches of size \texttt{micro\_batch}, apply optional gradient clipping (code: \texttt{grad\_clip}), and log per–component usage and shell radii in physical units.

\paragraph{Local–energy derivatives in practice.}
We support three Laplacian backends:
\begin{enumerate}
  \item \textbf{Exact:} chunked second partials (reference–accurate, heavier).
  \item \textbf{HVP–Hutchinson:} $\Delta \log\Psi = \mathbb{E}_v[v^\top H_{\log\Psi} v]$ with Rademacher $v$ (fast and stable; fallback to FD if a probe is non–finite).
  \item \textbf{FD–Hutchinson:} centered finite differences of directional gradients (robust in corner cases; slightly noisier).
\end{enumerate}
We average a small number of probes (2–16 in ablations), and trim quantiles before forming losses.

\section{Stochastic Reconfiguration (stage II)}
\label{sec:sr-stage}

SR uses the covariance form of the energy gradient
\begin{equation}
\partial_{\theta_k}E 
= 2\,\mathrm{Cov}_\pi\!\big(E_L, O_k\big)
= 2\Big(\langle E_L O_k\rangle - \langle E_L\rangle\,\langle O_k\rangle\Big),
\end{equation}
and approximates a natural–gradient step by solving $(S+\lambda I)\Delta\theta=-g$.
Implementation details:
\begin{itemize}
  \item \textbf{Centering/whitening.} We center $O$ per batch and whiten by the diagonal of $S$ before CG to improve conditioning.
  \item \textbf{Trust region.} The final step is scaled to satisfy a maximum parameter norm (code: \texttt{max\_param\_step}).
  \item \textbf{Damping.} A small $\lambda$ (code: \texttt{damping}) stabilizes low–variance directions.
\end{itemize}
Because our backflow only alters the Slater coordinates (and $W_\theta$ stays on safe features of $\tilde R$), the SR statistics remain well–behaved even when nodes shift.

\section{Units, scaling, and batch construction}
\label{sec:units-scaling}

All features are computed in trap units $\tilde{\mathbf x}=\sqrt{\omega}\,\mathbf x$ so that length scales and gates (e.g.\ the short–range gate in the NQS and backflow) behave consistently across $\omega$. The sampler internally widens/narrows Gaussians and shell radii using $a_{\rm ho}=1/\sqrt{\omega}$ in a fixed, explicit way, so that the five components cover comparable \emph{dimensionless} regimes for $\omega\in[0.01,1]$. Batches are i.i.d.\ by design; we rely on the stratified mixture (with caps, EG–adaptation, and hard–injection) rather than long MCMC chains to traverse configurations, which markedly reduces correlation and simplifies accounting.

\section{Workflow summary}
\label{sec:opt-summary}

\begin{enumerate}
  \item \textbf{Sample collocation set} $X$ from the stratified capped–simplex mixture (center/tails/mixed/shells/dimers), then apply rotation and permutation; optionally inject hard examples.
  \item \textbf{Compute} $E_L$ using the selected Laplacian backend; trim outliers.
  \item \textbf{Stage I (residual).} Minimize $\frac{1}{B}\sum_b(E_L(\tilde R_b)-E_{\rm eff}(\alpha))^2$ with $\alpha$ ramped from $0$ to $\alpha_{\rm end}$; update mixture/shell weights via EG.
  \item \textbf{Stage II (SR).} Form $S$ and $g$, solve $(S+\lambda I)\Delta\theta=-g$, apply a trust–region step.
\end{enumerate}
This procedure keeps the early dynamics variance–dominated and sampler–aware, then transitions to a geometry–aware natural–gradient refinement. Empirically it yields stable convergence across $(N,\omega)$, including weak traps where shell/dimer coverage and permutation–respecting stratification are essential.

\chapter{Analysis}
\label{ch:analysis}

\subsection{Analysis toolbox (what we measure and why)}
\label{subsec:toolbox}
All analyses are performed on fresh $|\Psi_\theta|^2$ samples and use the same scaled coordinates.

\paragraph{Forward taps and branch ablations.}
We expose the intermediates of the correlator pipeline:
\[
z_\theta=\big[\overline\phi\;\big|\;\overline\psi\;\big|\;\mathrm{extras}\big],\quad
f_{\rm base}=\rho_\theta(z_\theta),\quad
f_{\rm cusp}=\sum_{i<j}\gamma_{\sigma_i\sigma_j}\,r^{\rm soft}_{ij},\quad
f=f_{\rm base}+f_{\rm cusp}.
\]
\textbf{Branch ablation} zeroes one block in $z_\theta$ (either $\overline\phi$, $\overline\psi$, or \textit{extras}) before the linear head and records the absolute drop in the head output. This identifies which source the head actually uses.

\paragraph{Effective rank of the representation.}
Let $\Sigma=\mathrm{Cov}[z_\theta]$ and $\{\lambda_i\}$ its eigenvalues under $|\Psi|^{2}$ sampling. With $p_i=\lambda_i/\sum_j\lambda_j$ we report the entropy effective rank
\[
r_{\rm eff}=\exp\!\Big(-\sum_i p_i\log p_i\Big),
\]
and the explained-variance spectrum. Low $r_{\rm eff}$ indicates that $z_\theta$ lies close to a low-dimensional manifold.

\paragraph{PC projection ablation (PCs vs.\ random subspaces).}
We compute PCA of $z_\theta$ and project onto the top-$k$ PCs, reconstructing the linear head $f_{\rm base}$ from the stored weights. We report the mean absolute error (MAE) relative to the full head, \emph{normalized} by $\mathrm{std}(f_{\rm base})$, and compare against projections onto random $k$-dimensional orthonormal subspaces. If PC$(k)$ attains tiny relative MAE while random$(k)$ does not, the head effectively depends on a $k$-dimensional latent.

\paragraph{Linear probes (semantics of PCs).}
We regress simple, interpretable summaries per configuration against the PC scores:
(i) mean pair distance $\bar r$,
(ii) variance of pair distances $\mathrm{Var}(r)$,
(iii) near-contact fraction $\Pr(r<r_0)$ in scaled units,
(iv) a crude ``shell contrast'' from single-particle radial histograms.
We report $R^2$ to assign physical meaning to the leading PCs (e.g., \emph{PC1 $\approx$ global size}).

\paragraph{PC1 block power shares.}
We decompose the loading vector of PC1 across the $(\overline\phi\,|\,\overline\psi\,|\,\mathrm{extras})$ blocks and report squared-norm shares that sum to one. This shows \emph{where} the dominant latent is encoded (typically in $\overline\psi$).

\paragraph{Near-field gradient concentration (stable quantile metric).}
For the \emph{learned residual} $f_{\rm base}$ we measure concentration of gradient energy near electron--electron contact via
\[
\mathrm{share}(q)=\frac{\big\langle\|\nabla f_{\rm base}\|^2\;\big|\;r_{\min}\le q\text{-quantile}\big\rangle}{\big\langle\|\nabla f_{\rm base}\|^2\big\rangle},
\]
evaluated at $q\in\{1\%,5\%,10\%\}$ with a minimum-count safeguard. Values $\approx 1$ indicate that the analytic cusp successfully offloads short-range stiffness; values $\gg 1$ indicate the residual carries additional near-field structure.

\paragraph{Means of $(f,f_{\rm cusp},f_{\rm base})$.}
We report $\mathbb{E}[f]$, $\mathbb{E}[f_{\rm cusp}]$, and $\mathbb{E}[f_{\rm base}]$ to summarize the division of labor without relying on variance decompositions that can be confounded by strong cross-covariances.

\subsection{What the toolbox reveals across confinement}
\label{subsec:omega-story}
For $N=6$ we observe a clear change with trap strength:
\begin{itemize}
  \item \textbf{Tight confinement ($\omega=1.0$):} $r_{\rm eff}\!\approx\!1$, top PC explains $\gtrsim\!99\%$ of variance in $z_\theta$, PC$(1)$ reconstructs the head to $\lesssim\!1\%$ relative MAE while random$(1)$ fails by orders of magnitude, and the head computed from PC1 correlates $\approx\!1$ with the full head. Linear probes assign \emph{PC1 $\approx$ mean separation $\bar r$}; near-field gradient shares are $\approx\!1$. Branch ablation identifies $\overline\psi$ as the dominant source. \emph{Interpretation:} the correlator is effectively one-dimensional and smoothly encodes \emph{global size}.
  \item \textbf{Loose confinement ($\omega=0.1$):} $r_{\rm eff}\!\approx\!2\text{–}3$, PC$(1)$ alone is insufficient, PC$(2)$ reaches $\sim\!5\%$ relative MAE and random$(2)$ remains much worse; head@PC1 correlation is $\ll 1$. Linear probes assign \emph{PC1 $\approx \bar r$} and \emph{PC2 $\approx \mathrm{Var}(r)$} (\emph{spread}); near-field shares increase modestly. $\overline\psi$ remains dominant. \emph{Interpretation:} the correlator remains low-dimensional but genuinely depends on two smooth latents: \emph{size} and \emph{spread}.
\end{itemize}

\subsection{Uncertainty, reporting, and sanity checks}
We report sample means and standard errors $\mathrm{se}=\sigma/\sqrt{B}$ over the evaluation batch. All randomness (initialization, MCMC, probe vectors) uses fixed seeds. We verify:
(i) agreement between \texttt{hvp-hutch} and exact Laplacian within SE,
(ii) acceptance in the target window with adequate mixing,
(iii) stability of $r_{\rm eff}$ under resampling,
(iv) robustness of near-field conclusions under small changes of the soft-core $\delta$ and quantile levels.

% \section{Normalizing Flow}
% In training physics-informed neural networks (PINNs) to solve complex problems such as the many-body Schrödinger equation, the quality and distribution of training samples are critical. Conventional sampling techniques—such as narrow normal, broad normal, and uniform sampling—each have significant drawbacks. For example:
% \begin{itemize}
%     \item \textbf{Narrow Normal Sampling:} Concentrates samples near the center of the domain, leading to insufficient coverage at the boundaries where the solution may still be significant.
%     \item \textbf{Broad Normal Sampling:} Oversamples regions with little physical relevance, thus wasting computational resources.
%     \item \textbf{Uniform Sampling:} Provides inflexible coverage, often missing regions of high interest such as nodes or sharp gradients.
% \end{itemize}
% To overcome these issues, we adopt a normalizing flow framework. This method transforms a simple base distribution (typically a standard normal distribution) into a distribution that better aligns with the target function (e.g., the wavefunction). By learning an invertible mapping, normalizing flows adaptively concentrate samples where they are most needed, improving both efficiency and training accuracy.

% \begin{figure}[H]
%     \centering
%     \begin{subfigure}[t]{0.49\textwidth}
%         \centering
%         \includegraphics[width=0.9\textwidth]{Normalizing-Flow/Narrow_Norm.pdf}
%         \caption{Narrow Normal Distribution Oversampling the Central Region, and Broad Normal Distribution oversampling near the edges}
%         \label{fig:Normal_sampling}
%     \end{subfigure}
%     \hfill
%     \begin{subfigure}[t]{0.49\textwidth}
%         \centering
%         \includegraphics[width=0.9\textwidth]{Normalizing-Flow/Flow2.pdf}
%         \caption{Sampling with Normalizing Flow, reproducing the Slater Determinant}
%         \label{fig:Flow_sampling}
%     \end{subfigure}
% \end{figure}
% \section{Normalizing Flow Sampling}
% Normalizing flows work by constructing an invertible transformation \( T \) that maps a sample \( x \) drawn from a simple base distribution \( p_X(x) \) (e.g., \( x \sim \mathcal{N}(0, I) \)) to a sample \( y = T(x) \) that follows the target distribution \( p_Y(y) \). The change-of-variables formula governs this mapping:
% \[
% p_Y(y) = p_X(x) \left| \det \left( \frac{\partial T^{-1}(y)}{\partial y} \right) \right|.
% \]
% In our approach, we use a variant known as \emph{conditional flow matching}. Rather than learning the transformation \( T \) directly, we conceptualize the mapping in terms of a displacement field. Given a base sample \( x \) and an associated target sample \( y \), the ideal displacement is defined as:
% \[
% v_{\text{true}} = y - x.
% \]
% We then introduce a continuous interpolation parameter \( t \in [0,1] \) and define:
% \[
% \psi_t = (1-t)x + t\, y,
% \]
% which smoothly bridges the base and target distributions. Our goal is to learn a function \( v(\psi_t, t) \) that predicts the required displacement, thereby reconstructing the overall mapping \( T \).

% \section{Training Procedure}
% The normalizing flow model is trained to capture the transformation dynamics from the simple base distribution to the complex target distribution. The training process comprises the following steps:

% \begin{enumerate}
%     \item \textbf{Sample Generation:} 
%     \begin{itemize}
%         \item Generate a batch of base samples \( x \) from a standard normal distribution \( \mathcal{N}(0, I) \).
%         \item Obtain target samples \( y \) using a reliable method (e.g., via a Metropolis--Hastings algorithm or from experimental data) that reflects the desired distribution.
%     \end{itemize}
    
%     \item \textbf{Interpolation:} 
%     For each pair \( (x, y) \), select a random interpolation parameter \( t \in [0,1] \) and compute the interpolated state:
%     \[
%     \psi_t = (1-t)x + t\, y.
%     \]
    
%     \item \textbf{Displacement Prediction and Loss Calculation:}
%     A neural network—typically a multilayer perceptron (MLP)—is employed to predict the displacement \( v_{\text{predicted}}(\psi_t, t) \). The network is trained by minimizing the mean squared error (MSE) between the predicted and true displacements:
%     \[
%     \mathcal{L} = \mathbb{E}_{x,y,t} \left[ \| v_{\text{predicted}}(\psi_t, t) - (y - x) \|^2 \right].
%     \]
    
%     \item \textbf{Optimization:} 
%     We use an optimizer such as Adam to update the network parameters based on the computed loss. Training is performed over multiple epochs until convergence is observed. Once trained, the model is capable of rapidly generating uncorrelated samples that align with the target distribution.
% \end{enumerate}

% \section{Advantages and Implementation Details}
% The use of normalizing flows offers several advantages over traditional sampling methods:
% \begin{itemize}
%     \item \textbf{Adaptive Sampling:} The learned transformation focuses sampling in regions of high importance, avoiding both undersampling and oversampling.
%     \item \textbf{Efficiency:} Unlike methods requiring a burn-in period (as in Markov Chain Monte Carlo), the normalizing flow generates independent samples quickly once training is complete.
%     \item \textbf{Automation:} The framework removes the need for manual hyperparameter tuning of the sampling distribution, making it robust to changes in system parameters.
% \end{itemize}

% In our implementation, we integrate the normalizing flow sampling within the overall training pipeline of the PINN. Hyperparameters such as learning rate, batch size, and network architecture (e.g., number of hidden layers and neurons) are chosen based on preliminary experiments to ensure efficient convergence. Sample quality is monitored via validation metrics that compare the generated distribution against the target distribution.


\part{Results}
\section{Overview}
Write your results here.


\part{Discussion}
\chapter{Quantum Dots}
\section{Discussion of energy results}
\label{sec:energy-discussion}

Taken together, Tables~\ref{tab:Residual} and~\ref{tab:energies} show that the
Slater--Jastrow PINN (with and without backflow) is able to reproduce
state-of-the-art DMC energies across a fairly demanding grid in $(N,\omega)$,
and to extrapolate in a controlled way into regimes where no DMC reference is
available (the ultra-weak traps at $\omega=10^{-3}$).

\paragraph{Two electrons: residual-only vs.\ SR-refined.}
For $N{=}2$, residual-only training already provides an excellent test of the
bare architecture. Table~\ref{tab:Residual} shows that across
$\omega\in\{1.0,0.5,0.1,0.01\}$ both the plain PINN and PINN+BF track the DMC
references at the $\sim 10^{-4}$\,Ha level or better, with deviations well
within quoted uncertainties except for the ultra-shallow $\omega{=}0.01$
pretraining point. There, residual-only PINN+BF remains safely variational
(energies slightly above DMC) but sits about $3\times 10^{-4}$\,Ha high,
indicating that some residual bias remains after the residual stage alone.

The final SR-refined energies in Table~\ref{tab:energies} remove this bias:
for $N{=}2$ and all $\omega\ge 0.01$, the PINN+BF results are statistically
indistinguishable from DMC, with relative deviations at or below
$10^{-3}\,\%$. This confirms that, given a well-conditioned ansatz, a short
SR tail is sufficient to close the last few $10^{-4}$\,Ha of the variational
gap in this benchmark system.

The comparison between PINN and PINN+BF in Table~\ref{tab:Residual} also
supports the claim that backflow is \emph{not necessary} to reach DMC-level
accuracy for two electrons. Energies with and without backflow agree within
error bars at all but the loosest trap, and even there the improvement from
backflow is small compared to the SR refinement. In practice, the main role
of backflow in the $N{=}2$ case is to reduce the variance and stabilize
optimization in the shallow-trap regime, rather than to provide a qualitatively
new nodal structure.

\paragraph{Scaling to $N{=}6$ and $N{=}12$.}
The more demanding tests are the many-electron dots in
Table~\ref{tab:energies}. For $N\in\{6,12\}$ and the traps where DMC
references are available ($\omega\in\{0.1,0.5,1.0\}$), the final PINN+BF
energies are consistently close to DMC and always slightly higher, as
expected from a variational method. The relative deviations lie in the
range $\sim 0.026\text{--}0.048\%$ for the interacting cases, i.e.\ on
the order of a few $10^{-4}$\,Ha in absolute terms. There is no sign of a
catastrophic growth in error with either particle number or trap strength:
the worst relative deviations occur at intermediate $\omega$ and remain
comfortably below the percent level.

This behaviour is non-trivial for two reasons. First, the dots span very
different physical regimes: from relatively compact, kinetic-energy-dominated
states at $\omega=1.0$ to highly correlated, interaction-dominated states at
$\omega=0.1$. Second, the same architecture and training protocol are used
across all $(N,\omega)$, up to trivial scaling in trap units. The fact that
a single Slater--Jastrow+backflow ansatz can maintain sub-$0.1\%$ accuracy
across this grid suggests that the chosen combination of soft-core pair
features, analytic cusp, and trap-unit normalization is robust to both
particle number and confinement.

\paragraph{Ultra-weak traps: predictions at $\omega=10^{-3}$.}
For the ultra-weak traps at $\omega=10^{-3}$ there are no DMC references
in Table~\ref{tab:energies}; the PINN+BF values
($E\approx 0.0138$\,Ha for $N{=}2$, $0.1409$\,Ha for $N{=}6$,
$0.5158$\,Ha for $N{=}12$) should be regarded as high-quality variational
predictions rather than benchmarked numbers. Several consistency checks
nevertheless support their reliability:

\begin{itemize}
  \item The energies connect smoothly to the DMC-validated points at
        $\omega=0.01$ under the log--log scaling analysis of $T$, $V_{\rm int}$,
        and $V_{\rm trap}$ in Sec.~\ref{sec:wigner-molecule}: the inferred
        exponents for $E(\omega)$ and the partitioning between trap and
        interaction energies are consistent with classical expectations in
        the Wigner regime.
  \item For $N{=}2$ at $\omega=10^{-3}$, independent MCMC diagnostics
        (stable $r_{12}$ histograms, consistent one-body densities, smooth
        local-energy distributions) indicate that the sampling has converged
        and the residual statistical error is well below the quoted
        uncertainties in Table~\ref{tab:energies}.
  \item For $N{=}6$ and $N{=}12$, the structural and virial diagnostics in
        Sec.~\ref{sec:wigner-molecule} confirm that the states are in the
        expected Wigner-molecule regime (correct shell topologies,
        near-classical virial ratios), with no signs of pathologies that
        would hint at an incorrect energy scale.
\end{itemize}

Within these caveats, the $\omega=10^{-3}$ values can be taken as reliable
anchors for the Wigner-molecule analysis that follows.

\paragraph{Sample efficiency and conditioning.}
From an algorithmic perspective, these results also speak to the
\emph{efficiency} of the chosen training scheme. The SR tail uses on the
order of $10^3$ iterations with $\sim 3\times 10^3$ samples each, i.e.\ a
few million local-energy evaluations per $(N,\omega)$ point. This is modest
compared to some earlier neural approaches to quantum dots that require
orders of magnitude more samples to reach comparable accuracy.
The good performance of residual-only pretraining (especially at $N{=}2$)
and the small SR corrections suggest that most of the work is done by
learning a well-conditioned, physically informed ansatz rather than by
brute-force stochastic optimization.

The conditioning choices are crucial here:
working in trap units keeps the typical coordinates $\mathcal{O}(1)$ from
tight to shallow traps; soft-core pair features with $ds/dr\to 0$ at
coalescence and an explicit analytic cusp prevent large gradients and
Laplacian spikes near $r_{ij}=0$. As a result, local energies remain
numerically well-behaved even in the ultra-weak traps where electrons are
far apart and the wavefunction varies on long length scales.

\paragraph{Role of backflow across regimes.}
Finally, these tables let us separate the \emph{energetic} role of backflow
from its structural and representational role. For $N{=}2$ backflow has
negligible impact on the final SR-refined energies and is not required for
DMC-level accuracy. For $N\in\{6,12\}$ at intermediate traps
($\omega\in\{0.1,0.5\}$), backflow contributes a noticeable but still modest
improvement over the residual-only baselines (not shown in
Table~\ref{tab:energies}), bringing the final errors into the
$\mathcal{O}(10^{-2}\text{--}10^{-1})\%$ band. In the deepest Wigner
regimes ($\omega=10^{-3}$) the backflow correction is energetically tiny
(see Sec.~\ref{sec:repr-analysis}); here the base Slater--Jastrow correlator
already captures the essential structure and backflow becomes nearly
variationally inert.

This regime dependence will be explored in more detail in
Sec.~\ref{sec:wigner-molecule} and Sec.~\ref{sec:repr-analysis}, where we
show how the same ansatz that yields the accurate energies in
Table~\ref{tab:energies} also reproduces the expected Wigner-molecule
structure and compresses it into a low-dimensional correlator manifold
with a high-dimensional, largely local backflow correction.


\section{Discussion of the Wigner–molecule regime}
\label{sec:wigner-discussion}

The structural diagnostics in Sec.~\ref{sec:wigner-molecule} show that, for the
weakest traps considered, the dots are genuinely on the Wigner side and not
merely ``strongly correlated'' in a loose sense. Three complementary pieces of
evidence support this: (i) localization and energy partitioning, (ii) shell
geometry and bond order, and (iii) the statistics of competing topologies.

\paragraph{Localization and energy partitioning.}
On the one-body level, the weak-trap radial laws $P(r)$ are highly localized
around large radii with small relative fluctuations. For the two-electron dot
at $\omega=10^{-3}$ we find
$r_{\rm mode}\approx 64.3$, $\langle r\rangle\approx 64.0$ and
$\gamma=\sigma_r/r_{\rm mode}\approx 0.29$, with only
$\sim 4.6\times 10^{-3}$ of the single-particle mass inside
$r\le 0.25\,r_{\rm mode}$. The pair distribution $p(r_{12})$ is peaked at
$r_{12}^{\rm mode}\approx 122$ with a Lindemann ratio
$\gamma_{r_{12}}\approx 0.20$, and the relative angle is sharply
anti-aligned, $\sigma(\pi-\Delta\phi)\approx 0.35$, with
$p(r_{12}\le 0.25\,r_{12}^{\rm mode})\approx 2.6\times 10^{-3}$.
These numbers are characteristic of a dilute two-site Wigner dimer rather
than a delocalized Fermi liquid.

For $N\in\{6,12\}$, the same pattern reappears in the aggregate:
$r_{\rm mode}$ and $r_{\rm rms}$ grow roughly as $\omega^{-0.6}$,
$\gamma$ decreases only mildly with $N$ and $\omega$, and the inferred
$\langle r^2\rangle\propto\omega^{-\beta}$ with
$\beta\simeq 1.2\text{--}1.3$ exceeds the non-interacting $\beta=1$.
At the same time, the energy partitioning approaches the classical
virial-like ratio:
\[
\frac{2\langle V_{\rm trap}\rangle}{\langle V_{\rm int}\rangle}
\simeq 1.25,\;1.12,\;1.02 \quad\text{for}\quad N=2,6,12
\quad \text{at }\omega=10^{-3},
\]
with $N=12$ satisfying $V_{\rm int}\approx 2V_{\rm trap}$ to within~$2\%$.
Combined with the log--log scaling exponents $\alpha_T\approx 1$ and
$\alpha_{V_{\rm int}}\approx 0.66$, this indicates that the kinetic energy
has become a small correction on top of a nearly classical potential-energy
landscape, as expected in the Wigner-molecule limit.

\paragraph{Shell geometry and bond order.}
Beyond global localization, the weak-trap states exhibit clear shell
structure with non-trivial bond order. For $N{=}6$ the dot retains a
single pronounced shell across the entire $\omega$ grid; what changes
is the stiffness of angular correlations. The ring Lindemann index drops
from $\sim 0.70$ at $\omega=1$ to $\sim 0.64$ at $\omega=10^{-3}$, while
the pair-distance Lindemann ratio decreases from $\sim 0.35$ to $\sim 0.27$
(see Table~\ref{tab:N6}). After angular registration, the electrons form a
rotating Wigner ring with well-defined sixfold order.

At $\omega=10^{-3}$, the $N{=}6$ ensemble is dominated by two symmetry
classes, $(1,5)$ and $(0,6)$, with fractions
$\approx 0.801$ and $\approx 0.199$.
Both show strong ring order:
$|\Phi_5|\approx 0.60$, $|\Phi_6|\approx 0.66$ and
ring Lindemann $\approx 0.21$.
The fact that the shell never breaks, and the system fluctuates only between
these two topologies, matches classical predictions where $(1,5)$ is the
ground-state geometry and $(0,6)$ a low-lying competitor.

For $N{=}12$, the situation is richer but equally structured. At
$\omega=10^{-3}$, the two-shell fraction is very high across a range
of radial-gap thresholds ($\sim 0.99,0.97,0.92$ for $\tau=2.0,2.5,3.0$),
indicating that shelling is robust rather than a threshold artifact.
Within the two-shell subset, the inner-ring occupancy histogram is sharply
peaked at $(3,9)$, with $(2,10)$ and $(1,11)$ providing the leading
subdominant topologies. On $(3,9)$ frames we find
$|\Phi_3|\approx 0.69$, $|\Phi_9|\approx 0.53$ and
Lindemann indices $0.22$ on both rings, i.e.\ clear threefold and ninefold
bond order with small angular fluctuations. This is precisely the type of
commensurate, shell-resolved crystalline order expected in a twelve-electron
Wigner molecule.

\paragraph{Topology statistics and reconstruction of $g(r)$.}
A particularly stringent check is the reconstruction of $g(r)$ from
shell-resolved sectors. For $N{=}12$ at $\omega=10^{-3}$ we partition
the two-shell configurations into II/IO/OO pairs, weight their
histograms by the combinatorial mixture $(0.045{:}0.409{:}0.545)$,
and compare the resulting synthetic $g(r)$ to the global one. The cosine
similarity $0.9982$ indicates that, within statistical resolution, the
global pair correlations are completely explained by the superposition of
shell-resolved contributions and combinatorics. The same holds for $N{=}6$,
where OO/IO mixtures of the $(1,5)$ and $(0,6)$ sectors reproduce $g(r)$
essentially exactly.

This has two implications. First, the observed structure in $g(r)$ is
unambiguously geometric (shells and topologies), rather than a subtle
finite-size or sampling artifact. Second, it shows that the neural ansatz
is not just matching energies: it is reproducing the full two-body
correlation structure expected from classical and quantum analyses of
parabolic Coulomb clusters~\cite{Egger_1999,Filinov_2001,schweigert1994spectralpropertieschargedparticles,Kong_2002}.

\paragraph{From dimer to multi-shell Wigner molecules.}
Looking across $N=2,6,12$ and decreasing $\omega$, a coherent picture
emerges. The two-electron system evolves into a dilute, strongly
anti-aligned dimer with negligible overlap near the origin. The six-electron
dot forms a single, stiff Wigner ring whose lab-frame symmetry is restored by
rotation but whose co-rotating frame shows clear sixfold order and a quantum
mixture of classically known $(1,5)$ and $(0,6)$ topologies. The twelve-electron
dot, finally, develops a robust two-shell structure with a commensurate
$(3,9)$ split as the modal configuration and quantitative agreement with
classical ground-state topologies.

The fact that all of this structure appears in samples drawn from a single
variational neural ansatz---whose energies match or closely track DMC
benchmarks---gives confidence that the states studied in
Sec.~\ref{sec:repr-analysis} are bona fide Wigner molecules. In the next
section we show that, despite this rich real-space structure, the learned
correlator compresses the Wigner regime into a remarkably low-dimensional
manifold and that backflow plays a strongly regime-dependent role, becoming
almost variationally inert in the deepest Wigner cases.

\section{Discussion of learned representations}
\label{sec:repr-discussion}

The analysis in Sec.~\ref{sec:repr-analysis} shows a surprisingly coherent picture
across all $(N,\omega)$: despite the rich real-space structure documented in
Sec.~\ref{sec:wigner-molecule}, the learned correlator $f_{\text{net}}$ lives
on a very low-dimensional manifold, while backflow realizes a comparatively
high-dimensional but strongly localized correction that becomes almost
variationally inert in the deepest Wigner regimes.

\paragraph{Low-dimensional correlator manifold.}
Across all systems, the entropy effective rank $r_{\rm eff}(Z)$ of the
correlator features remains close to~1, rarely exceeding~2
(Table~\ref{tab:fnet-geometry-upd}). For the most extreme case,
$N{=}2$ at $\omega=10^{-3}$, we find $r_{\rm eff}(Z)\approx 1.00$ and a
head--PC1 correlation of $\approx 0.998$, with PC1 block power almost
entirely in the pair branch $\psi$. The PC ablations confirm that this is
not an artifact of the metric: projecting inputs onto the top-$k$ PCs
reduces the head error dramatically (e.g.\ to $\mathcal{O}(10^{-2})$ for
$k\le 4$) whereas random $k$-dimensional subspaces leave almost all of the
error intact (Table~\ref{tab:head-ablations}). In other words, most of what
the head does is encoded along a single, very special latent axis.

The same pattern holds, with mild variations, for $N\in\{6,12\}$ and for
both shallow and intermediate traps. Effective ranks cluster between
$\sim 1.2$ and $\sim 2.0$, head--PC1 correlations stay above~0.95 in most
cases, and PC ablations show that a handful of principal directions capture
the bulk of the head sensitivity much more efficiently than random
subspaces. From a representation-learning perspective, the correlator thus
behaves more like a low-dimensional order-parameter model than a generic
high-dimensional feature extractor.

\paragraph{What the leading axes encode.}
Linear probes from the correlator PCs to coarse physical summaries clarify
what these axes actually represent. For $N{=}2$ (both at $\omega=0.01$ and
$10^{-3}$) the first few PCs predict $r_{\rm mean}$ and $r_{\rm var}$ essentially
perfectly ($R^2\approx 1$), and even the near-origin mass and
$P(r_{12}\le0.25\,r_{12}^{\rm mode})$ are well captured. In the six- and
twelve-electron dots the situation is slightly more complex, but the trend
is the same: $R^2(r_{\rm mean})$ remains close to unity and
$R^2(r_{\rm var})$ lies in the $0.4$–$0.85$ range across regimes, whereas
shell-contrast is only weakly linearly encoded
(Table~\ref{tab:fnet-geometry-upd}). This is consistent with the structural
analysis in Sec.~\ref{sec:wigner-molecule}: the correlator primarily tracks
global size and radial spread (how big is the dot, how swollen are the
shells), while the detailed angular order requires explicit registration of
configurations within a given shell topology.

\paragraph{Backflow as a local, high-dimensional correction.}
In contrast, the BackflowNet lives in a genuinely high-dimensional regime.
For $N{=}6$ and $N{=}12$ the displacement field $\Delta x$ has
$r_{\rm eff}(\Delta x)$ in the range $\sim 10$–$22$
(Table~\ref{tab:bf-geometry-upd}), with relatively flat PCA spectra and
gradual error reduction as more PCs are retained. Linear probes from $\Delta x$
PCs to global observables are essentially null: $R^2$ for $r_{\rm mean}$,
$r_{\rm var}$, and shell contrast stays near zero, indicating that the
backflow field does not encode simple global rescalings or shape parameters.

Near-field diagnostics for $\Delta E$ and $\|\Delta x\|^2$ also support a
local interpretation. In intermediate traps (e.g.\ $N{=}6$, $\omega=0.01$,
or $N{=}12$, $\omega=0.01$), the lowest $q\%$ of configurations by
$r_{\min}$ carry a disproportionately large share of the backflow energy
correction: near-field $\Delta E$ shares at $q{=}1\%$ can exceed unity by
factors of $2$–$3$, and remain enhanced at $q{=}5\%$ and $10\%$
(Table~\ref{tab:bf-geometry-upd}). This is precisely where backflow is
expected to matter physically: when electrons approach each other and small
nodal adjustments can strongly affect the local energy.

\paragraph{Deep Wigner limit: backflow switches off.}
The extreme Wigner points provide an instructive counterexample. For
$N{=}2$ at $\omega=10^{-3}$, the backflow correction to the energy is
$\Delta E\sim 2\times 10^{-7}$\,Ha with uncertainty of the same order, the
PC1 alignment between no-BF and BF correlator features is essentially
perfect (cosine $=1$), the ranks of $Z$ and $\Delta Z$ are both $\approx 1$,
and the BackflowNet displacement field has vanishing impact on observables:
channel ablations produce no change, near-field $\|\Delta x\|^2$ shares are
exactly unity, and even the PCA of $\Delta x$ reveals a nearly trivial
structure. In this regime, backflow is effectively turned off by the
optimizer; the Slater--Jastrow correlator is sufficient to represent the
Wigner dimer.

For $N{=}6$ and $N{=}12$ at $\omega=10^{-3}$ the pattern is similar, albeit
less extreme. The correlator manifold remains low-rank, PC1 alignment
between no-BF and BF stays near one, and the entropy rank of $\Delta Z$ is
close to that of $Z$. Backflow still spans a higher-dimensional space in
principle, but its energetic effect is very small and its near-field energy
shares are close to unity. In other words, once the system is deep in the
Wigner-molecule regime and the static correlator has adapted to the
interaction-dominated landscape, backflow becomes nearly variationally
inert.

\paragraph{Regime dependence and architectural implications.}
Putting these observations together, the learned representation exhibits a
clear regime dependence:

\begin{itemize}
  \item In the \emph{weakly to moderately correlated} regime (tighter traps,
        smaller $r_s$), the correlator is still low-rank but backflow
        carries a non-trivial fraction of the energy correction, concentrated
        in near-field configurations. Here, widening the BackflowNet and
        providing rich pairwise inputs ($r_{ij}$, $\mathbf r_{ij}$, etc.)
        pays off, while the correlator can remain compact.

  \item In the \emph{deep Wigner} regime (ultra-weak traps at
        $\omega=10^{-3}$), the correlator alone captures both the global
        size/radius and the shell structure; backflow collapses to a
        nearly idle high-dimensional field. Extra expressivity in backflow
        does not translate into better energies or visibly different
        correlations.
\end{itemize}

From a design perspective, this suggests that Slater--Jastrow PINNs with
small, smooth correlators and moderately wide backflow layers are a good
default for parabolic dots. The correlator behaves like a learned
few-parameter ``collective coordinate'' summarizing the dot size and rough
shelling, while backflow functions as a flexible local nodal corrector that
only activates in regimes where near-field physics is delicate. In the
Wigner limit backflow naturally switches off, and the model reduces to a
static Wigner-molecule ansatz with low-rank latent geometry.

\paragraph{Relation to the physical picture.}
Finally, the representation analysis dovetails with the structural results
of Sec.~\ref{sec:wigner-molecule}. The same correlator that reproduces the
expected shell topologies, bond order, and virial partitioning in the
Wigner regime is found to live on a one- or two-dimensional manifold whose
leading axis correlates almost perfectly with global size and spread.
Backflow, in turn, is high-dimensional and local exactly in the regimes
where the shell structure is less rigid and near-field encounters are more
frequent. This supports the view that the ansatz has learned a physically
meaningful decomposition: a small set of global ``collective coordinates''
captured by $f_{\text{net}}$ and a local, many-body correction channel
implemented by BackflowNet.


\part{Conclusion}
\section{Overview}
Write your conclusion here.


%\bibliographystyle{johd}
\bibliographystyle{unsrtnat} 
\bibliography{references}

\end{document}
