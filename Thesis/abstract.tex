\chapter*{Abstract}
How far can a \emph{compact, well-conditioned} neural correlator go when layered on an exactly antisymmetric Slater reference, and what extra structure does a lightweight, symmetry-faithful \emph{backflow} learn? We answer this for interacting electrons in two-dimensional harmonic traps over \((N,\omega)\in\{(2,1\!\to\!10^{-3}), (6,1\!\to\!10^{-3}), (12,1\!\to\!10^{-3})\}\). Our ansatz is a Slater–Jastrow with (i) \(\sqrt{\omega}\) coordinate scaling (trap units), (ii) \emph{soft-core} pair features with vanishing radial derivatives at coalescence, and (iii) an \emph{analytic} electron–electron cusp; a compact BackflowNet supplies a smooth, radial displacement field \(\Delta x\) that deforms nodes while preserving the cusp. Training uses brief residual preconditioning followed by stochastic reconfiguration.

Beyond energies, we quantify the \emph{emergent geometry} of the learned states with automatic shell detection, shell-resolved reconstructions of \(g(r)\) (cosine similarity \(0.989\!-\!1.000\) to the global histogram), bond-orientational order parameters \(|\Phi_m|\), and angular Lindemann-like ratios \(\lambda_\phi\) (ring “stiffness”). Three findings stand out:  
(1) For \(N{=}2\), decreasing \(\omega\) yields a Wigner dimer with a sharp \(r_{12}\) peak, tight anti-alignment near \(\pi\), and \(\gamma=\sigma_r/r_{\rm mode}\downarrow\) (\(r_{\rm mode}\!\approx\!63.3\,a_0\) at \(\omega\!=\!10^{-3}\)).  
(2) For \(N{=}6\), a single-ring Wigner molecule persists down to \(\omega\!=\!10^{-3}\) with a robust near-degeneracy between \((1,5)\) and \((0,6)\) sectors (fractions \(\approx 0.63\!-\!0.73\) vs.\ \(0.27\!-\!0.37\) across runs), comparable \(|\Phi_5|\) and \(|\Phi_6|\), and small \(\lambda_\phi\).  
(3) For \(N{=}12\), a two-shell Wigner molecule forms; at \(\omega\!=\!10^{-3}\) the \((3,9)\) occupancy is \emph{modal} under standard gap thresholds (\(\sim\!35\%\)) but significant weight remains in \((2,10)\) and \((1,11)\), demonstrating a \emph{rotating Wigner molecule} where several classical minima are quasi-degenerate. Inner/outer rings show strong orientational order (\(|\Phi_3|\!\sim\!0.67\), \(|\Phi_9|\!\sim\!0.50\)) and low \(\lambda_\phi\).

Energetically, PINN\(+\)BF matches diffusion Monte Carlo (DMC) within uncertainties for \(N{=}2\) and is within \(\sim 10^{-2}\!\!-\!10^{-1}\%\) relative for \(N\in\{6,12\}\). The methodological lesson is that once antisymmetry, cusp, and \emph{conditioning} (trap units, safe features) are explicit, \emph{conditioning—not depth—}governs accuracy; backflow then supplies high-dimensional, local nodal refinements. Physically, we provide, to our knowledge, the first topology-resolved, bond-orientational and stiffness metrics—together with shell-mixture reconstructions of \(g(r)\)—for parabolic dots across the Fermi-liquid \(\to\) Wigner-molecule crossover, consistent with and extending prior theory of the crossover and classical shelling \cite{Egger_1999,Mazars_2008,Kong_2002,schweigert1994spectralpropertieschargedparticles,Filinov_2001,manninen2007metalclustersquantumdots}.
\clearpage