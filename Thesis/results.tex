\graphicspath{{../results/figures/results/}}

\chapter{Quantum Dots}

We study two-dimensional harmonic quantum dots, a numerically delicate setting
due to the Coulomb singularity at short range and the second-order derivatives
in the kinetic term. Both effects can yield ill-conditioned objectives that
typically require many epochs and small learning rates. We consider
two-electron, six-electron and twelve-electron dots across a wide confinement
range ($\omega\!\in\!\{1.0,\,0.5,\,0.1,\,0.01,\,0.001\}$). 
When $\omega$ becomes sufficiently small, the system approaches the Wigner
molecule limit, where inter-electron distances become large compared to the
confinement length scale and the electrons localize into geometric
arrangements that minimize their mutual repulsion while balancing the trap
potential.
As $\omega$ decreases, the well broadens and correlations strengthen; 
nevertheless, the compact Slater--Jastrow PINN with soft-core pair features
and an explicit analytic cusp remains stable and accurate in practice.
In addition, we provide a detailed analysis of the learned representations,
focusing on the correlator manifold and the backflow field.

\section{Training protocol and energy overview}
Unless otherwise stated, we first apply residual-based pretraining to obtain a
well-conditioned initializer, and then run a Stochastic Reconfiguration (SR)
tail to refine the last digits of the energy. In the residual phase, the
network initially chooses its own target energy as a stabilizer, before we
switch to the DMC reference energy (and optionally add a variance penalty).
The SR phase then uses on the order of $10^3$--$3\times10^3$ iterations with
batches of $\sim 3{\times}10^3$ samples per iteration.

\subsection{Energies}

Table~\ref{tab:Residual} reports two-electron energies from
\emph{residual-only} training (PINN and PINN+BF). For $N{=}2$, backflow is not
required to reach DMC accuracy, but it slightly reduces the variance and
smooths convergence.

\begin{table}[h!]
\centering
\caption{Ground-state energies (Hartree) for two electrons in a harmonic trap
using \emph{residual-only} training. DMC references as in [Ref.]. Parentheses
denote $1\sigma$ on the last digits.}
\label{tab:Residual}
\begin{tabular}{c c c c}
\hline
$\omega$ & PINN+BF & PINN & DMC (Ref.) \\
\hline
1.00  & 2.99998(3)   & 2.999940(5)  & 3.00000(1) \\
0.50  & 1.65975(2)   & 1.65979(3)   & 1.65977(1) \\
0.10  & 0.440795(3)  & 0.440833(1)  & 0.44079(1) \\
0.01  & 0.0740724(6) & 0.0741679(6) & 0.073839(2) \\
\hline
\end{tabular}
\end{table}

Table~\ref{tab:energies} compares diffusion Monte Carlo (DMC) references with
our \emph{final} PINN+BF energies after residual pretraining and an SR tail.
From the ultra–shallow two-electron case ($\omega{=}0.01$) to the many-electron
systems, absolute deviations are small; for $N{=}2$ they lie within DMC
uncertainties, and for $N\!\in\!\{6,12\}$ the relative deviations are 
typically in the range $0.01$–$0.08\%$ (see also
Fig.~\ref{fig:rel_error_vs_omega}).

\begin{table}[htbp]
  \centering
  \caption{Residual-based pretraining + SR tail. Reference DMC and our PINN+BF
  ground-state energies (Hartree) for selected $(N,\omega)$. Parentheses denote
  $1\sigma$ on the last digit(s). The final column is the relative deviation
  from DMC (\%).}
  \label{tab:energies}
  \begin{threeparttable}
    \begin{tabular}{c|c|c|cc|cc}
      \toprule
      $N$ & $\omega$ & Reference (DMC) & PINN+BF & \%\,err & PINN+CTNN & \%\,err \\
      \midrule
      2 & 0.001 & --- & $0.0137948(8)$ & --- & $\mathbf{0.013778(1)}$ & --- \\
        & 0.01 & $0.073839(2)$ & $0.073838(1)$ & $-0.0014$  & $0.0738382(5)$ & $-0.0014$ \\
        & 0.10 & $0.44079(1)$  & $0.44079(1)$  & $+0.0000$  & $0.440796(4)$ & $-0.0000$  \\
        & 0.50 & $1.65977(1)$  & $1.65976(2)$  & $-0.0006$  & $1.65976(1)$ & $-0.0006$ \\
        & 1.00 & $3.00000$     & $2.99999(1)$  & $-0.0003$  & $3.00000(2)$ & $-0.0000$  \\
      \midrule
      6 & 0.001 & ---           & $0.140913(1)$ & --- & $\mathbf{0.140832(1)}$ & --- \\
        & 0.01 & ---            & $0.69125(2)$  & --- & $\mathbf{0.69036(1)}$ & --- \\
        & 0.10 & $3.55385(5)$   & $3.5549(1)$   & $+0.0295$ & $\mathbf{3.55388(5)}$ & $+0.0008$ \\
        & 0.50 & $11.78484(6)$  & $11.7895(4)$  & $+0.0395$  & $\mathbf{11.7847(2)}$ & $-0.0000$ \\
        & 1.00 & $20.15932(8)$  & $20.1610(6)$  & $+0.0083$  & $\mathbf{20.1585(3)}$ & $-0.0041$ \\
      \midrule
      12 & 0.001 & ---          & $0.515823(3)$ & --- & $\mathbf{0.515412(3)}$ & --- \\
         & 0.01 & ---           & $2.48620(5)$  & --- & $\mathbf{2.47363(4)}$ & --- \\
         & 0.10 & $12.26984(8)$ & $12.2731(2)$  & $+0.0266$ & --- & --- \\
         & 0.50 & $39.1596(1)$  & $39.1786(8)$  & $+0.0485$ & $\mathbf{39.1604(3)}$ & $+0.0018$ \\
         & 1.00 & $65.7001(1)$  & $65.717(1)$   & $+0.0257$ & $\mathbf{65.6985(5)}$ & $-0.0024$ \\
      \bottomrule
    \end{tabular}
  \end{threeparttable}
\end{table}

\begin{figure}[htbp]
  \centering
  \includegraphics[width=0.8\textwidth]{rel_error_vs_omega_allN}
  \caption{Relative deviation of PINN+BF ground-state energies from DMC
  references as a function of trap frequency $\omega$ for the cases where
  DMC data are available. For $N{=}2$ the deviations are at or below the DMC
  error bars ($\sim 10^{-3}\,\%$), while for $N{=}6$ and $N{=}12$ the errors
  remain in the range $2.5\times 10^{-2}$–$5\times 10^{-2}\,\%$ across
  $\omega$, with no sign of deterioration in the shallow- or tight-trap
  limits.}
  \label{fig:rel_error_vs_omega}
\end{figure}

\paragraph{Summary.}
A compact Slater--Jastrow PINN with soft-core pair features and an explicit
analytic cusp achieves DMC-level accuracy for $N{=}2$ across all $\omega$, and for
$N\!\in\!\{6,12\}$ reaches small, systematic relative deviations (roughly
$0.01$–$0.08\%$ across the grid; Fig.~\ref{fig:rel_error_vs_omega}). In terms
of sample efficiency, the SR tail uses on the order of $10^3$–$3\times10^3$
iterations with $\sim 3{\times}10^3$ samples each; this is \emph{orders of
magnitude fewer} training samples than some earlier neural approaches to
quantum dots (see \S\ref{sec:related} for a quantitative comparison).
We attribute the stability to conditioning choices (trap-unit scaling,
soft-core features with $ds/dr\!\to\!0$ at coalescence, and an explicit cusp),
which keep both gradients and Laplacians numerically tame from shallow to tight
traps. Convergence traces and ablations (activation, initialization, SR
settings, backflow on/off) are reported in Appendix~X.


\section{What the networks learn}
\label{sec:repr-analysis}

We probe the internal representations learned by (i) the residual \emph{correlator head}
$f_{\text{net}}$, fed by the concatenated branches $\overline{\phi}\,\|\,\overline{\psi}\,\|\,\mathbf g$,
and (ii) the \emph{BackflowNet} displacement field $\Delta x$.
All diagnostics are evaluated on Monte Carlo samples drawn from $|\Psi|^2$ for each $(N,\omega)$.
Our goal is not to improve energies further, but to clarify \emph{what information the networks compress} and
\emph{how} that compression changes from tight quantum dots to the Wigner-side regime.

\paragraph{Diagnostics (common across all systems).}
We summarize representation structure using:
(i) an entropy \emph{effective rank} $r_{\mathrm{eff}}$ as a variance-weighted dimensionality;
(ii) principal-component (PC) \emph{ablation} curves, where we project onto the top-$k$ PCs and re-evaluate the
network to quantify how many directions matter;
(iii) \emph{PC1 block power}, which decomposes the leading axis across branches
$\{\phi,\psi,\text{extras}\}$ and reveals which part of the architecture supplies the dominant direction;
(iv) simple linear \emph{probes} from leading PCs to coarse physical summaries (cloud size and fluctuations);
and (v) \emph{near-field conditioning}, where we focus on the lowest-$q\%$ of the minimum pair distance
$r_{\min}$ to test whether energetic leverage is localized at short distances.

\paragraph{Executive summary.}
Across all $(N,\omega)$ studied, the correlator manifold is strongly compressed: a small number of latent
directions dominate the head output and many diagnostics are nearly saturated by PC1.
The key regime dependence is not a growth of dimensionality, but a \emph{recomposition} of the leading axis
across architectural branches.
Backflow, when active, tends to be a \emph{local} correction: it can be geometrically complex as a field
($\Delta x$ may be high-rank and poorly reconstructible by a few PCs), yet it typically alters the correlator
only along very few directions.
Energetically, backflow is most effective in the tighter, more quantum regimes, while in Wigner-side regimes
its gains are smaller and may concentrate on rare near-field events.

\paragraph{Practical note on reliability.}
Two sanity checks are important when interpreting the automated summaries.
First, entries with $\mathrm{SE}=0$ or near-field shares reported as exactly $0$ indicate missing/invalid
statistics rather than genuine physical zeros, and should be treated as ``not available''.
Second, if a displacement ``rank'' proxy and a PC-ablation error appear mutually inconsistent, this typically
means the two quantities were computed from different objects (e.g.\ a covariance-based spectrum versus a
field-based reconstruction) or different normalizations. The qualitative conclusions below rely on patterns
that are stable across multiple metrics (alignment, probeability, near-field shares), rather than on any
single number.

\subsection{Correlator geometry: a compact manifold across regimes}
\label{sec:repr-geom}

Table~\ref{tab:z-geom-tight} summarizes correlator geometry in tight-to-moderate traps
$\omega\in\{1.0,0.5,0.1\}$, while Table~\ref{tab:z-geom-wigner} covers the Wigner-side regimes
$\omega\in\{10^{-2},10^{-3}\}$.
For each system we report the effective rank $r_{\mathrm{eff}}(Z)$, alignment between the head output and its
PC1-only reconstruction, and the PC1 block power written compactly as $\phi/\psi/\mathrm{extras}$.

%%table_z_geometry_tight.tex
\begin{table}[t]
\centering
\small
\setlength{\tabcolsep}{4.5pt}
\renewcommand{\arraystretch}{1.05}
\begin{tabular}{r r r r l}
\toprule
$N$ & $\omega$ & $r_{\mathrm{eff}}(Z)$ & $\rho(\text{head},\text{PC1})$ & \makecell{PC1 power\\$\phi/\psi/\mathrm{extras}$} \\
\midrule
\multirow{2}{*}{2} & \num{0.1} & \num{1.1934303045272827} & \num{0.9971688389778137} & \powtrip{0.002204949841757148}{0.6575334489541872}{0.3402616012030557} \\
 & \num{1.0} & \num{1.2718749374925924} & \num{0.9985596627041398} & \powtrip{2.134598472846666e-05}{0.32727057420040756}{0.672708079814864} \\
\midrule
\multirow{3}{*}{6} & \num{0.1} & \num{1.7749711801443553} & \num{0.938723761477276} & \powtrip{0.7019797068931405}{0.21466329043865442}{0.08335700266820491} \\
 & \num{0.5} & \num{2.985547142534079} & \num{0.9690812306902238} & \powtrip{0.15468924986386895}{0.6501196167747401}{0.19519113336139093} \\
 & \num{1.0} & \num{1.2703213691711426} & \num{0.9978663325309753} & \powtrip{0.0903717502951622}{0.7180420756340027}{0.19158616662025452} \\
\midrule
\multirow{3}{*}{12} & \num{0.1} & \num{2.41208553314209} & \num{0.9893168210983276} & \powtrip{0.03596583753824234}{0.7978293895721436}{0.1662048101425171} \\
 & \num{0.5} & \num{1.912732481956482} & \num{0.9897698760032654} & \powtrip{0.09841396659612656}{0.7751478552818298}{0.1264382004737854} \\
 & \num{1.0} & \num{1.8840608596801758} & \num{0.9924935102462769} & \powtrip{0.08450708538293839}{0.7812861204147339}{0.13420678675174713} \\
\bottomrule
\end{tabular}
\caption{\textbf{Correlator geometry in tight-to-moderate traps ($\omega\in\{1.0,0.5,0.1\}$).} Effective rank, head–PC1 alignment, and PC1 block power (shown as $\phi/\psi/\mathrm{extras}$).}
\label{tab:z-geom-tight}
\end{table}

\input{../results/tables/new/summary_tables/summary_tables.tex}

%%table_z_geometry_wigner.tex
\begin{table}[t]
\centering
\small
\setlength{\tabcolsep}{4.5pt}
\renewcommand{\arraystretch}{1.05}
\begin{tabular}{r r r r l}
\toprule
$N$ & $\omega$ & $r_{\mathrm{eff}}(Z)$ & $\rho(\text{head},\text{PC1})$ & \makecell{PC1 power\\$\phi/\psi/\mathrm{extras}$} \\
\midrule
\multirow{2}{*}{2} & $10^{-3}$ & \num{1.000109076499939} & \num{0.9975659847259521} & \powtrip{2.2638216705672676e-06}{0.9999754428863525}{2.231966755061876e-05} \\
 & $10^{-2}$ & \num{1.002153754234314} & \num{0.9981261491775513} & \powtrip{3.2292762625729665e-05}{0.9993026852607727}{0.0006649706047028303} \\
\midrule
\multirow{2}{*}{6} & $10^{-3}$ & \num{1.5344276312355336} & \num{0.9677401841909734} & \powtrip{0.008845935132918757}{0.1256635664942807}{0.8654904983728006} \\
 & $10^{-2}$ & \num{1.3809024530465976} & \num{0.987120881483588} & \powtrip{0.048846693764802315}{0.4481492961738133}{0.5030040100613844} \\
\midrule
\multirow{2}{*}{12} & $10^{-3}$ & \num{1.5486689304547137} & \num{0.9671884971206108} & \powtrip{0.5220731706087531}{0.12423242588450438}{0.3536944035067426} \\
 & $10^{-2}$ & \num{1.8545435290470527} & \num{0.9559132325261951} & \powtrip{0.5305460824147924}{0.0600770038073888}{0.4093769137778188} \\
\bottomrule
\end{tabular}
\caption{\textbf{Correlator geometry in Wigner-side regimes ($\omega\in\{10^{-2},10^{-3}\}$).} Same metrics as Table~\ref{tab:z-geom-tight}.}
\label{tab:z-geom-wigner}
\end{table}


\paragraph{Interpretation: compression is stable, content reorganizes.}
Two stable patterns appear across the full grid.
First, the correlator head is \emph{nearly one-dimensional} in most regimes:
$r_{\mathrm{eff}}(Z)$ remains close to unity and $\rho(\text{head},\text{PC1})$ is consistently high.
This remains true even on the Wigner side, where the real-space density exhibits pronounced shell structure.
Second, the \emph{content} of PC1 reorganizes with $(N,\omega)$:
in some regimes the leading axis is pair-dominated ($\psi$-heavy), while in others it shifts towards
global/extras-like structure.
Crucially, the network expresses regime changes primarily as a \emph{rotation and redistribution} of a low-dimensional
latent axis across branches, not as an explosion of latent dimensionality.

A helpful way to read the block power is as an ``architectural attribution'' of the dominant latent direction.
For example, in the two-electron dot the Wigner-side PC1 is almost purely pair-like ($\psi\simeq 1$),
consistent with the idea that a single collective pair coordinate largely controls the energy and density.
For larger systems, PC1 can become more global/extras-heavy in the deepest Wigner regimes, suggesting that once
crystalline order emerges the network summarizes the configuration primarily through a single global coordinate
(e.g.\ an overall size/shape mode) rather than through many distinct pair channels.

PC ablations (not shown in tables) are consistent with this picture: across $(N,\omega)$ only a small handful of PCs
are required to recover the head output to high accuracy, with most of the error reduction occurring by $k\sim 4$--$8$.
Thus, even when real-space structure becomes complex, the correlator head remains organized around a compact set of
latent directions.

\subsection{What the correlator PCs represent: size and fluctuations}
\label{sec:repr-probes}

To connect latent axes to physics, we fit linear probes from the leading PCs to coarse observables:
the mean radius $r_{\mathrm{mean}}$, radial variance $r_{\mathrm{var}}$, and a short-distance mass
$\Pr(r<r_0)$ (here using the same $r_0$ as in the JSON probes).
The results are summarized in Table~\ref{tab:probe-summary}, comparing probes on correlator features $Z$
(no backflow) to probes on backflow displacement features $\Delta x$ (when available).

%table_probe_summary.tex
\begin{table}[t]
\centering
\small
\setlength{\tabcolsep}{3.2pt}
\renewcommand{\arraystretch}{1.05}
\begin{tabular}{r r r r r r r r}
\toprule
$N$ & $\omega$ & $R^2_Z(r_{\mathrm{mean}})$ & $R^2_Z(r_{\mathrm{var}})$ & $R^2_Z(\Pr(r<0.25))$ & $R^2_{\Delta x}(r_{\mathrm{mean}})$ & $R^2_{\Delta x}(r_{\mathrm{var}})$ & $R^2_{\Delta x}(\Pr(r<0.25))$ \\
\midrule
\multirow{4}{*}{2} & $10^{-3}$ & \num{0.9998593330383301} & \num{1.0} & \num{1.0} & -- & -- & -- \\
 & $10^{-2}$ & \num{0.9991859793663025} & \num{1.0} & \num{1.0} & \num{0.8365819454193115} & \num{1.0} & \num{1.0} \\
 & \num{0.1} & \num{0.9906408786773682} & \num{1.0} & \num{0.01065605878829956} & -- & -- & -- \\
 & \num{1.0} & \num{0.9916961908340606} & \num{1.0} & \num{0.3292320702747056} & \num{0.052064739744731026} & \num{1.0} & \num{0.003139786391593624} \\
\midrule
\multirow{5}{*}{6} & $10^{-3}$ & \num{0.9884054819597593} & \num{0.6458078318616691} & \num{0.0013352071077133987} & \num{5.551115123125783e-16} & \num{4.440892098500626e-16} & \num{2.2502788521450157e-10} \\
 & $10^{-2}$ & \num{0.9932156895671214} & \num{0.7608386125657696} & \num{0.0014099672715170986} & \num{0.0008956570838948297} & \num{0.0001564176947208562} & \num{1.0} \\
 & \num{0.1} & \num{0.9771254868408659} & \num{0.4888795335623055} & \num{0.005918540528595195} & \num{0.00011376048435540742} & \num{0.00018405430999313133} & \num{0.0006715698956994975} \\
 & \num{0.5} & \num{0.9780979676437129} & \num{0.40587633020209446} & \num{0.010660148777387746} & \num{6.592805162897175e-05} & \num{0.00020810357693368253} & \num{0.0012030795123103566} \\
 & \num{1.0} & \num{0.9842498302459717} & \num{0.6537197828292847} & \num{0.024350464344024658} & \num{0.0005335807800292969} & \num{0.0004315376281738281} & \num{0.001131892204284668} \\
\midrule
\multirow{5}{*}{12} & $10^{-3}$ & \num{0.9960694884941751} & \num{0.8260597271086828} & \num{0.009063813986401104} & \num{0.01893216864850955} & \num{0.006425540122680484} & \num{0.010785537293282021} \\
 & $10^{-2}$ & \num{0.9781608873593045} & \num{0.8082932096272196} & \num{0.010273319902807776} & \num{0.016296084193267668} & \num{0.010456095055835002} & \num{0.002007440884294831} \\
 & \num{0.1} & \num{0.9672423005104065} & \num{0.6394591331481934} & \num{0.007377207279205322} & \num{0.00028705596923828125} & \num{0.0005279183387756348} & \num{0.00028055906295776367} \\
 & \num{0.5} & \num{0.9817021489143372} & \num{0.6515886783599854} & \num{0.013809919357299805} & \num{0.0004922151565551758} & \num{0.0002778768539428711} & \num{0.00041472911834716797} \\
 & \num{1.0} & \num{0.9914807677268982} & \num{0.6813317537307739} & \num{0.015794456005096436} & \num{0.0006495118141174316} & \num{2.8312206268310547e-05} & \num{0.00027185678482055664} \\
\bottomrule
\end{tabular}
\caption{\textbf{Linear probes of representation content.} $R^2$ of simple probes from leading PCs to coarse observables. We compare correlator features $Z$ (noBF) against backflow displacement features $\Delta x$ (BF).}
\label{tab:probe-summary}
\end{table}


\paragraph{Interpretation: $Z$ primarily tracks global size, not near-field events.}
Across essentially all $(N,\omega)$, the correlator features are highly predictive of global size:
$R^2_Z(r_{\mathrm{mean}})$ is near unity for $N=2,6,12$.
This supports a simple structural interpretation of the compressed manifold:
the leading correlator direction acts as a \emph{global size coordinate} of the electron cloud.
Radial variance $r_{\mathrm{var}}$ is captured moderately well for $N\ge 6$, consistent with variance being a
mixture of size and shape/fluctuation effects that require more than one linear direction.

In contrast, $\Pr(r<r_0)$ is generally \emph{not} linearly encoded in $Z$ for $N\ge 6$:
$R^2_Z(\Pr(r<r_0))$ remains small compared to the size probes.
This indicates that the correlator PCs are not behaving like a simple ``collision detector'' in larger systems.
Instead, near-field physics appears to enter the energy through non-linear couplings and/or through the backflow
mechanism, rather than being explicitly linearized in the main correlator manifold.

\subsection{Backflow: energetics, locality, and stability of the correlator}
\label{sec:repr-backflow}

Backflow augments the base model with a learned displacement field $\Delta x$.
We analyze it from two angles: (i) its \emph{energetic contribution} (does it lower the variational energy?),
and (ii) its \emph{geometric footprint} (does it substantially alter the correlator manifold or only nudge it?).

\paragraph{Energetic impact and near-field leverage.}
Table~\ref{tab:bf-energy} reports $E_{\mathrm{noBF}}$ and the matched per-configuration difference
$\Delta E = E_{\mathrm{BF}}-E_{\mathrm{noBF}}$ together with near-field enrichment factors
for the lowest $5\%$ and $10\%$ of $r_{\min}$ (values $>1$ indicate disproportionate contribution from close pairs).
This separates ``global'' energy improvements from gains driven by rare near-collision configurations.

%table_bf_energy.tex
\begin{table}[t]
\centering
\small
\setlength{\tabcolsep}{4.0pt}
\renewcommand{\arraystretch}{1.05}
\begin{tabular}{r r l l l}
\toprule
$N$ & $\omega$ & $E_{\mathrm{noBF}}$ [Ha] & $\Delta E$ [Ha] & \makecell{Near-field $\Delta E$\\(5\%/10\%)} \\
\midrule
\multirow{4}{*}{2} & $10^{-3}$ & \pmnum{0.013819664716720581}{6.744159236404812e-06} & \pmnum{-1.9832514226436615e-05}{1.0538174421981239e-07} & \nfshare{1.0448053300901492}{1.0297964545108664} \\
 & $10^{-2}$ & \pmnum{0.07408207654953003}{2.046969711955171e-05} & \pmnum{1.4647841453552246e-05}{5.65204845770495e-06} & \nfshare{1.5941240947461006}{1.0557228613883436} \\
 & \num{0.1} & \pmnum{0.4408929944038391}{6.935203418834135e-05} & -- & -- \\
 & \num{1.0} & \pmnum{3.000246484668727}{0.00044654565987045387} & \pmnum{-0.0003872328504095357}{0.00033389850052462376} & \nfshare{0.5032530689905623}{0.6793692749704435} \\
\midrule
\multirow{5}{*}{6} & $10^{-3}$ & \pmnum{0.14182264681305012}{3.865184003651171e-05} & \pmnum{-2.9684157596104876e-05}{0.0} & \nfshare{0.0}{0.0} \\
 & $10^{-2}$ & \pmnum{0.6918454563459062}{9.08638855812678e-05} & \pmnum{-0.00047727080424553314}{6.870060671147044e-05} & \nfshare{1.1887710516793284}{1.0415067762060448} \\
 & \num{0.1} & \pmnum{3.583229758984681}{0.0014627165278403322} & \pmnum{-0.028114113252731165}{0.0017386524831997552} & \nfshare{2.7287313403413225}{2.2224907944652266} \\
 & \num{0.5} & \pmnum{11.929874290886579}{0.007483812844352975} & \pmnum{-0.13768824737427288}{0.00831598301499328} & \nfshare{2.1060377937945396}{1.862088706412662} \\
 & \num{1.0} & \pmnum{20.30365753173828}{0.00977847584765969} & \pmnum{-0.1335735321044922}{0.031119878930869458} & \nfshare{1.5113542321561184}{1.4490586901369693} \\
\midrule
\multirow{5}{*}{12} & $10^{-3}$ & \pmnum{0.521015720897955}{0.0010929860001209718} & \pmnum{-0.0010159463938169688}{1.0174702396110345e-06} & \nfshare{1.038740645293576}{1.0370653433928148} \\
 & $10^{-2}$ & \pmnum{2.5002791820659285}{0.0008639834703051807} & \pmnum{0.001343807396593899}{0.004348531283829157} & \nfshare{5.942922091199255}{4.70389029721622} \\
 & \num{0.1} & \pmnum{12.592391967773438}{0.0035093185175391222} & \pmnum{-0.327117919921875}{0.0052498168516704885} & \nfshare{1.7746985308071863}{1.654357267680921} \\
 & \num{0.5} & \pmnum{39.876953125}{0.014080741888508413} & \pmnum{-0.6988677978515625}{0.014867546629661025} & \nfshare{1.2369688939905505}{1.2709383200524431} \\
 & \num{1.0} & \pmnum{66.50838470458984}{0.02005534443436798} & \pmnum{-0.8258056640625}{0.02320765224320683} & \nfshare{1.217424326966035}{1.2120606456625544} \\
\bottomrule
\end{tabular}
\caption{\textbf{Energetic impact of backflow.} Reported on matched samples: $E_{\mathrm{noBF}}$ and $\Delta E = E_{\mathrm{BF}}-E_{\mathrm{noBF}}$. Near-field column gives the share of $\Delta E$ carried by the lowest $5\%/10\%$ of $r_{\min}$ (1.0 = fair share).}
\label{tab:bf-energy}
\end{table}


\paragraph{Interpretation: backflow is strongest in dense regimes and often near-field driven.}
In the tight-to-moderate traps ($\omega\gtrsim 0.1$), backflow yields substantial energy reductions for
$N=6$ and $N=12$ and the near-field enrichment is consistently above unity, indicating that close-pair
configurations contribute disproportionately to the energy gain.
In these denser regimes, backflow behaves like a short-range corrector that refines local structure (and therefore
local kinetic/interaction balance) beyond what the base correlator captures.

On the Wigner side ($\omega\in\{10^{-2},10^{-3}\}$), the picture changes.
Energy gains become smaller and, when present, may be dominated by rare near-field events rather than by broad
improvements across typical configurations. Large near-field enrichment factors alongside noisy or small mean
$\Delta E$ are consistent with ``rare-event leverage'': the displacement field matters primarily when atypically
close pairs occur, which may be infrequent in strongly ordered Wigner configurations.

\paragraph{Displacement complexity and correlator stability.}
Table~\ref{tab:bf-geom} summarizes: an effective-rank proxy for $\Delta x$, a compact PC-ablation score for
$\Delta x$ (top-2 PC reconstruction error), the PC1 alignment between correlator axes (noBF vs BF),
an effective rank for the correlator change $\Delta Z$, and near-field concentration of $\|\Delta x\|^2$.
Together these metrics separate \emph{high-rank local fields} from \emph{low-rank manifold shifts}.

%table_bf_geometry.tex
\begin{table}[t]
\centering
\small
\setlength{\tabcolsep}{3.8pt}
\renewcommand{\arraystretch}{1.05}
\begin{tabular}{r r r r r r r}
\toprule
$N$ & $\omega$ & $r_{\mathrm{eff}}(\Delta x)$ & \makecell{PC ablation\\$\Delta x$ (k=2)} & PC1 cosine & $r_{\mathrm{eff}}(\Delta Z)$ & NF $\|\Delta x\|^2$ (5\%) \\
\midrule
\multirow{4}{*}{2} & $10^{-3}$ & \num{3.6238982677459717} & \num{0.0} & \num{1.0} & \num{1.0001094341278076} & \num{1.0} \\
 & $10^{-2}$ & \num{2.4740917682647705} & \num{0.2346760779619217} & \num{1.0} & \num{1.0021824836730957} & \num{0.30695810387990213} \\
 & \num{0.1} & -- & -- & -- & -- & -- \\
 & \num{1.0} & \num{2.347846071844718} & \num{0.1952260583639145} & \num{0.9999994751369439} & \num{1.267816172534333} & \num{0.0667831407221286} \\
\midrule
\multirow{5}{*}{6} & $10^{-3}$ & \num{10.672235550292687} & \num{0.03963865339756012} & \num{0.9999943084937837} & \num{1.548339201871405} & \num{1.0000000000000002} \\
 & $10^{-2}$ & \num{9.759326663471045} & \num{0.8294326066970825} & \num{0.9999957944376336} & \num{1.3780863189288863} & \num{0.9945388289563737} \\
 & \num{0.1} & \num{10.420986759194957} & \num{0.8720242977142334} & \num{0.9996565860781425} & \num{1.7467941866864882} & \num{1.4087701019226342} \\
 & \num{0.5} & \num{10.631481166770067} & \num{0.8890220522880554} & \num{0.998472041559447} & \num{3.0006171224308678} & \num{1.1271533087701864} \\
 & \num{1.0} & \num{10.88346004486084} & \num{0.8824098706245422} & \num{0.9993336200714111} & \num{1.2690986394882202} & \num{0.9875355526318716} \\
\midrule
\multirow{5}{*}{12} & $10^{-3}$ & \num{21.97973364005276} & \num{8.232251502704457e-07} & \num{0.9999710657130808} & \num{1.5509512579015885} & \num{0.9999999998530947} \\
 & $10^{-2}$ & \num{12.070210266987656} & \num{0.8761083483695984} & \num{0.9999886308488887} & \num{1.8529525123192496} & \num{0.9997456618023859} \\
 & \num{0.1} & \num{22.786664962768555} & \num{0.9484862685203552} & \num{-0.9889825582504272} & \num{2.5750527381896973} & \num{1.010467421209829} \\
 & \num{0.5} & \num{23.842897415161133} & \num{0.9518084526062012} & \num{0.9990471005439758} & \num{1.939249873161316} & \num{0.9875718860988458} \\
 & \num{1.0} & \num{23.429546356201172} & \num{0.9489565491676331} & \num{0.9995853900909424} & \num{1.8922200202941895} & \num{1.0296314447112258} \\
\bottomrule
\end{tabular}
\caption{\textbf{Backflow geometry and correlator stability.} We report an effective-rank proxy for $\Delta x$, displacement PC-ablation error at $k{=}2$, PC1 alignment between correlator axes (noBF vs BF), an effective rank for $\Delta Z$, and near-field concentration of $\|\Delta x\|^2$ at $5\%$ of $r_{\min}$.}
\label{tab:bf-geom}
\end{table}


\paragraph{Interpretation: a complex field with a low-rank footprint.}
For $N\ge 6$, the displacement field typically appears geometrically rich: $\Delta x$ has a high effective-rank
proxy and is not well reconstructed by only a couple of PCs (large $k{=}2$ ablation errors in several regimes).
Yet, the correlator manifold remains remarkably stable under backflow:
the PC1 alignment between noBF and BF correlator axes is almost always extremely close to unity (up to the
irrelevant sign ambiguity of principal directions), and $r_{\mathrm{eff}}(\Delta Z)$ remains small.
This is a strong separation-of-concerns statement: backflow can be a complicated local field without rewriting
the correlator representation. Instead, it typically nudges the correlator along a few directions while injecting
local, configuration-dependent corrections through $\Delta x$ itself.

Near-field concentration of $\|\Delta x\|^2$ varies across regimes.
Values near unity indicate that displacement magnitude is broadly distributed across configurations,
while values significantly above unity indicate stronger displacements in near-collision events.
Importantly, near-field concentration of displacement energy does not need to match near-field concentration of
$\Delta E$: a field can be distributed in magnitude yet have its \emph{energetic} leverage concentrated in the
near field, because energy sensitivity to $\Delta x$ is itself configuration-dependent.

\subsection{Synthesis: how the architecture partitions the physics}
\label{sec:repr-synthesis}

Taken together, Tables~\ref{tab:z-geom-tight}--\ref{tab:bf-geom} support a coherent representation picture:
the correlator head $Z$ forms a compact, globally organized manifold whose leading directions track coarse
cloud-level coordinates (size and fluctuations), while backflow—when it matters—acts as a local corrector that
targets configuration-dependent refinements, often with disproportionate leverage in near-field events.
Even when the displacement field is high-rank, its effect on the correlator manifold is typically low-rank,
appearing as a small deformation rather than a wholesale change of representation.

From a modeling perspective, this partition is desirable.
It suggests that the base correlator learns the dominant collective coordinates robustly across regimes, while
backflow provides an optional refinement channel that improves energy primarily when local corrections are needed
(tight/dense traps) and becomes less important when the system is already well described by a compact set of
collective modes (deep Wigner-side ordering).


\section{Wigner–molecule crossover}
\label{sec:wigner-molecule}


\begin{figure*}[htbp]
  \centering
  \captionsetup[subfigure]{justification=centering,font=small}

  % ---------- Row 1: ω = 1.0 ----------
  \begin{subfigure}[t]{\threecolw}
    \centering
    \includegraphics[width=\linewidth]{one_body_density_2_omega_1.00000_20251029_231309.pdf}
    \caption{$N{=}2,\;\omega=1.0$}
  \end{subfigure}\hspace{\figgutter}%
  \begin{subfigure}[t]{\threecolw}
    \centering
    \includegraphics[width=\linewidth]{one_body_density_6_omega_1.00000_20251030_093439.pdf}
    \caption{$N{=}6,\;\omega=1.0$}
  \end{subfigure}\hspace{\figgutter}%
  \begin{subfigure}[t]{\threecolw}
    \centering
    \includegraphics[width=\linewidth]{one_body_density_12_omega_1.00000_20251104_194000.pdf}
    \caption{$N{=}12,\;\omega=1.0$}
  \end{subfigure}

  \vspace{0.5em}

  % ---------- Row 2: ω = 0.001 ----------
  \begin{subfigure}[t]{\threecolw}
    \centering
    \includegraphics[width=\linewidth]{one_body_density_2_omega_0.00100_20251106_160757.pdf}
    \caption{$N{=}2,\;\omega=0.001$}
  \end{subfigure}\hspace{\figgutter}%
  \begin{subfigure}[t]{\threecolw}
    \centering
    \includegraphics[width=\linewidth]{one_body_density_6_omega_0.00100_20251106_214930.pdf}
    \caption{$N{=}6,\;\omega=0.001$}
  \end{subfigure}\hspace{\figgutter}%
  \begin{subfigure}[t]{\threecolw}
    \centering
    \includegraphics[width=\linewidth]{one_body_density_12_omega_0.00100_20251109_110913.pdf}
    \caption{$N{=}12,\;\omega=0.001$}
  \end{subfigure}

  \caption{One-body densities for 2, 6 and 12 particles for confinements
  $\omega=1.0$ (top row) and $\omega=0.001$ (bottom row). As the confinement
  weakens, the densities broaden and develop pronounced shell structures
  characteristic of Wigner molecules.}
  \label{fig:onebody_grid}
\end{figure*}


\subsection{What we measure}
For each $(N,\omega)$ we draw $\mathbf X\!\sim|\Psi|^2$ and report
(i) the pair distribution $g(r)$,
(ii) the single-particle radial law $P(r)\!\propto\!r^{d-1}g(r)$ ($d=2$),
and scalars from $P(r)$: $r_{\rm mode}$, $\langle r\rangle$, $\sigma_r$, FWHM, $q_{10},q_{50},q_{90}$.
We condense localization via
\[
\gamma \;=\; \sigma_r/r_{\rm mode},
\]
(smaller $\gamma$ $\Rightarrow$ stiffer localization).
For $N>2$ we also monitor ring bond–orientational order $|\Phi_m|$ on resolved shells 
and a dimensionless Lindemann-type ratio for nearest-neighbour angular spacings 
(see also bond-orientational diagnostics in classical Wigner-like systems~\cite{Mazars_2008}).

\paragraph{Density parameter $r_s$.}
Following Egger \textit{et al.}~\cite{Egger_1999}, we estimate $r_s$ from the first maximum $r^\ast$ of the (spin-summed) pair correlation,
\(
r_s \equiv r^\ast/a_B^\ast.
\)
The Fermi–liquid $\to$ Wigner–molecule crossover occurs already near $r_s\!\simeq\!4$ in parabolic dots; 
our weak–trap cases lie well beyond this threshold and should display clear Wigner signatures~\cite{Egger_1999,Filinov_2001}.

\begin{figure*}[htbp]
  \centering
  \captionsetup[subfigure]{justification=centering}

  \begin{subfigure}[t]{0.48\textwidth}
    \includegraphics[width=\linewidth]{ratio_T_over_Vint_desc_allN}
    \caption{$T/V_{\rm int}$ vs.\ $\omega$ for $N{=}2,6,12$. The horizontal line marks $T/V_{\rm int}{=}0.1$.}
  \end{subfigure}
  \hfill
  \begin{subfigure}[t]{0.48\textwidth}
    \includegraphics[width=\linewidth]{ratio_2Vtrap_over_Vint_desc_allN}
    \caption{$2V_{\rm trap}/V_{\rm int}$ vs.\ $\omega$ for $N{=}2,6,12$. The horizontal line marks the classical virial ratio $2V_{\rm trap}=V_{\rm int}$.}
  \end{subfigure}

  \caption{Energy-based Wigner diagnostics as functions of trap frequency.  
  As $\omega$ decreases the dots become interaction dominated ($T/V_{\rm int}\!\ll1$) and the trap--interaction partition approaches the classical virial form, most clearly for $N{=}12$.}
  \label{fig:wigner_ratios}
\end{figure*}

\paragraph{Energy-based crossover diagnostics.}
As a complementary probe we analyze the expectation values of the kinetic energy $T$,
the Coulomb energy $V_{\rm int}$, and the trap energy $V_{\rm trap}$ as functions of $\omega$.
Log–log fits of the form $E(\omega)\propto\omega^{\alpha}$ on the grid
$\omega\in\{10^{-3},10^{-2},10^{-1},0.5,1\}$ give
\[
\alpha_T \simeq 1.00,\qquad
\alpha_{V_{\rm int}} \simeq 0.66,\qquad
\alpha_{V_{\rm trap}} \simeq 0.81,\,0.73,\,0.72
\]
for $N=2,6,12$, respectively. Thus $T$ scales essentially linearly with $\omega$, while both
$V_{\rm int}$ and $V_{\rm trap}$ grow sublinearly with exponents that are nearly $N$-independent.
The ratios
\[
\frac{T}{V_{\rm int}} \;\propto\; \omega^{\alpha_T-\alpha_{V_{\rm int}}}
\quad \text{and} \quad
\frac{T}{V_{\rm trap}} \;\propto\; \omega^{\alpha_T-\alpha_{V_{\rm trap}}}
\]
therefore decrease as power laws with exponents $\alpha_T-\alpha_{V_{\rm int}}\!\approx\!0.32\text{--}0.35$
and $\alpha_T-\alpha_{V_{\rm trap}}\!\approx\!0.19\text{--}0.28$.
Across $\omega\in[1,10^{-3}]$ the interaction-to-kinetic ratio
$\Gamma= \langle V_{\rm int}\rangle/\langle T\rangle$ increases from
$\Gamma\simeq0.9\!\to\!8.7$ ($N=2$), $2.4\!\to\!25.4$ ($N=6$), and $3.6\!\to\!39.9$ ($N=12$),
while $T/V_{\rm int}$ falls correspondingly (Fig.~\ref{fig:wigner_ratios}a), e.g.\
$T/V_{\rm int}$ decreases from $\approx 0.28$ to $\approx 0.025$ for $N=12$.

The partitioning between trap and interaction energies becomes quasi-classical at weak confinement.
At $\omega=10^{-3}$ we find
\[
\frac{2\langle V_{\rm trap}\rangle}{\langle V_{\rm int}\rangle}
\simeq 1.25,\;1.12,\;1.02 \quad\text{for}\quad N=2,6,12,
\]
so that the $N=12$ dot satisfies $V_{\rm int}\!\approx\!2V_{\rm trap}$ to within $2\%$,
as expected for nearly classical charges in a harmonic trap (Fig.~\ref{fig:wigner_ratios}b).
Using $V_{\rm trap} = \tfrac12 \omega^2 \sum_i \langle r_i^2\rangle$ we can extract
a root-mean-square radius
$r_{\rm rms} = \sqrt{2V_{\rm trap}/(N\omega^2)}$.
For $(N,\omega)=(2,6,12;1)$ we obtain
$r_{\rm rms}\!\approx\!1.14,1.62,2.03\,a_B^\ast$, which expands to
$\approx 70,128,169\,a_B^\ast$ at $\omega=10^{-3}$.
Fitting $r_{\rm rms}\propto\omega^{\alpha_r}$ yields
$\alpha_r\simeq-0.60,-0.63,-0.64$ for $N=2,6,12$,
i.e.\ $\langle r^2\rangle\propto\omega^{-\beta}$ with
$\beta\simeq1.19,1.27,1.28$.
The cloud thus expands \emph{faster} than the non-interacting
$\omega^{-1}$ expectation, in line with interaction-driven swelling and
shell formation in the Wigner regime.

% ------------------------- N=2 -------------------------
\subsection{Two electrons ($N=2$): monotone approach to the Wigner side}
As $\omega$ decreases, $r_{\rm mode}$ grows and $\gamma=\sigma_r/r_{\rm mode}$ shrinks
(Table~\ref{tab:two_e_summary}),
with a power-law $r_{\rm mode}\!\sim C\,\omega^{\alpha}$ giving $\alpha\simeq-0.62$ over
$\omega\in\{1,0.5,0.1,0.01\}$, close to the classical two-body scaling $-2/3$.  

At the weakest trap, $\omega=10^{-3}$, the radial diagnostics read
$r_{\rm mode}\approx64.3$, $\langle r\rangle\approx64.0$,
$\sigma_r\approx18.5$, FWHM$\approx44.0$ and
$\gamma\approx0.29$, with only
$\sim 4.6\times 10^{-3}$ of the single-particle mass inside
$r\le 0.25\,r_{\rm mode}$.
The pair distance distribution has mode $r_{12}\approx122$ and
Lindemann $\gamma_{r_{12}}\approx0.20$, while the relative angle
is sharply peaked near $\pi$ with
$\sigma(\pi-\Delta\phi)\approx0.35$ and a tiny near-origin mass
$p(r_{12}\le0.25\,r_{12}^{\rm mode})\approx2.6\times 10^{-3}$.
Together these show a very dilute, strongly anti-aligned Wigner dimer.

\begin{figure}[htp]
  \centering
  \includegraphics[width=0.8\textwidth]{N2_all_densities}
  \caption{Two-electron radial densities $P(r)$ across $\omega$.}
  \label{fig:2N_radial_densities}
\end{figure}

\begin{table}[H]
\centering
\caption{$N=2$ summary across $\omega$ (Bohr).}
\label{tab:two_e_summary}
\begin{tabular}{lcccccccc}
\toprule
$\omega$ 
& $r_{\rm mode}$ 
& $\langle r\rangle$ 
& $\sigma_r$ 
& FWHM 
& $q_{10}$ 
& $q_{50}$ 
& $q_{90}$ 
& $\gamma$ \\
\midrule
1.00 & 1.435 & 1.632 & 0.705 & 1.743 & 0.735 & 1.581 & 2.581 & 0.491 \\
0.50 & 2.303 & 2.475 & 1.007 & 2.466 & 1.209 & 2.409 & 3.829 & 0.437 \\
0.10 & 3.601 & 3.830 & 1.728 & 4.449 & 1.640 & 3.714 & 6.127 & 0.480 \\
0.01 & 14.191 & 14.593 & 5.615 & 13.602 & 7.390 & 14.322 & 22.038 & 0.396 \\
0.001 & 64.292 & 63.953 & 18.479 & 43.977 & 40.392 & 63.814 & 87.714 & 0.287 \\
\bottomrule
\end{tabular}
\end{table}

\paragraph{Takeaway.}
All observables—growth of $r_{\rm mode}$, shrinking $\gamma$, sharper anti-alignment,
and suppressed near-origin mass—evolve smoothly with decreasing $\omega$,
consistent with the expected two-site Wigner dimer limit.

% ------------------------- N=6 -------------------------
\subsection{Six electrons ($N=6$): one ring persists, correlations stiffen}
Across $\omega\in[1,10^{-3}]$, $N=6$ retains a single pronounced shell; 
localization ratios and Lindemann indices decrease steadily (Table~\ref{tab:N6}).  
This is precisely the ``rotating Wigner molecule'' (RWM) scenario—lab-frame densities are ring-like, 
while crystalline order is recovered after angular registration~\cite{Egger_1999,manninen2007metalclustersquantumdots}.

\begin{table}[H]
\centering
\caption{$N=6$ diagnostics vs.\ $\omega$ (Bohr; angles in radians).}
\label{tab:N6}
\begin{tabular}{c|ccccc}
\toprule
$\omega$ & $r_{\rm mode}$ & $\sigma_r$ & $\gamma$ & $\sigma(\text{ring angles})$ & $\gamma_{r_{ij}}$ \\
\midrule
1.00  & 2.0907 & 0.9249  & 0.4424 & 0.7046 & 0.3538 \\
0.50  & 3.0702 & 1.3648  & 0.4445 & 0.6943 & 0.3397 \\
0.10  & 8.1686 & 3.5025  & 0.4288 & 0.6842 & 0.3149 \\
0.01  & 33.1538 & 13.9094 & 0.4195 & 0.6497 & 0.2889 \\
0.001 & 113.5542 & 47.8076 & 0.4210 & 0.6368 & 0.2664 \\
\bottomrule
\end{tabular}
\end{table}

\paragraph{Weak confinement at $\omega=0.01$.}
A near-degeneracy between the $(1,5)$ and $(0,6)$ sectors is observed, with
\[
(1,5)\ \text{fraction}\approx 0.692,\qquad (0,6)\approx 0.308,
\]
and phase–aligned ring order
\[
|\Phi_5|=0.453\!\pm\!0.214,\ \ \text{Lind}=0.329;\qquad
|\Phi_6|=0.466\!\pm\!0.206,\ \ \text{Lind}=0.315.
\]
The global $g(r)$ is exactly reproduced by the expected inner–inner (II),
inner–outer (IO) and outer–outer (OO) mixture (cosine similarity~$=1.000$).

\begin{figure}[htp]
  \centering
  \includegraphics[width=0.8\textwidth]{N6_all_densities}
  \caption{Six-electron radial densities $P(r)$ across $\omega$.}
  \label{fig:6N_radial_densities}
\end{figure}

\paragraph{Pushing to $\omega=10^{-3}$.}
The ring stiffens further and both symmetry classes persist with high-quality order.
From $K=300{,}000$ frames we obtain
\[
(1,5)\ \text{fraction}\approx 0.801,\qquad (0,6)\approx 0.199,
\]
with phase-aligned ring order
\[
|\Phi_5| = 0.597\!\pm\!0.224,\ \ \text{Lind}\approx0.225;\qquad
|\Phi_6| = 0.663\!\pm\!0.183,\ \ \text{Lind}\approx0.205.
\]
Thus the shell is always present; lab-frame symmetry is restored by rotation,
but in the co-rotating frame the six electrons form a stiff Wigner ring,
predominantly in the $(1,5)$ sector with a significant $(0,6)$ minority.
Classically, $(1,5)$ is favored and $(0,6)$ a proximate competitor; the quantum
mixture at large $r_s$ mirrors this~\cite{schweigert1994spectralpropertieschargedparticles,Kong_2002}.
(In both weak-trap cases the global $g(r)$ is reproduced to numerical precision
by the appropriate II/IO/OO mixtures.)

% ------------------------- N=12 -------------------------

\subsection{Twelve electrons ($N=12$): two shells and commensurability at \boldmath{$\omega=10^{-3}$}}

Already at $\omega=0.01$ the system is well on the Wigner side, with two-shell formation frequent.  
Using a conservative radial-gap threshold $\tau{=}3.0$ we find a two-shell fraction of $0.761$, with inner–ring occupancies
\[
(1,11)=32.639\%,\quad (2,10)=25.688\%,\quad (3,9)=13.716\%,
\]
and bond order / angular fluctuations
\[
\begin{aligned}
&(1,11):\quad |\Phi_{\rm out}|=0.276,\ \ \text{Lind}_{\rm out}=0.495,\\
&(2,10):\quad |\Phi_{\rm in}|=0.661,\ \ \text{Lind}_{\rm in}=0.315;\ \ |\Phi_{\rm out}|=0.300,\ \ \text{Lind}_{\rm out}=0.432,\\
&(3,9):\quad |\Phi_{\rm in}|=0.565,\ \ \text{Lind}_{\rm in}=0.342;\ \ |\Phi_{\rm out}|=0.322,\ \ \text{Lind}_{\rm out}=0.408.
\end{aligned}
\]

Upon lowering the trap to $\omega=10^{-3}$, shelling becomes ubiquitous and a \emph{commensurate} split emerges as modal.
For the full $500{,}000$-frame ensemble the two-shell fraction is
\[
f_{\rm 2shell}(\tau) \approx 0.991,\ 0.966,\ 0.923
\quad\text{for}\quad \tau=2.0,2.5,3.0,
\]
with inner–ring occupancy histograms that are very stable under~$\tau$.
At $\tau{=}3.0$ we also consider a stricter two-shell subset, obtained by combining the
radial gap criterion with additional quality cuts (see Appendix~Y); this subset constitutes
$46.15\%$ of all frames and has
\[
(3,9)=43.29\%,\quad (2,10)=19.89\%,\quad (1,11)=12.54\%,
\]
with the remaining weight spread over $(4,8)$, $(5,7)$ and rarer configurations.
The median outer radius in trap units is ${\approx}7.02$, consistent with the
energy-based estimate of a strongly swollen cloud.

On $(3,9)$ frames the rings show strong bond order with small angular Lindemann:
\[
|\Phi_3|=0.687\!\pm\!0.229,\ \ \text{Lind}_{\rm in}=0.223;\qquad
|\Phi_9|=0.526\!\pm\!0.194,\ \ \text{Lind}_{\rm out}=0.224.
\]
Using a reduced bundle with $K\approx2.5\times 10^5$ frames that satisfy the
two-shell criterion, we construct II/IO/OO histograms and weight them by the
combinatorial mixture $(0.045{:}0.409{:}0.545)$.
The resulting synthetic $g(r)$ has cosine similarity $0.9982$ to the global
$g(r)$, i.e.\ is indistinguishable at our statistical resolution
(see Fig.~\ref{fig:N12_w0001_decomposition}).

\begin{figure}[htbp]
  \centering
  \includegraphics[width=0.85\textwidth]{mixfit_N12_w0.001_nin3.pdf}
  \caption{Shell-resolved reconstruction of $g(r)$ at $\omega=10^{-3}$ for $N{=}12$.  
  The global pair distribution is reproduced by a convex mixture of inner--inner (II),
  inner--outer (IO) and outer--outer (OO) histograms with weights set by the observed
  $(n_{\rm in},n_{\rm out})$ statistics, confirming that the structure is geometric
  rather than a sampling artifact.}
  \label{fig:N12_w0001_decomposition}
\end{figure}

This is exactly the Wigner-molecule regime: the \emph{radial} structure freezes
into concentric shells, while residual \emph{azimuthal} dynamics mixes a few
low-lying ring occupancies.
The $(3,9)$ commensurate split at the weakest trap aligns with classical
ground-state analyses for parabolic Coulomb clusters~\cite{schweigert1994spectralpropertieschargedparticles,Kong_2002}; 
the finite weight of $(1,11)$ and $(2,10)$ reflects near-degeneracies and
quantum rotational mixing~\cite{manninen2007metalclustersquantumdots,Filinov_2001}.

\begin{figure}[htp]
  \centering
  \includegraphics[width=0.8\textwidth]{N12_all_densities}
  \caption{Twelve-electron radial densities $P(r)$ across $\omega$.}
  \label{fig:12N_radial_densities}
\end{figure}

\paragraph{Significance.}
Taken together, the $\omega=10^{-3}$ results provide a sharp, quantitative realization of the Wigner-molecule picture in parabolic dots:
(i) shell-resolved reconstruction of $g(r)$ at numerical precision,
(ii) strong, commensurate bond order on rings with small angular Lindemann ratios,
(iii) shell-occupancy (``topology'') fractions that smoothly approach the classical ordering as confinement weakens,
all consistent with~\cite{Egger_1999,Filinov_2001,schweigert1994spectralpropertieschargedparticles,Kong_2002}.
Our sector-resolved $|\Phi_m|$ are extracted \emph{within} a fully variational quantum ansatz and measured conditionally on shell occupancy, 
which (to our knowledge) has not been reported before for $N=12$ parabolic dots.
