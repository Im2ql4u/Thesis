
\section{Results}

\subsection{Interacting two-dimensional quantum dots}

We study two-dimensional harmonic quantum dots, a numerically delicate setting due to the Coulomb singularity at short range and the second-order derivatives from the kinetic energy. Both effects can yield ill-conditioned objectives that typically require many epochs and small learning rates. We consider two-electron dots across a wide confinement range ($\omega\!\in\!\{1.0,\,0.5,\,0.1,\,0.01\}$), six-electron dots at $\omega\!\in\!\{1.0,\,0.5,\,0.1\}$, and twelve-electron dots at the same three confinements. As $\omega$ decreases, the well broadens and correlations strengthen; nevertheless, the Slater\,+\,smooth-$F_\theta$\,+\,analytic-cusp ansatz remains stable and accurate in practice.

\paragraph{Training protocol (summary).}
Unless otherwise stated, we use residual-based pretraining to obtain a well-conditioned initializer, followed by a Stochastic Reconfiguration (SR) tail to tighten the last digits of the energy. The SR phase uses $\mathcal{O}(10^3\!-\!10^3.5)$ iterations with batches of $\sim\!3{\times}10^3$ samples per iteration.

\subsubsection{Energies}

Table~\ref{tab:Residual} reports two-electron energies from \emph{residual-only} training (PINN and PINN+BF). For $N{=}2$, backflow is not required to reach DMC accuracy, but it tends to reduce variance and smooth convergence.

\begin{table}[h!]
\centering
\caption{Ground-state energies (Hartree) for two electrons in a harmonic trap using \emph{residual-only} training. DMC references as in [Ref.]. Parentheses denote $1\sigma$ on the last digits.}
\label{tab:Residual}
\begin{tabular}{c c c c}
\hline
$\omega$ & PINN+BF & PINN & DMC (Ref.) \\
\hline
1.00  & 2.99998(3)   & 2.999940(5)  & 3.00000(1) \\
0.50  & 1.65975(2)   & 1.65979(3)   & 1.65977(1) \\
0.10  & 0.440795(3)  & 0.440833(1)  & 0.44079(1) \\
0.01  & 0.0740724(6) & 0.0741679(6) & 0.073839(2) \\
\hline
\end{tabular}
\end{table}

Table~\ref{tab:energies} compares diffusion Monte Carlo (DMC) references with our \emph{final} PINN+BF energies after residual pretraining and an SR tail. From the ultra–shallow two-electron case ($\omega{=}0.01$) to the many-electron systems, absolute deviations are small; for $N{=}2$ they lie within DMC uncertainties, and for $N\!\in\!\{6,12\}$ the relative deviations are $\mathcal{O}(10^{-2}\,\%)$–$\mathcal{O}(10^{-1}\,\%)$.

\begin{table}[htbp]
  \centering
  \caption{Residual-based pretraining + SR tail. Reference DMC and our PINN+BF ground-state energies (Hartree) for selected $(N,\omega)$. Parentheses denote $1\sigma$ on the last digit(s). The final column is the relative deviation from DMC (\%).}
  \label{tab:energies}
  \begin{threeparttable}
    \begin{tabular}{c|c|ccc}
      \toprule
      $N$ & $\omega$ & Reference (DMC) & PINN+BF & \%\,err \\
      \midrule
      2 & 0.01 & $0.073839(2)$ & $0.073838(1)$ & $-0.0014$ \\
        & 0.10 & $0.44079(1)$  & $0.44079 \pm 0.00001$ & $+0.0000$ \\
        & 0.50 & $1.65977(1)$  & $1.65976 \pm 0.00002$ & $-0.0006$ \\
        & 1.00 & $3.00000$     & $2.99999 \pm 0.00001$ & $-0.0003$ \\
      \midrule
      6 & 0.10 & $3.55385(5)$  & $3.5549 \pm 0.0001$ & $+0.0295$ \\
        & 0.50 & $11.78484(6)$ & $11.7895 \pm 0.0004$ & $+0.0395$ \\
        & 1.00 & $20.15932(8)$ & $20.1610 \pm 0.0006$ & $+0.0083$ \\
      \midrule
      12 & 0.10 & $12.26984(8)$ & $12.2782 \pm 0.0003$ & $+0.0681$ \\
         & 0.50 & $39.1596(1)$  & $39.1886 \pm 0.0008$ & $+0.0748$ \\
         & 1.00 & $65.7001(1)$  & $65.723 \pm 0.001$   & $+0.0364$ \\
      \bottomrule
    \end{tabular}
  \end{threeparttable}
\end{table}

\paragraph{Summary.}
A compact Slater–Jastrow PINN with soft-core pair features and an explicit analytic cusp achieves DMC-level accuracy for $N{=}2$ across $\omega$, and for $N\!\in\!\{6,12\}$ reaches small, systematic relative deviations (roughly $0.01$–$0.08\%$ across the grid). In terms of sample efficiency, the SR tail uses on the order of $10^3$ iterations with $\sim\!3{\times}10^3$ samples each; this is \emph{orders of magnitude fewer} training samples than some earlier neural approaches to quantum dots (see \S\ref{sec:related} for a quantitative comparison). We attribute the stability to conditioning choices (trap-unit scaling, soft-core features with $ds/dr\!\to\!0$ at coalescence, and an explicit cusp), which keep both gradients and Laplacians numerically tame from shallow to tight traps. Convergence traces and ablations (activation, initialization, SR settings, backflow on/off) are reported in Appendix~X.

\section{What the networks learn: geometry of the correlator and of backflow}
\label{sec:repr-analysis}

We characterize the learned \emph{correlator} $f_{\text{net}}$ (the PINN residual readout fed by
$\overline{\phi}\,\|\,\overline{\psi}\,\|\,\mathbf g$) and the \emph{BackflowNet} by probing their
representations on $|\Psi|^2$ samples. We report:
(i) the entropy \textbf{effective rank} $r_{\text{eff}}=\exp\!\big(-\sum_i p_i\log p_i\big)$ of the feature
covariance spectrum (a variance-weighted dimensionality);
(ii) \textbf{PC ablations} where we project inputs onto the top-$k$ PCs and recompute the nonlinear head,
reporting normalized error;
(iii) \textbf{PC1 block power} to apportion the leading axis across $\{\phi,\psi,\text{extras}\}$;
(iv) \textbf{linear probes} from PC scores to coarse physical summaries; and
(v) \textbf{near-field} shares computed by conditioning on the lowest-$q\%$ of the per-configuration minimum pair distance.

\subsection{Correlator geometry ($f_{\text{net}}$)}
The correlator is \emph{consistently low-dimensional} (Table~\ref{tab:fnet-geometry}): $r_{\text{eff}}\!\in[1.25,3.0]$,
and the head evaluated on PC1 alone correlates strongly with the full head ($\rho\!\in[0.94,\,0.999]$).
Which branch drives PC1 is regime dependent:
global/extras dominate for tight 2e; per-particle ($\phi$) dominates for loose 6e; and pairs ($\psi$) dominate for 6e at intermediate/tight traps. Although variance concentrates in a few PCs, PC ablations show the head can remain sensitive to additional low-variance directions (typically $k\!\approx\!6$–$8$ PCs for stable reconstruction at 6e). Near-field residual-gradient shares locate where the amplitude model is most active: concentrated at small separations for 2e loose traps, more uniform in 6e tight traps.

\begin{table*}[t] \centering \small \setlength{\tabcolsep}{6.5pt} \begin{tabular}{cccccccc} \toprule $N$ & $\omega$ & $r_{\rm eff}(Z)$ & corr(head, head@PC1) & \multicolumn{3}{c}{PC1 block power} & Near-field residual grad (q=5\%) \\ & & & & $\phi$ & $\psi$ & extras & share \\ \midrule 2 & 0.10 & 1.253 & 0.9966 & 0.004 & 0.615 & 0.382 & 3.709 \\ 2 & 1.00 & 1.272 & 0.9986 & 0.000 & 0.327 & 0.673 & 0.437 \\ 6 & 0.10 & 1.775 & 0.9387 & 0.702 & 0.215 & 0.083 & 1.582 \\ 6 & 0.50 & 2.986 & 0.9691 & 0.155 & 0.650 & 0.195 & 0.989 \\ 6 & 1.00 & 1.270 & 0.9979 & 0.090 & 0.718 & 0.192 & 0.966 \\ \bottomrule \end{tabular} \caption{\textbf{Geometry of the correlator $f_{\text{net}}$.} Entropy effective rank of $\rho$'s input $Z$, alignment of the head with its top PC, composition of the leading axis across branches, and near-field amplification of residual gradients (ratio of mean $\|\nabla f_{\rm res}\|^2$ on the lowest 5\% $r_{\min}$ to the global mean).} \label{tab:fnet-geometry} \end{table*}

\paragraph{Implications.}
(i) The correlator compresses onto $1$–$3$ macro-axes, supporting a small, smooth head.
(ii) The dominant latent axis shifts with $(N,\omega)$—global $\rightarrow$ per-particle $\rightarrow$ pair—which
peaks in complexity (largest $r_{\text{eff}}$) at intermediate $\omega$ for 6e.
(iii) Sensitivity to low-variance directions motivates optional PC-aware regularization.

\subsection{BackflowNet geometry ($\Delta x$ and messages)}
Backflow learns a \emph{vector field} $\Delta x$ that bends nodes locally and is therefore intrinsically higher-dimensional.
For 6e, $r_{\text{eff}}(\Delta x)\!\approx\!10$–$11$ (of $2N{=}12$ DoF) with a flat spectrum; for 2e, $r_{\text{eff}}\!\approx\!2.3$–$2.8$
(of $4$ DoF) with two dominant modes. PC ablations confirm the flow lies on a specific manifold (top PCs reconstruct well).
Linear probes from $\Delta x$ PCs to global summaries are nearly null at 6e, indicating that backflow performs \emph{local}
nodal adjustments rather than global rescalings. Input-channel ablations identify \emph{radial} channels ($r^2$, then $r$)
as the dominant drivers, consistent with the safe features and analytic cusp.

\begin{table*}[t] \centering \small \setlength{\tabcolsep}{6.0pt} \begin{tabular}{cccccccc} \toprule $N$ & $\omega$ & $r_{\rm eff}(\Delta x)$ & PC ablation ($k{=}2$) & PC ablation ($k{=}4$) & $R^2$ size & $R^2$ var & Near-field $\|\Delta x\|^2$ (q=5\%) \\ \midrule 2 & 0.10 & 2.790 & $0.34$ & $\,\to 0$ & 0.60 & 1.00 & 0.131 \\ 2 & 1.00 & 2.348 & 0.195 & $\,\to 0$ & 0.052 & 1.00 & 0.067 \\ 6 & 0.10 & 10.421 & 0.872 & 0.751 & $1.1{\times}10^{-4}$ & $1.8{\times}10^{-4}$ & 1.409 \\ 6 & 0.50 & 10.631 & 0.889 & 0.765 & $6.6{\times}10^{-5}$ & $2.1{\times}10^{-4}$ & 1.127 \\ 6 & 1.00 & 10.883 & 0.882 & 0.751 & $5.3{\times}10^{-4}$ & $4.3{\times}10^{-4}$ & 0.988 \\ \bottomrule \end{tabular} \caption{\textbf{Geometry of the BackflowNet.} Entropy effective rank of the displacement field, relative reconstruction error of $\Delta x$ after projecting onto the top-$k$ PCs (smaller is better), linear-probe $R^2$ from $\Delta x$ PCs to global size/variance of pair distances, and near-field amplification of $\|\Delta x\|^2$ at the lowest 5\% $r_{\min}$.} \label{tab:bf-geometry} \end{table*}

\paragraph{Implications.}
(i) For 6e, backflow is a many-mode deformation (most spatial modes are active).
(ii) The flow is radial-feature driven, aligning with cusp preservation.
(iii) Global size/variance linearly explain the 2e flow but not the 6e flow (which is local/structured).

\subsection{How the correlator manifold moves under backflow}
Comparing correlator PCs with and without backflow shows that PC1 is \emph{stable} (cosine $\approx$ 1) across regimes;
the change $\Delta Z$ in pooled features has small effective rank ($\sim 1.25$–$3.0$), peaking at $\omega{=}0.5$ for 6e.
Thus, backflow does not alter \emph{what} the correlator summarizes; it subtly reweights a small set of directions.

\begin{table}[h] \centering \small \caption{Alignment of correlator features across samplers. `PC1 cosine'' is the cosine between no-BF and BF PC1 directions. $r_{\text{eff}}(\Delta Z)$ is the entropy rank of the change in $\rho_{\text{in}}$.} \label{tab:alignment} \begin{tabular}{lcc} \toprule System & PC1 cosine (noBF vs BF) & $r_{\text{eff}}(\Delta Z)$ \\ \midrule 2e,\ $\omega{=}0.1$ & 0.999988 & 1.25 \\ 2e,\ $\omega{=}1.0$ & 0.999999 & 1.27 \\ 6e,\ $\omega{=}0.1$ & 0.999657 & 1.75 \\ 6e,\ $\omega{=}0.5$ & 0.998472 & \textbf{3.00} \\ 6e,\ $\omega{=}1.0$ & 0.999334 & 1.27 \\ \bottomrule \end{tabular} \end{table}

\paragraph{Design takeaway.}
Keep $\rho$ small (1–3 axes suffice) and route capacity where the regime demands (pair features for intermediate/tight 6e;
extras/$\phi$ for tight 2e or loose 6e). For backflow, invest width in \emph{radial} pair channels; high $r_{\rm eff}(\Delta x)$
is expected and reflects rich spatial modes rather than overfitting.




\section{Wigner–molecule crossover in 2D quantum dots}
\label{sec:wigner-molecule}

\subsection{What we measure}
For each $(N,\omega)$ we draw $\mathbf X\!\sim |\Psi|^2$ and report:
(i) the radial pair distribution $g(r)$,
(ii) the single-particle radial law $P(r)\propto r^{d-1}g(r)$ ($d{=}2$),
and scalar summaries extracted from $P(r)$: $r_{\rm mode}$ (mode), $\langle r\rangle$, $\sigma_r$, FWHM, and quantiles $q_{10},q_{50},q_{90}$.
We condense radial localization via $\gamma=\sigma_r/r_{\rm mode}$ (smaller $\gamma$ = stiffer localization).
For $N>2$ we also monitor an angular ring–order parameter $|\Phi_m|$ on resolved shells and the pair–distance Lindemann index $\gamma_{r_{12}}$.
All lengths are in Bohr.

\paragraph{Density parameter $r_s$.}
Following Egger \emph{et al.}, we compute the density parameter from the first maximum $r^\ast$ of the (spin-summed) pair correlation function:
\[
r_s \equiv \frac{r^\ast}{a_B^\ast}, \qquad 
r^\ast = \arg\max_r \sum_S g_S(r).
\]

which provides a universal control knob for the weak\,$\to$\,strong correlation crossover in parabolic dots.  The crossover occurs at a surprisingly small value $r_c\simeq 4$; for $r_s>r_c$ the observables are consistent with a Wigner–molecule description, while for $r_s<r_c$ they match Fermi–liquid behavior. \emph{(Definition and criterion from Egger \emph{et al.}).} :contentReference[oaicite:0]{index=0} :contentReference[oaicite:1]{index=1}

\subsection{Two electrons ($N=2$): monotone approach to the Wigner side}
Decreasing $\omega$ shifts probability density to larger $r$ and simultaneously narrows its relative width, reflected in the steady drop of $\gamma=\sigma_r/r_{\rm mode}$ (Table~\ref{tab:two_e_summary}).  
A power-law fit $r_{\rm mode}\!\sim\!C\,\omega^{\alpha}$ over $\omega\in\{1,0.5,0.1,0.01\}$ yields $\alpha\simeq-0.62$, close to the classical two-body scaling $\alpha=-2/3$.  
This indicates a smooth and monotone evolution from a correlated Fermi pair toward a localized two-site Wigner dimer.

\begin{table}[H]
\centering
\caption{$N=2$ summary across $\omega$. Distances in Bohr.}
\label{tab:two_e_summary}
\begin{tabular}{lcccccccc}
\toprule
$\omega$ 
& $r_{\rm mode}$ 
& $\langle r\rangle$ 
& $\sigma_r$ 
& FWHM 
& $q_{10}$ 
& $q_{50}$ 
& $q_{90}$ 
& $\gamma=\sigma_r/r_{\rm mode}$ \\
\midrule
1.00 & 1.435 & 1.632 & 0.705 & 1.743 & 0.735 & 1.581 & 2.581 & 0.491 \\
0.50 & 2.303 & 2.475 & 1.007 & 2.466 & 1.209 & 2.409 & 3.829 & 0.437 \\
0.10 & 6.347 & 6.716 & 2.303 & 5.588 & 3.819 & 6.614 & 9.780 & 0.363 \\
0.01 & 25.807 & 27.230 & 7.035 & 16.039 & 18.548 & 26.885 & 36.540 & 0.273 \\
\bottomrule
\end{tabular}
\end{table}

\paragraph{Takeaway.}
All observables—growth of $r_{\rm mode}$, shrinking $\gamma$, and tighter anti-alignment around $\pi$—evolve smoothly as $\omega$ decreases.  
The two-electron system thus provides a clean microscopic realization of the Wigner-molecule crossover.

---

\subsection{Six electrons ($N=6$): one ring persists, correlations stiffen}
For $N=6$ the radial distributions retain a single pronounced shell across the examined $\omega$ range, while the localization ratios and Lindemann indices progressively decrease (Table~\ref{tab:N6}).  
The structure remains one-ring, but correlations become increasingly stiff as confinement weakens.

\begin{table}[H]
\centering
\caption{$N=6$ diagnostics vs.\ $\omega$. Distances in Bohr; angles in radians.}
\label{tab:N6}
\begin{tabular}{c|ccccc}
\toprule
$\omega$ & $r_{\rm mode}$ & $\sigma_r$ & $\gamma=\sigma_r/r_{\rm mode}$ & $\sigma(\text{ring angles})$ & $\gamma_{r_{ij}}$ \\
\midrule
1.0  & 2.0907 & 0.9249 & 0.4424 & 0.7046 & 0.3538 \\
0.50 & 3.0702 & 1.3648 & 0.4445 & 0.6943 & 0.3397 \\
0.10 & 8.1686 & 3.5025 & 0.4288 & 0.6842 & 0.3149 \\
0.01 & 33.1538 & 13.9094 & 0.4195 & 0.6497 & 0.2889 \\
\bottomrule
\end{tabular}
\end{table}

\paragraph{Takeaway.}
Across two decades of confinement, the geometry remains single-ring while relative fluctuations shrink.  
The decreasing $\gamma$ and $\gamma_{r_{ij}}$ indicate growing correlation strength—a continuous drift toward the Wigner regime without topological transition.

---

\subsection{Six electrons ($N=6$): near-degenerate $(1,5)$ and $(0,6)$ topologies at weak confinement}

At $\omega=0.01$ the six-electron system exhibits all hallmarks of a Wigner molecule, yet with a striking internal degeneracy between two symmetry classes.

\paragraph{Shell statistics.}
From $1.2{\times}10^6$ sampled configurations we find that $69.2\%$ correspond to a $(1,5)$ shell (one central electron plus a five-site ring) and $30.8\%$ to a $(0,6)$ hexagonal ring.  
The pair-distance histogram decomposes exactly into outer–outer and inner–outer contributions with $w_{\rm OO}\!:\!w_{\rm IO}=2\!:\!1$, reproducing the global $g(r)$ with cosine similarity $1.0000$.  
Split-half stability diagnostics give $\text{JSD}=6.2{\times}10^{-6}$ and $\|{\Delta}\rho\|_2=3.9{\times}10^{-4}$, confirming stationary and robust sampling.

\paragraph{Bond–orientational order.}
Phase-aligned frames within each class yield
\[
|\Phi_5| = 0.453 \pm 0.214 \quad (\text{for $(1,5)$}), \qquad
|\Phi_6| = 0.466 \pm 0.206 \quad (\text{for $(0,6)$}),
\]
with angular Lindemann ratios $\gamma_{\rm ring}\!\approx\!0.33$ and $0.32$.  
Alternative orientational measures ($|\Psi_5|{=}0.422\!\pm\!0.203$, $|\Psi_6|{=}0.409\!\pm\!0.199$) corroborate this near equality.

\paragraph{Physical picture.}
The coexistence of $(1,5)$ and $(0,6)$ geometries signals a \emph{rotating Wigner molecule} (RWM): quantum fluctuations restore rotational symmetry and drive tunneling between two nearly degenerate classical minima.  
The lab-frame density remains ring-like, but crystalline order is recovered upon phase alignment.  
Localization ratios ($\gamma=0.4195$, $\gamma_{r_{12}}=0.2889$) and small angular Lindemann indices demonstrate a stiff yet quantum-delocalized structure.

\paragraph{Interpretation and comparison.}
This mixed-symmetry regime matches the predictions of Egger \emph{et al.} (\emph{Phys.\ Rev.\ Lett.}\ \textbf{82}, 3320, 1999) and Yannouleas \& Landman (\emph{Rep.\ Prog.\ Phys.}\ \textbf{70}, 2067, 2007):
\begin{quote}
“Even at large $r_s$, the six-electron dot remains ring-like with a blurred central density; the $(1,5)$ and $(0,6)$ configurations are nearly degenerate.”
\end{quote}
Our neural-network wavefunction reproduces this quantitative balance of topology fractions, polygonal order, and stiffness—demonstrating accurate Wigner-molecule behavior without modifying the ansatz.

---

\subsection{Twelve electrons ($N=12$): two-shell formation and $(1,11)$–$(3,9)$ competition}

At $\omega=0.01$ the twelve-electron dot enters a deep Wigner regime.  
From the first pair-distance peak we estimate $r_s\!\approx\!25$–$30$, far beyond the $r_s\simeq4$ crossover identified by Egger \emph{et al.}, confirming Wigner localization.

\paragraph{Shell statistics.}
A robust two-shell structure emerges, with fractions of two-shell frames depending weakly on the detection threshold $\tau$:
\[
f_{\text{two-shell}}=
\begin{cases}
0.761 & (\tau=3.0),\\
0.872 & (\tau=2.5),\\
0.956 & (\tau=2.0).
\end{cases}
\]
The inner-count histogram is broad, but the commensurate $(3,9)$ shell appears persistently at $\approx13.7\%$ across thresholds.  
Shell-decomposed histograms reproduce the global $g(r)$ (cosine similarity $\ge0.9998$), and the fitted II/IO/OO weights $(0.045{:}0.409{:}0.545)$ match the pair combinatorics of two concentric shells.

\paragraph{Sector-resolved bond order.}
The dominant $(1,11)$ sector shows modest orientational order, $|\Phi_{11}|=0.276\pm0.140$, with ring Lindemann $\approx0.495$ and a softly vibrating center ($\langle r_c\rangle\!\approx\!4.5$~Bohr, $\sigma(r_c)\!\approx\!2.3$~Bohr).  
In contrast, the commensurate $(3,9)$ subset exhibits stronger orientational locking:
\[
|\Phi_3|=0.561\pm0.246, \quad \text{Lind}_{\text{inner}}\!\approx\!0.342; \qquad
|\Phi_9|=0.321\pm0.161, \quad \text{Lind}_{\text{outer}}\!\approx\!0.408.
\]

\paragraph{Physical interpretation.}
The $(1,11)$ configuration corresponds to a soft, rotating Wigner molecule with weak inner-ring order, while the $(3,9)$ sector displays enhanced crystalline rigidity—consistent with the classical $N{=}12$ ground state reported by Bedanov and Peeters for Coulomb clusters.  
The coexistence of both occupancies reflects a \emph{quantum mixture of near-degenerate shell topologies}: radial shelling is complete, but azimuthal order is only partially frozen.

\paragraph{Novelty and context.}
To our knowledge, no previous quantum many-body study has reported explicit occupancy fractions or bond-order parameters for $(1,11)$ and $(3,9)$ geometries in $N=12$ parabolic dots.  
Earlier PIMC work (Harting, Mülken \& Borrmann, 2000) and RWM analyses (Yannouleas \& Landman, 2006) identified general shell formation up to $N\le10$ but did not quantify topology competition.  
Our data thus provide the first quantitative characterization of this two-shell interplay within a fully variational neural ansatz.

\paragraph{Summary.}
The $N=12$ system at $\omega=0.01$ exhibits a rotating, floppy Wigner molecule:  
high two-shell probability, perfect reconstruction of the global $g(r)$ from shell mixtures, and sector-dependent bond order with enhanced rigidity in the commensurate $(3,9)$ subset.  
Only at still weaker confinement (or under external pinning) would a rigid classical crystal emerge.

