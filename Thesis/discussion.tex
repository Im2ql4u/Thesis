\chapter{Quantum Dots}
\section{Discussion of energy results}
\label{sec:energy-discussion}

Taken together, Tables~\ref{tab:Residual} and~\ref{tab:energies} show that the
Slater--Jastrow PINN (with and without backflow) is able to reproduce
state-of-the-art DMC energies across a fairly demanding grid in $(N,\omega)$,
and to extrapolate in a controlled way into regimes where no DMC reference is
available (the ultra-weak traps at $\omega=10^{-3}$).

\paragraph{Two electrons: residual-only vs.\ SR-refined.}
For $N{=}2$, residual-only training already provides an excellent test of the
bare architecture. Table~\ref{tab:Residual} shows that across
$\omega\in\{1.0,0.5,0.1,0.01\}$ both the plain PINN and PINN+BF track the DMC
references at the $\sim 10^{-4}$\,Ha level or better, with deviations well
within quoted uncertainties except for the ultra-shallow $\omega{=}0.01$
pretraining point. There, residual-only PINN+BF remains safely variational
(energies slightly above DMC) but sits about $3\times 10^{-4}$\,Ha high,
indicating that some residual bias remains after the residual stage alone.

The final SR-refined energies in Table~\ref{tab:energies} remove this bias:
for $N{=}2$ and all $\omega\ge 0.01$, the PINN+BF results are statistically
indistinguishable from DMC, with relative deviations at or below
$10^{-3}\,\%$. This confirms that, given a well-conditioned ansatz, a short
SR tail is sufficient to close the last few $10^{-4}$\,Ha of the variational
gap in this benchmark system.

The comparison between PINN and PINN+BF in Table~\ref{tab:Residual} also
supports the claim that backflow is \emph{not necessary} to reach DMC-level
accuracy for two electrons. Energies with and without backflow agree within
error bars at all but the loosest trap, and even there the improvement from
backflow is small compared to the SR refinement. In practice, the main role
of backflow in the $N{=}2$ case is to reduce the variance and stabilize
optimization in the shallow-trap regime, rather than to provide a qualitatively
new nodal structure.

\paragraph{Scaling to $N{=}6$ and $N{=}12$.}
The more demanding tests are the many-electron dots in
Table~\ref{tab:energies}. For $N\in\{6,12\}$ and the traps where DMC
references are available ($\omega\in\{0.1,0.5,1.0\}$), the final PINN+BF
energies are consistently close to DMC and always slightly higher, as
expected from a variational method. The relative deviations lie in the
range $\sim 0.026\text{--}0.048\%$ for the interacting cases, i.e.\ on
the order of a few $10^{-4}$\,Ha in absolute terms. There is no sign of a
catastrophic growth in error with either particle number or trap strength:
the worst relative deviations occur at intermediate $\omega$ and remain
comfortably below the percent level.

This behaviour is non-trivial for two reasons. First, the dots span very
different physical regimes: from relatively compact, kinetic-energy-dominated
states at $\omega=1.0$ to highly correlated, interaction-dominated states at
$\omega=0.1$. Second, the same architecture and training protocol are used
across all $(N,\omega)$, up to trivial scaling in trap units. The fact that
a single Slater--Jastrow+backflow ansatz can maintain sub-$0.1\%$ accuracy
across this grid suggests that the chosen combination of soft-core pair
features, analytic cusp, and trap-unit normalization is robust to both
particle number and confinement.

\paragraph{Ultra-weak traps: predictions at $\omega=10^{-3}$.}
For the ultra-weak traps at $\omega=10^{-3}$ there are no DMC references
in Table~\ref{tab:energies}; the PINN+BF values
($E\approx 0.0138$\,Ha for $N{=}2$, $0.1409$\,Ha for $N{=}6$,
$0.5158$\,Ha for $N{=}12$) should be regarded as high-quality variational
predictions rather than benchmarked numbers. Several consistency checks
nevertheless support their reliability:

\begin{itemize}
  \item The energies connect smoothly to the DMC-validated points at
        $\omega=0.01$ under the log--log scaling analysis of $T$, $V_{\rm int}$,
        and $V_{\rm trap}$ in Sec.~\ref{sec:wigner-molecule}: the inferred
        exponents for $E(\omega)$ and the partitioning between trap and
        interaction energies are consistent with classical expectations in
        the Wigner regime.
  \item For $N{=}2$ at $\omega=10^{-3}$, independent MCMC diagnostics
        (stable $r_{12}$ histograms, consistent one-body densities, smooth
        local-energy distributions) indicate that the sampling has converged
        and the residual statistical error is well below the quoted
        uncertainties in Table~\ref{tab:energies}.
  \item For $N{=}6$ and $N{=}12$, the structural and virial diagnostics in
        Sec.~\ref{sec:wigner-molecule} confirm that the states are in the
        expected Wigner-molecule regime (correct shell topologies,
        near-classical virial ratios), with no signs of pathologies that
        would hint at an incorrect energy scale.
\end{itemize}

Within these caveats, the $\omega=10^{-3}$ values can be taken as reliable
anchors for the Wigner-molecule analysis that follows.

\paragraph{Sample efficiency and conditioning.}
From an algorithmic perspective, these results also speak to the
\emph{efficiency} of the chosen training scheme. The SR tail uses on the
order of $10^3$ iterations with $\sim 3\times 10^3$ samples each, i.e.\ a
few million local-energy evaluations per $(N,\omega)$ point. This is modest
compared to some earlier neural approaches to quantum dots that require
orders of magnitude more samples to reach comparable accuracy.
The good performance of residual-only pretraining (especially at $N{=}2$)
and the small SR corrections suggest that most of the work is done by
learning a well-conditioned, physically informed ansatz rather than by
brute-force stochastic optimization.

The conditioning choices are crucial here:
working in trap units keeps the typical coordinates $\mathcal{O}(1)$ from
tight to shallow traps; soft-core pair features with $ds/dr\to 0$ at
coalescence and an explicit analytic cusp prevent large gradients and
Laplacian spikes near $r_{ij}=0$. As a result, local energies remain
numerically well-behaved even in the ultra-weak traps where electrons are
far apart and the wavefunction varies on long length scales.

\paragraph{Role of backflow across regimes.}
Finally, these tables let us separate the \emph{energetic} role of backflow
from its structural and representational role. For $N{=}2$ backflow has
negligible impact on the final SR-refined energies and is not required for
DMC-level accuracy. For $N\in\{6,12\}$ at intermediate traps
($\omega\in\{0.1,0.5\}$), backflow contributes a noticeable but still modest
improvement over the residual-only baselines (not shown in
Table~\ref{tab:energies}), bringing the final errors into the
$\mathcal{O}(10^{-2}\text{--}10^{-1})\%$ band. In the deepest Wigner
regimes ($\omega=10^{-3}$) the backflow correction is energetically tiny
(see Sec.~\ref{sec:repr-analysis}); here the base Slater--Jastrow correlator
already captures the essential structure and backflow becomes nearly
variationally inert.

This regime dependence will be explored in more detail in
Sec.~\ref{sec:wigner-molecule} and Sec.~\ref{sec:repr-analysis}, where we
show how the same ansatz that yields the accurate energies in
Table~\ref{tab:energies} also reproduces the expected Wigner-molecule
structure and compresses it into a low-dimensional correlator manifold
with a high-dimensional, largely local backflow correction.


\section{Discussion of the Wigner–molecule regime}
\label{sec:wigner-discussion}

The structural diagnostics in Sec.~\ref{sec:wigner-molecule} show that, for the
weakest traps considered, the dots are genuinely on the Wigner side and not
merely ``strongly correlated'' in a loose sense. Three complementary pieces of
evidence support this: (i) localization and energy partitioning, (ii) shell
geometry and bond order, and (iii) the statistics of competing topologies.

\paragraph{Localization and energy partitioning.}
On the one-body level, the weak-trap radial laws $P(r)$ are highly localized
around large radii with small relative fluctuations. For the two-electron dot
at $\omega=10^{-3}$ we find
$r_{\rm mode}\approx 64.3$, $\langle r\rangle\approx 64.0$ and
$\gamma=\sigma_r/r_{\rm mode}\approx 0.29$, with only
$\sim 4.6\times 10^{-3}$ of the single-particle mass inside
$r\le 0.25\,r_{\rm mode}$. The pair distribution $p(r_{12})$ is peaked at
$r_{12}^{\rm mode}\approx 122$ with a Lindemann ratio
$\gamma_{r_{12}}\approx 0.20$, and the relative angle is sharply
anti-aligned, $\sigma(\pi-\Delta\phi)\approx 0.35$, with
$p(r_{12}\le 0.25\,r_{12}^{\rm mode})\approx 2.6\times 10^{-3}$.
These numbers are characteristic of a dilute two-site Wigner dimer rather
than a delocalized Fermi liquid.

For $N\in\{6,12\}$, the same pattern reappears in the aggregate:
$r_{\rm mode}$ and $r_{\rm rms}$ grow roughly as $\omega^{-0.6}$,
$\gamma$ decreases only mildly with $N$ and $\omega$, and the inferred
$\langle r^2\rangle\propto\omega^{-\beta}$ with
$\beta\simeq 1.2\text{--}1.3$ exceeds the non-interacting $\beta=1$.
At the same time, the energy partitioning approaches the classical
virial-like ratio:
\[
\frac{2\langle V_{\rm trap}\rangle}{\langle V_{\rm int}\rangle}
\simeq 1.25,\;1.12,\;1.02 \quad\text{for}\quad N=2,6,12
\quad \text{at }\omega=10^{-3},
\]
with $N=12$ satisfying $V_{\rm int}\approx 2V_{\rm trap}$ to within~$2\%$.
Combined with the log--log scaling exponents $\alpha_T\approx 1$ and
$\alpha_{V_{\rm int}}\approx 0.66$, this indicates that the kinetic energy
has become a small correction on top of a nearly classical potential-energy
landscape, as expected in the Wigner-molecule limit.

\paragraph{Shell geometry and bond order.}
Beyond global localization, the weak-trap states exhibit clear shell
structure with non-trivial bond order. For $N{=}6$ the dot retains a
single pronounced shell across the entire $\omega$ grid; what changes
is the stiffness of angular correlations. The ring Lindemann index drops
from $\sim 0.70$ at $\omega=1$ to $\sim 0.64$ at $\omega=10^{-3}$, while
the pair-distance Lindemann ratio decreases from $\sim 0.35$ to $\sim 0.27$
(see Table~\ref{tab:N6}). After angular registration, the electrons form a
rotating Wigner ring with well-defined sixfold order.

At $\omega=10^{-3}$, the $N{=}6$ ensemble is dominated by two symmetry
classes, $(1,5)$ and $(0,6)$, with fractions
$\approx 0.801$ and $\approx 0.199$.
Both show strong ring order:
$|\Phi_5|\approx 0.60$, $|\Phi_6|\approx 0.66$ and
ring Lindemann $\approx 0.21$.
The fact that the shell never breaks, and the system fluctuates only between
these two topologies, matches classical predictions where $(1,5)$ is the
ground-state geometry and $(0,6)$ a low-lying competitor.

For $N{=}12$, the situation is richer but equally structured. At
$\omega=10^{-3}$, the two-shell fraction is very high across a range
of radial-gap thresholds ($\sim 0.99,0.97,0.92$ for $\tau=2.0,2.5,3.0$),
indicating that shelling is robust rather than a threshold artifact.
Within the two-shell subset, the inner-ring occupancy histogram is sharply
peaked at $(3,9)$, with $(2,10)$ and $(1,11)$ providing the leading
subdominant topologies. On $(3,9)$ frames we find
$|\Phi_3|\approx 0.69$, $|\Phi_9|\approx 0.53$ and
Lindemann indices $0.22$ on both rings, i.e.\ clear threefold and ninefold
bond order with small angular fluctuations. This is precisely the type of
commensurate, shell-resolved crystalline order expected in a twelve-electron
Wigner molecule.

\paragraph{Topology statistics and reconstruction of $g(r)$.}
A particularly stringent check is the reconstruction of $g(r)$ from
shell-resolved sectors. For $N{=}12$ at $\omega=10^{-3}$ we partition
the two-shell configurations into II/IO/OO pairs, weight their
histograms by the combinatorial mixture $(0.045{:}0.409{:}0.545)$,
and compare the resulting synthetic $g(r)$ to the global one. The cosine
similarity $0.9982$ indicates that, within statistical resolution, the
global pair correlations are completely explained by the superposition of
shell-resolved contributions and combinatorics. The same holds for $N{=}6$,
where OO/IO mixtures of the $(1,5)$ and $(0,6)$ sectors reproduce $g(r)$
essentially exactly.

This has two implications. First, the observed structure in $g(r)$ is
unambiguously geometric (shells and topologies), rather than a subtle
finite-size or sampling artifact. Second, it shows that the neural ansatz
is not just matching energies: it is reproducing the full two-body
correlation structure expected from classical and quantum analyses of
parabolic Coulomb clusters~\cite{Egger_1999,Filinov_2001,schweigert1994spectralpropertieschargedparticles,Kong_2002}.

\paragraph{From dimer to multi-shell Wigner molecules.}
Looking across $N=2,6,12$ and decreasing $\omega$, a coherent picture
emerges. The two-electron system evolves into a dilute, strongly
anti-aligned dimer with negligible overlap near the origin. The six-electron
dot forms a single, stiff Wigner ring whose lab-frame symmetry is restored by
rotation but whose co-rotating frame shows clear sixfold order and a quantum
mixture of classically known $(1,5)$ and $(0,6)$ topologies. The twelve-electron
dot, finally, develops a robust two-shell structure with a commensurate
$(3,9)$ split as the modal configuration and quantitative agreement with
classical ground-state topologies.

The fact that all of this structure appears in samples drawn from a single
variational neural ansatz---whose energies match or closely track DMC
benchmarks---gives confidence that the states studied in
Sec.~\ref{sec:repr-analysis} are bona fide Wigner molecules. In the next
section we show that, despite this rich real-space structure, the learned
correlator compresses the Wigner regime into a remarkably low-dimensional
manifold and that backflow plays a strongly regime-dependent role, becoming
almost variationally inert in the deepest Wigner cases.

\section{Discussion of learned representations}
\label{sec:repr-discussion}

The analysis in Sec.~\ref{sec:repr-analysis} shows a surprisingly coherent picture
across all $(N,\omega)$: despite the rich real-space structure documented in
Sec.~\ref{sec:wigner-molecule}, the learned correlator $f_{\text{net}}$ lives
on a very low-dimensional manifold, while backflow realizes a comparatively
high-dimensional but strongly localized correction that becomes almost
variationally inert in the deepest Wigner regimes.

\paragraph{Low-dimensional correlator manifold.}
Across all systems, the entropy effective rank $r_{\rm eff}(Z)$ of the
correlator features remains close to~1, rarely exceeding~2
(Table~\ref{tab:fnet-geometry-upd}). For the most extreme case,
$N{=}2$ at $\omega=10^{-3}$, we find $r_{\rm eff}(Z)\approx 1.00$ and a
head--PC1 correlation of $\approx 0.998$, with PC1 block power almost
entirely in the pair branch $\psi$. The PC ablations confirm that this is
not an artifact of the metric: projecting inputs onto the top-$k$ PCs
reduces the head error dramatically (e.g.\ to $\mathcal{O}(10^{-2})$ for
$k\le 4$) whereas random $k$-dimensional subspaces leave almost all of the
error intact (Table~\ref{tab:head-ablations}). In other words, most of what
the head does is encoded along a single, very special latent axis.

The same pattern holds, with mild variations, for $N\in\{6,12\}$ and for
both shallow and intermediate traps. Effective ranks cluster between
$\sim 1.2$ and $\sim 2.0$, head--PC1 correlations stay above~0.95 in most
cases, and PC ablations show that a handful of principal directions capture
the bulk of the head sensitivity much more efficiently than random
subspaces. From a representation-learning perspective, the correlator thus
behaves more like a low-dimensional order-parameter model than a generic
high-dimensional feature extractor.

\paragraph{What the leading axes encode.}
Linear probes from the correlator PCs to coarse physical summaries clarify
what these axes actually represent. For $N{=}2$ (both at $\omega=0.01$ and
$10^{-3}$) the first few PCs predict $r_{\rm mean}$ and $r_{\rm var}$ essentially
perfectly ($R^2\approx 1$), and even the near-origin mass and
$P(r_{12}\le0.25\,r_{12}^{\rm mode})$ are well captured. In the six- and
twelve-electron dots the situation is slightly more complex, but the trend
is the same: $R^2(r_{\rm mean})$ remains close to unity and
$R^2(r_{\rm var})$ lies in the $0.4$–$0.85$ range across regimes, whereas
shell-contrast is only weakly linearly encoded
(Table~\ref{tab:fnet-geometry-upd}). This is consistent with the structural
analysis in Sec.~\ref{sec:wigner-molecule}: the correlator primarily tracks
global size and radial spread (how big is the dot, how swollen are the
shells), while the detailed angular order requires explicit registration of
configurations within a given shell topology.

\paragraph{Backflow as a local, high-dimensional correction.}
In contrast, the BackflowNet lives in a genuinely high-dimensional regime.
For $N{=}6$ and $N{=}12$ the displacement field $\Delta x$ has
$r_{\rm eff}(\Delta x)$ in the range $\sim 10$–$22$
(Table~\ref{tab:bf-geometry-upd}), with relatively flat PCA spectra and
gradual error reduction as more PCs are retained. Linear probes from $\Delta x$
PCs to global observables are essentially null: $R^2$ for $r_{\rm mean}$,
$r_{\rm var}$, and shell contrast stays near zero, indicating that the
backflow field does not encode simple global rescalings or shape parameters.

Near-field diagnostics for $\Delta E$ and $\|\Delta x\|^2$ also support a
local interpretation. In intermediate traps (e.g.\ $N{=}6$, $\omega=0.01$,
or $N{=}12$, $\omega=0.01$), the lowest $q\%$ of configurations by
$r_{\min}$ carry a disproportionately large share of the backflow energy
correction: near-field $\Delta E$ shares at $q{=}1\%$ can exceed unity by
factors of $2$–$3$, and remain enhanced at $q{=}5\%$ and $10\%$
(Table~\ref{tab:bf-geometry-upd}). This is precisely where backflow is
expected to matter physically: when electrons approach each other and small
nodal adjustments can strongly affect the local energy.

\paragraph{Deep Wigner limit: backflow switches off.}
The extreme Wigner points provide an instructive counterexample. For
$N{=}2$ at $\omega=10^{-3}$, the backflow correction to the energy is
$\Delta E\sim 2\times 10^{-7}$\,Ha with uncertainty of the same order, the
PC1 alignment between no-BF and BF correlator features is essentially
perfect (cosine $=1$), the ranks of $Z$ and $\Delta Z$ are both $\approx 1$,
and the BackflowNet displacement field has vanishing impact on observables:
channel ablations produce no change, near-field $\|\Delta x\|^2$ shares are
exactly unity, and even the PCA of $\Delta x$ reveals a nearly trivial
structure. In this regime, backflow is effectively turned off by the
optimizer; the Slater--Jastrow correlator is sufficient to represent the
Wigner dimer.

For $N{=}6$ and $N{=}12$ at $\omega=10^{-3}$ the pattern is similar, albeit
less extreme. The correlator manifold remains low-rank, PC1 alignment
between no-BF and BF stays near one, and the entropy rank of $\Delta Z$ is
close to that of $Z$. Backflow still spans a higher-dimensional space in
principle, but its energetic effect is very small and its near-field energy
shares are close to unity. In other words, once the system is deep in the
Wigner-molecule regime and the static correlator has adapted to the
interaction-dominated landscape, backflow becomes nearly variationally
inert.

\paragraph{Regime dependence and architectural implications.}
Putting these observations together, the learned representation exhibits a
clear regime dependence:

\begin{itemize}
  \item In the \emph{weakly to moderately correlated} regime (tighter traps,
        smaller $r_s$), the correlator is still low-rank but backflow
        carries a non-trivial fraction of the energy correction, concentrated
        in near-field configurations. Here, widening the BackflowNet and
        providing rich pairwise inputs ($r_{ij}$, $\mathbf r_{ij}$, etc.)
        pays off, while the correlator can remain compact.

  \item In the \emph{deep Wigner} regime (ultra-weak traps at
        $\omega=10^{-3}$), the correlator alone captures both the global
        size/radius and the shell structure; backflow collapses to a
        nearly idle high-dimensional field. Extra expressivity in backflow
        does not translate into better energies or visibly different
        correlations.
\end{itemize}

From a design perspective, this suggests that Slater--Jastrow PINNs with
small, smooth correlators and moderately wide backflow layers are a good
default for parabolic dots. The correlator behaves like a learned
few-parameter ``collective coordinate'' summarizing the dot size and rough
shelling, while backflow functions as a flexible local nodal corrector that
only activates in regimes where near-field physics is delicate. In the
Wigner limit backflow naturally switches off, and the model reduces to a
static Wigner-molecule ansatz with low-rank latent geometry.

\paragraph{Relation to the physical picture.}
Finally, the representation analysis dovetails with the structural results
of Sec.~\ref{sec:wigner-molecule}. The same correlator that reproduces the
expected shell topologies, bond order, and virial partitioning in the
Wigner regime is found to live on a one- or two-dimensional manifold whose
leading axis correlates almost perfectly with global size and spread.
Backflow, in turn, is high-dimensional and local exactly in the regimes
where the shell structure is less rigid and near-field encounters are more
frequent. This supports the view that the ansatz has learned a physically
meaningful decomposition: a small set of global ``collective coordinates''
captured by $f_{\text{net}}$ and a local, many-body correction channel
implemented by BackflowNet.
