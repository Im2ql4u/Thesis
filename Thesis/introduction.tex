\chapter*{Introduction}

Quantum mechanics is a central pillar of modern physics. It provides the
framework in which we describe atoms, molecules, solids, and nanoscale
structures, and it underlies much of today’s technology. At the same time, it
leads directly to some of the most challenging computational problems we face.
The main example is the \emph{many-body problem}: solving the Schrödinger
equation for interacting particles and extracting physically meaningful
quantities from the resulting wavefunction.

In principle, the many-body wavefunction contains complete information about
the system. In practice, its dimension grows exponentially with the number of
particles and the size of the single-particle basis. The competition between
kinetic energy, which tends to delocalize particles, and interaction and
external potentials, which drive localization and correlations, takes place in
a configuration space that is far too large to handle naively. Nevertheless,
we need accurate results—energies, correlation functions, and phase
boundaries—to understand real materials and model strongly correlated systems.

A wide range of electronic-structure methods has been developed to address
this. Mean-field approaches such as Hartree–Fock replace the full many-body
problem with an effective single-particle picture, where each electron moves
in an average field from the others. This captures Fermi statistics and some
exchange effects but leaves most correlation physics untreated. More systematic
methods such as Full Configuration Interaction (FCI) are, in principle, exact
within a finite basis, but their cost grows combinatorially with system size.
Coupled Cluster (CC) theory provides a more efficient and size-extensive
hierarchy of approximations, but it remains computationally demanding and can
struggle in strongly correlated regimes or near degeneracies.

Quantum Monte Carlo (QMC) methods, including Variational Monte Carlo (VMC) and
Diffusion Monte Carlo (DMC), take a different approach: they represent the
wavefunction implicitly through stochastic sampling. For suitable trial
states, they can produce highly accurate benchmark energies and correlations
for continuum systems. However, QMC is not free of difficulties. The fermion
sign problem limits direct simulations, fixed-node approximations inherit bias
from the chosen trial nodes, and constructing high-quality trial wavefunctions
is increasingly difficult as systems become more strongly correlated or
inhomogeneous. In practice, a large part of the many-body problem becomes the
problem of choosing an expressive, well-conditioned ansatz and optimizing it
reliably.

\paragraph{Machine learning and many-body wavefunctions.}
In parallel with these developments, machine learning—especially deep
learning—has become a standard tool for representing complex, high-dimensional
functions and fitting them to data. Neural networks now play a central role in
domains such as computer vision, language modelling and reinforcement learning.
They combine flexible parametric families with inductive biases (e.g.\ symmetry
and locality), normalization, and optimization procedures that scale well with
dimension.

In physics, this has led to two broad uses of neural networks. In the first,
they act as \emph{surrogates}: they learn mappings from parameters or boundary
conditions to observables, or approximate solution operators of partial
differential equations. Physics-informed neural networks (PINNs) and
operator-learning approaches fall into this category by embedding differential
operators and constraints directly into the training objective. In the second,
neural networks are used as \emph{variational ansätze} for quantum states
(neural quantum states), where the network outputs amplitudes, phases or
log-probabilities for many-body configurations and is trained by variational
Monte Carlo.

In principle, neural networks provide highly expressive ansätze for quantum
problems. In practice, several issues arise. Objectives based on local
energies involve second derivatives and can be numerically delicate.
Coordinate parameterizations that do not respect known structure can lead to
large gradients and Laplacian spikes near electron coalescence. Generic
architectures can waste capacity on degrees of freedom that are irrelevant
from a physical point of view. These challenges motivate hybrid approaches in
which as much known structure as possible—antisymmetry, cusp conditions,
scaling, and symmetries—is built into the ansatz and its coordinates, and the
network is used to learn the remaining correlation corrections.

\paragraph{Quantum dots as a controlled testbed.}
Two-dimensional parabolic quantum dots provide a clean and tunable setting in
which to explore such ideas. They offer a simple route from Fermi-liquid–like
behaviour to Wigner-molecule physics. As the trap frequency $\omega$ decreases,
the single-particle level spacing collapses while Coulomb repulsion becomes
dominant. Electrons then tend to localize into ring or shell patterns. In the
lab frame, rotational symmetry and quantum fluctuations smear these into smooth
densities; in a co-rotating frame, one recovers crystalline-like structures
whose topology depends on particle number. For $N{=}6$ and $N{=}12$, classical
and semiclassical studies have identified preferred shellings such as $(1,5)$
and $(3,9)$ and mapped parts of the crossover
\cite{Egger_1999,Kong_2002,schweigert1994spectralpropertieschargedparticles,manninen2007metalclustersquantumdots}.

From the perspective of this thesis, quantum dots are attractive for two
reasons. First, they are genuinely non-trivial: Coulomb singularities, strong
correlations and an actual Fermi-liquid–to–Wigner crossover appear already for
moderate $N$. Second, they are still simple enough that accurate DMC
references exist in many regimes and that detailed structural diagnostics
(shell occupancies, bond order, virial ratios) are computationally feasible.
For fully quantum dots, however, quantitative topology-resolved diagnostics
(occupancy fractions, bond-orientational order, shell “stiffness”) and a
systematic picture of the Wigner regime remain relatively sparse. In addition,
it is not well understood how modern neural ansätze internally represent these
states.

\paragraph{Scope and questions.}
In this thesis, we revisit two-dimensional harmonic quantum dots using a
Slater–Jastrow(+Backflow) ansatz implemented as a physics-informed neural
network. The design is intended to be \emph{safe by construction} for
Laplacian-based objectives: coordinates are scaled to trap units
($\tilde R=\sqrt{\omega}R$), learned pair features are soft-core (avoiding
spurious $1/r$ behaviour in derivatives), and the electron–electron cusp is
enforced analytically rather than learned. A compact backflow network provides
smooth nodal deformations on top of a Slater determinant built from harmonic
oscillator orbitals.

Within this setup, we focus on three questions:
\begin{enumerate}
  \item How much neural capacity is actually required once antisymmetry, cusp
        conditions and basic conditioning are implemented analytically?
        In particular, can a relatively small, well-conditioned network match
        DMC energies across particle numbers and trap strengths?
  \item What additional structure does backflow learn on top of a good
        Slater–Jastrow correlator? Does it mainly act as a local, many-body
        correction near electron–electron encounters, or does it significantly
        reshape the global structure of the dot?
  \item Can we extract detailed Wigner physics—shell occupancies, bond
        order, and stiffness—from $|\Psi|^2$ samples with minimal modelling
        bias, and relate these to the latent representations learned by the
        network?
\end{enumerate}

\paragraph{Diagnostics and relation to prior work.}
To address these questions, we go beyond total energies. Alongside DMC
benchmarks, we use:
\begin{itemize}
  \item automatic shell detection and shell-resolved reconstructions of the
        pair distribution $g(r)$, testing whether the global $g(r)$ is
        explained by shell geometry and combinatorics;
  \item bond-orientational order parameters $|\Phi_m|$ and angular
        Lindemann-like ratios, adapted from studies of classical Wigner and
        Yukawa layers~\cite{Mazars_2008}, to quantify ring order and angular
        stiffness;
  \item the density parameter $r_s$, estimated from the first peak of $g(r)$
        following Egger \emph{et al.}~\cite{Egger_1999}, to locate our dots
        along the Fermi-liquid–to–Wigner scale;
  \item representation-focused diagnostics for the neural ansatz: entropy
        effective ranks of correlator and backflow features, principal
        component ablations, and simple linear probes that relate latent axes
        to physical observables (global size, radial variance, near-origin
        mass).
\end{itemize}
All of these are applied as post-processing to $|\Psi|^2$ samples, so the same
analysis could, in principle, be used to study classical, quantum-chemical, or
other neural wavefunctions on the same footing.

\paragraph{Main contributions.}
The main contributions of this thesis are:
\begin{enumerate}
  \item \textbf{Compact, conditioned accuracy.}  
        A small Slater–Jastrow(+Backflow) ansatz in trap units achieves
        DMC-level energies for $N{=}2$ and maintains
        $\sim 10^{-2}\text{--}10^{-1}\%$ relative errors for
        $N\in\{6,12\}$ across $\omega\in\{0.1,0.5,1.0\}$, while extrapolating
        smoothly to ultra-weak traps ($\omega=10^{-3}$). The results indicate
        that conditioning (trap units, analytic cusp, safe pair features) is
        at least as important as network size for stability and accuracy in
        these dots.
  \item \textbf{Topology-resolved Wigner diagnostics.}  
        Across $\omega\in[1,10^{-3}]$ we quantify the Wigner crossover using
        two-shell probabilities and inner-ring occupancy histograms,
        shell-mixture reconstructions of $g(r)$ with cosine similarity
        $0.989$–$1.000$ to global data, and sector-resolved bond order and
        Lindemann indices. At $\omega=10^{-3}$ the $N{=}6$ dot forms a
        persistent single-ring Wigner molecule with a clear
        $(1,5)\leftrightarrow(0,6)$ near-degeneracy, while the $N{=}12$ dot
        forms a two-shell molecule with a commensurate $(3,9)$ split as the
        dominant topology. These results provide quantitative, sector-resolved
        benchmarks that complement classical and quantum studies of parabolic
        dots~\cite{Egger_1999,Kong_2002,schweigert1994spectralpropertieschargedparticles,Filinov_2001}.
  \item \textbf{Representation insight.}  
        We show that the correlator features live on a low-dimensional
        manifold (typically one to three principal directions) whose leading
        axes act as collective coordinates for global size and radial
        fluctuations, while finer angular structure is encoded conditionally
        on shell topology. Backflow, by contrast, spans many local modes: its
        entropy rank is higher, linear probes to global observables are
        essentially null, and its energetic effect is concentrated in
        near-field configurations at intermediate traps. Deep in the Wigner
        regime, the optimizer effectively switches backflow off, and the model
        reduces to a low-rank Wigner-molecule ansatz dominated by the
        Slater–Jastrow correlator.
\end{enumerate}

\paragraph{Thesis structure.}
The remainder of the thesis is organised as follows. The next chapter
introduces the theoretical background: the many-body Schrödinger equation in
harmonic traps, basic properties of two-dimensional quantum dots, and the
notion of Wigner molecules. A subsequent chapter reviews the machine-learning
tools used here, with emphasis on neural function approximators,
physics-informed training objectives and neural quantum states. We then
present the concrete Slater–Jastrow(+Backflow) PINN ansatz, the optimization
pipeline (residual-based pretraining followed by stochastic reconfiguration),
and the numerical diagnostics used throughout. The central results chapter
reports energy benchmarks, Wigner-molecule diagnostics and representation
analyses for $N{=}2,6,12$ across trap strengths. Finally, the concluding
chapter summarizes the main findings, discusses limitations, and outlines
possible extensions to larger systems and more general many-body settings.
