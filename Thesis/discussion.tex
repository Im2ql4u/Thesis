\chapter{Quantum Dots}

\section{Discussion of energy results}
\label{sec:energy-discussion}

The energy benchmarks in Tables~\ref{tab:Residual} and~\ref{tab:energies}
show that the Slater--Jastrow PINN, with and without backflow, achieves
DMC-level accuracy for two-electron dots and maintains sub-$0.1\%$ errors
for $N\in\{6,12\}$ across a wide range of trap frequencies. At the same
time, it extrapolates in a controlled way into ultra-weak traps
($\omega=10^{-3}$) where no DMC data are available.

\paragraph{Two electrons: architecture test and SR refinement.}
For $N{=}2$, residual-only training provides a clean test of the bare
architecture. Across $\omega\in\{1.0,0.5,0.1,0.01\}$, both the plain PINN
and PINN+BF track DMC references at the $\sim 10^{-4}$\,Ha level or better
(Table~\ref{tab:Residual}), with deviations fully compatible with quoted
uncertainties except at the shallowest pretraining point $\omega=0.01$.
There, residual-only PINN+BF remains variational but sits
$\sim 3\times 10^{-4}$\,Ha above DMC, indicating a small residual bias
from the PDE-based training alone.

The SR tail in Table~\ref{tab:energies} removes this bias. For
$N{=}2$ and all $\omega\ge 0.01$, the SR-refined PINN+BF energies are
statistically indistinguishable from DMC, with relative deviations
$\lesssim 10^{-3}\,\%$. Thus, once the ansatz is well conditioned, a short
SR phase suffices to close the last few $10^{-4}$\,Ha of variational gap
without changing the qualitative structure learned during residual training.

The comparison between PINN and PINN+BF in Table~\ref{tab:Residual} also
confirms that backflow is \emph{not necessary} to reach DMC accuracy for
two electrons. Energies with and without backflow agree within error bars
at all traps, and the only visible effect of backflow in this case is a
mild variance reduction and smoother optimization in shallow traps. For
$N{=}2$ the nodal structure is adequately captured by the Slater--Jastrow
correlator alone.

\paragraph{Scaling to $N{=}6$ and $N{=}12$.}
The more demanding tests are the many-electron dots in
Table~\ref{tab:energies}. For $N\in\{6,12\}$ and the traps with DMC
references ($\omega\in\{0.1,0.5,1.0\}$), the final PINN+BF energies are
consistently close to DMC and always slightly higher, as required by
variationality. Relative deviations fall in the band
$0.026\text{--}0.048\%$, corresponding to absolute errors of a few
$10^{-4}$\,Ha for the interacting cases, with no sign of catastrophic
growth in error with either particle number or trap strength. The worst
relative deviations occur at intermediate $\omega$ and remain safely below
the percent level.

This is non-trivial for two reasons. First, the grid spans qualitatively
different physical regimes: compact, kinetic-dominated states at
$\omega=1.0$ and swollen, interaction-dominated states at $\omega=0.1$.
Second, the same architecture and training protocol are used across all
$(N,\omega)$, aside from trivial scaling in trap units. The fact that a
single Slater--Jastrow+backflow ansatz maintains sub-$0.1\%$ accuracy
across this grid suggests that the chosen combination of (i) trap-unit
normalization, (ii) soft-core pair features, and (iii) an explicit analytic
cusp is robust with respect to both particle number and confinement.

\paragraph{Ultra-weak traps and extrapolation to $\omega=10^{-3}$.}
At $\omega=10^{-3}$ there are no DMC references in
Table~\ref{tab:energies}, so the PINN+BF values
($E\approx 0.0138$\,Ha for $N{=}2$, $0.1409$\,Ha for $N{=}6$,
$0.5158$\,Ha for $N{=}12$) should be viewed as variational predictions
rather than benchmarked numbers. Several internal checks, however, support
their reliability.

First, the energies connect smoothly to the DMC-validated points at
$\omega=0.01$ under the log--log scaling analysis of $T$, $V_{\rm int}$ and
$V_{\rm trap}$ in Sec.~\ref{sec:wigner-molecule}, with exponents and virial
ratios consistent with classical expectations in the Wigner regime.
Second, for $N{=}2$ at $\omega=10^{-3}$, independent MCMC diagnostics
(stable $r_{12}$ and $P(r)$ histograms, smooth local-energy distributions,
consistent block estimates) indicate that sampling noise is well below the
quoted uncertainties in Table~\ref{tab:energies}. Third, for $N{=}6$ and
$N{=}12$, the structural and energy-based diagnostics in
Sec.~\ref{sec:wigner-molecule} confirm that the states lie on the Wigner
side of the crossover (correct shell topologies, near-classical virial
ratios), with no indication of a mis-set energy scale.

Within these caveats, the $\omega=10^{-3}$ energies can be regarded as
trusted anchors for the Wigner-molecule analysis in
Sec.~\ref{sec:wigner-molecule} and for the representation analysis in
Sec.~\ref{sec:repr-analysis}.

\paragraph{Efficiency and conditioning.}
From an algorithmic point of view, the benchmarks highlight the importance
of conditioning. The SR tail uses only
$\mathcal{O}(10^3)$ iterations with $\sim 3{\times}10^3$ samples each,
i.e.\ a few million local-energy evaluations per $(N,\omega)$ point. This is
modest compared to some earlier neural approaches to quantum dots that
require orders of magnitude more samples to reach similar accuracy.

The key design choices are: (i) working in trap units, which keeps typical
coordinates $\mathcal{O}(1)$ from tight to shallow traps; (ii) soft-core
pair features with $ds/dr\to 0$ at coalescence, which control gradients and
Laplacian spikes near $r_{ij}=0$; and (iii) an explicit analytic cusp that
imposes the correct short-range behaviour analytically rather than learning
it numerically. Together, these keep local energies numerically tame even in
ultra-weak traps where electrons are far apart and the wavefunction varies
on long length scales.

\paragraph{Energetic role of backflow.}
Finally, the tables clarify the energetic role of backflow across regimes.
For $N{=}2$, backflow is not needed to attain DMC-level accuracy. For
$N\in\{6,12\}$ at intermediate traps ($\omega\in\{0.1,0.5\}$), backflow
provides a modest but systematic improvement over residual-only baselines
(not shown in Table~\ref{tab:energies}), bringing final errors into the
$\mathcal{O}(10^{-2}\text{--}10^{-1})\%$ band. In the deepest Wigner cases
($\omega=10^{-3}$), however, the backflow correction becomes energetically
tiny (Sec.~\ref{sec:repr-analysis}): the Slater--Jastrow correlator already
captures the dominant structure and the optimizer effectively \emph{turns
off} backflow. This regime dependence is mirrored in the representation
analysis, where backflow appears as a high-dimensional, predominantly local
correction channel that is only variationally active away from the
ultra-classical limit.


\section{Discussion of the Wigner–molecule regime}
\label{sec:wigner-discussion}

The structural diagnostics in Sec.~\ref{sec:wigner-molecule} show that the
weakest traps considered place the dots firmly on the Wigner side of the
crossover, in a sense that goes beyond the loose label ``strongly
correlated''. Three strands of evidence support this: (i) localization and
energy partitioning, (ii) shell geometry and bond order, and
(iii) topology statistics and reconstruction of $g(r)$.

\paragraph{Localization and energy partitioning.}
On the one-body level, the weak-trap radial laws $P(r)$ are sharply peaked
at large radii with relatively small fluctuations. For the two-electron dot
at $\omega=10^{-3}$ we find
$r_{\rm mode}\approx 64.3$, $\langle r\rangle\approx 64.0$ and
$\gamma=\sigma_r/r_{\rm mode}\approx 0.29$, with only
$\sim 4.6\times 10^{-3}$ of the single-particle mass inside
$r\le 0.25\,r_{\rm mode}$ (Table~\ref{tab:two_e_summary}). The pair
distribution $p(r_{12})$ is peaked at $r_{12}^{\rm mode}\approx 122$ with
Lindemann ratio $\gamma_{r_{12}}\approx 0.20$, and the relative angle is
sharply anti-aligned, $\sigma(\pi-\Delta\phi)\approx 0.35$, with
$p(r_{12}\le 0.25\,r_{12}^{\rm mode})\approx 2.6\times 10^{-3}$. These
numbers are characteristic of a dilute, essentially two-site Wigner dimer
rather than a delocalized Fermi liquid.

For $N\in\{6,12\}$ the same pattern appears in the aggregate. Both
$r_{\rm mode}$ and $r_{\rm rms}$ grow roughly as $\omega^{-0.6}$, the
localization ratio $\gamma$ decreases only mildly with $N$ and $\omega$, and
the extracted $\langle r^2\rangle\propto\omega^{-\beta}$ exponents
$\beta\simeq 1.2\text{--}1.3$ exceed the non-interacting value $\beta=1$.
At the same time, the energy partitioning approaches the classical virial
limit: at $\omega=10^{-3}$,
\[
\frac{2\langle V_{\rm trap}\rangle}{\langle V_{\rm int}\rangle}
\simeq 1.25,\;1.12,\;1.02 \quad\text{for}\quad N=2,6,12,
\]
with the $N{=}12$ dot satisfying $V_{\rm int}\approx 2V_{\rm trap}$ to
within $2\%$ (Fig.~\ref{fig:wigner_ratios}). Combined with the log--log
scaling exponents $\alpha_T\approx 1$ and
$\alpha_{V_{\rm int}}\approx 0.66$, this indicates that the kinetic energy
has become a small correction on top of a nearly classical potential-energy
landscape, as expected in the Wigner-molecule limit.

\paragraph{Shell geometry and bond order.}
Beyond global localization, the weak-trap states exhibit robust shell
structure with non-trivial bond order. For $N{=}6$ the dot retains a single
pronounced shell across all $\omega\in[1,10^{-3}]$; what evolves is the
stiffness of angular correlations. The ring Lindemann index decreases from
$\sim 0.70$ at $\omega=1$ to $\sim 0.64$ at $\omega=10^{-3}$, while the
pair-distance Lindemann ratio drops from $\sim 0.35$ to $\sim 0.27$
(Table~\ref{tab:N6}). After angular registration, the electrons form a
rotating Wigner ring with clear sixfold order.

At $\omega=10^{-3}$, the $N{=}6$ ensemble is dominated by two symmetry
classes, $(1,5)$ and $(0,6)$, with fractions $\approx 0.801$ and
$\approx 0.199$. Both show strong ring order, with
$|\Phi_5|\approx 0.60$, $|\Phi_6|\approx 0.66$ and ring Lindemann
indices $\approx 0.21$. The shell itself never breaks; quantum fluctuations
manifest primarily as rotations and occasional transitions between
$(1,5)$ and $(0,6)$. This matches classical analyses where $(1,5)$ is the
ground-state geometry and $(0,6)$ a proximate competitor.

For $N{=}12$ the structure is richer but equally ordered. At $\omega=10^{-3}$
the two-shell fraction is very high for a broad range of radial-gap
thresholds, confirming that shelling is robust rather than a threshold
artifact. Within the two-shell subset the inner-ring occupancy histogram is
sharply peaked at $(3,9)$, with $(2,10)$ and $(1,11)$ as the leading
subdominant topologies. On $(3,9)$ frames we find
$|\Phi_3|\approx 0.69$, $|\Phi_9|\approx 0.53$ and Lindemann indices
$\approx 0.22$ on both rings, i.e.\ clear threefold and ninefold bond order
with small angular fluctuations. This is precisely the commensurate,
multi-shell crystalline order expected for a twelve-electron Wigner molecule.

\paragraph{Topology statistics and reconstruction}
A more stringent test of the Wigner picture comes from shell-resolved
reconstruction of the pair distribution. For $N{=}12$ at $\omega=10^{-3}$ we
partition two-shell configurations into II/IO/OO pairs, form their individual
histograms and weight them by the observed topology mixture
$(0.045{:}0.409{:}0.545)$. The resulting synthetic $g(r)$ has cosine
similarity $0.9982$ to the global $g(r)$
(Fig.~\ref{fig:N12_w0001_decomposition}), i.e.\ is indistinguishable at our
statistical resolution. The same holds for $N{=}6$, where the global $g(r)$
is reproduced essentially exactly by an OO/IO mixture of $(1,5)$ and $(0,6)$
sectors.

This has two important implications. First, the structure seen in $g(r)$ is
unambiguously geometric (shells and topologies) rather than a subtle
finite-size or sampling artifact: once the shell occupancies and topology
statistics are fixed, the global $g(r)$ is determined. Second, it shows that
the neural ansatz is not merely ``getting the energy right'' but also
reproducing the detailed two-body correlation structure expected from
classical and quantum studies of parabolic Coulomb clusters.

\paragraph{From dimer to multi-shell Wigner molecules.}
Viewed across $N=2,6,12$ and decreasing $\omega$, a coherent Wigner-molecule
picture emerges. The two-electron dot evolves into a dilute, strongly
anti-aligned dimer with negligible overlap near the origin. The six-electron
dot forms a single, stiff Wigner ring whose lab-frame symmetry is restored by
rotation but whose co-rotating frame exhibits clear sixfold order and a
quantum mixture of classically known $(1,5)$ and $(0,6)$ topologies. The
twelve-electron dot, finally, develops a robust two-shell structure with a
commensurate $(3,9)$ split as the modal configuration and topology fractions
that track classical ground-state expectations.

All of this structure is obtained from samples drawn from the same
variational neural ansatz whose energies match or closely track DMC
benchmarks. This gives strong evidence that the states analyzed in
Sec.~\ref{sec:repr-analysis} are bona fide Wigner molecules, and not
artefacts of an over-flexible wavefunction. The next section discusses how
this rich real-space structure is compressed into a remarkably low-dimensional
latent geometry by the learned correlator, and how backflow behaves across
the same regimes.


\section{Discussion of learned representations}
\label{sec:repr-discussion}

The representation analysis in Sec.~\ref{sec:repr-analysis} reveals a
surprisingly simple internal organization of the ansatz: the correlator
$f_{\text{net}}$ lives on a low-dimensional manifold whose leading axes track
global size and fluctuations of the electron cloud, while the BackflowNet
implements a high-dimensional but predominantly local correction that becomes
variationally inert in the deepest Wigner regimes. This mirrors the physical
picture developed in Sec.~\ref{sec:wigner-molecule}.

\paragraph{Low-dimensional correlator manifold.}
Across all $(N,\omega)$, the entropy effective rank $r_{\rm eff}(Z)$ of the
correlator features remains close to~1 and rarely exceeds~2
(Tables~\ref{tab:fnet-geometry-omega-big} and~\ref{tab:fnet-geometry-wigner}).
In the most extreme case, $N{=}2$ at $\omega=10^{-3}$, we find
$r_{\rm eff}(Z)\approx 1.00$ with head--PC1 correlation $\approx 0.998$ and
PC1 block power almost entirely in the pair branch $\psi$. PC ablations show
that projecting onto the top-$k$ PCs rapidly recovers the head, while random
$k$-dimensional subspaces leave most of the error intact: for
$(N,\omega)=(2,10^{-3})$ the relative MAE drops from $\sim 4.5\times10^{-2}$
with $k{=}1$ to $\sim 2.0\times10^{-5}$ with $k{=}12$, whereas random
projections of the same dimension do not achieve comparable reductions.

For $N\in\{6,12\}$ and for both quantum and Wigner regimes, the picture is
similar. Effective ranks cluster between $\sim 1.3$ and $\sim 2.5$, head--PC1
correlations remain $\gtrsim 0.95$, and a small number of principal
directions captures the bulk of sensitivity. From a representation-learning
perspective, the correlator behaves more like a low-dimensional order-parameter
model than like a generic high-dimensional feature extractor.

\paragraph{What the leading axes encode.}
Linear probes from the correlator PCs to coarse physical summaries clarify
what these axes represent. For $N{=}2$ (both at $\omega=0.01$ and
$\omega=10^{-3}$), the first few PCs predict the mean radius and its variance
essentially perfectly ($R^2(r_{\rm mean})\approx 1$,
$R^2(r_{\rm var})\approx 1$), and even near-origin masses such as
$\Pr(r\leq 0.25\,r_{\rm mode})$ are accurately captured. For $N\in\{6,12\}$,
the leading PCs still track global size with $R^2(r_{\rm mean})$ close to
unity and radial fluctuations with $R^2(r_{\rm var})$ in the range
$0.4\text{--}0.85$ across regimes, while shell contrast and angular order
remain only weakly linearly encoded. This is consistent with the structural
analysis: the correlator primarily encodes how big the dot is and how widely
its shells fluctuate, whereas detailed bond order and topology require
explicit angular registration and conditioning on shell occupancy.

\paragraph{Backflow as a local, high-dimensional correction.}
Backflow behaves very differently. For $N{=}6$ and $N{=}12$ the displacement
field $\Delta x$ has $r_{\rm eff}(\Delta x)$ in the range
$\sim 10\text{--}22$ (Table~\ref{tab:bf-geometry-smallomega}), with relatively
flat PCA spectra and gradual reconstruction under PC ablation. Linear probes
from $\Delta x$ PCs to global observables such as $r_{\rm mean}$,
$r_{\rm var}$ or shell contrast yield negligible $R^2$, indicating that the
backflow field does not act as a simple global rescaling or shape parameter.

Near-field diagnostics of $\Delta E$ and $\|\Delta x\|^2$ support a
predominantly local interpretation. At intermediate traps (e.g.\ $N{=}6$ or
$12$ at $\omega=10^{-2}$), the lowest $q\%$ of configurations by $r_{\min}$
carry a disproportionately large fraction of the backflow energy correction:
for $N{=}12$ at $\omega=10^{-2}$ the lowest $5\%$ bin accounts for about six
times its fair share of $\Delta E$ while the corresponding share of
$\|\Delta x\|^2$ remains close to unity
(Table~\ref{tab:bf-energetics-smallomega}). Thus, many spatial modes contribute
to the displacement field, but the \emph{energetic leverage} of backflow is
concentrated in short-distance configurations where small nodal adjustments
have a large impact on the local energy.

\paragraph{Deep Wigner limit: backflow switches off.}
The extreme Wigner points provide a clean limit. For $N{=}2$ at
$\omega=10^{-3}$, the backflow correction to the energy is
$\Delta E\sim 10^{-7}$\,Ha with comparable uncertainty, head--PC1 alignment
between no-BF and BF correlator features is essentially perfect (cosine~$=1$),
and the ranks of $Z$ and $\Delta Z$ are both $\approx 1$. PC ablations of
$\Delta x$ show that even though $r_{\rm eff}(\Delta x)$ can be nominally high,
the field is effectively constant over the sampled manifold (PC ablation error
$\sim 10^{-6}$), and near-field shares of both $\Delta E$ and
$\|\Delta x\|^2$ are $\approx 1$. In this regime, the optimizer has
effectively \emph{learned away} backflow: the Slater--Jastrow correlator alone
is sufficient to represent the Wigner dimer.

For $N{=}6$ and $N{=}12$ at $\omega=10^{-3}$ the pattern is similar, albeit
less extreme. The correlator manifold remains low-rank with PC1 almost
unchanged between no-BF and BF models, $r_{\rm eff}(\Delta Z)$ stays close to
$r_{\rm eff}(Z)$, and the energetic effect of backflow is very small. Near-field
shares of $\Delta E$ revert to values close to one. Once the system is deep in
the Wigner-molecule regime and the static correlator has adjusted to the
interaction-dominated landscape, there is simply little left for backflow to
correct.

\paragraph{Regime dependence and architectural implications.}
Taken together, the representation results show a clear regime dependence:
\begin{itemize}
  \item In \emph{weakly to moderately correlated} traps (larger $\omega$,
        smaller $r_s$), the correlator remains low-rank but backflow carries
        a non-trivial fraction of the energy correction, focused on
        near-field configurations. Here it is beneficial to allocate
        capacity to a reasonably wide BackflowNet with rich pairwise inputs,
        while keeping the correlator compact.
  \item In the \emph{deep Wigner} regime (ultra-weak traps such as
        $\omega=10^{-3}$), the correlator alone captures both the global
        size/shelling and the relevant near-field structure; backflow
        becomes nearly variationally idle. Additional expressivity in
        backflow does not translate into better energies or qualitatively
        different correlations.
\end{itemize}
From a design perspective, this suggests that Slater--Jastrow PINNs with a
small, smooth correlator and moderately wide backflow layers are a natural
default for parabolic dots. The correlator behaves like a learned set of
collective coordinates summarizing dot size and shell fluctuations, while
backflow functions as a flexible local correction channel that is activated
only where near-field physics is delicate. In the Wigner limit, the model
naturally reduces to a static Wigner-molecule ansatz with low-rank latent
geometry.

\paragraph{Connection to the physical picture.}
Finally, the learned representations dovetail with the structural analysis in
Sec.~\ref{sec:wigner-molecule}. The same correlator that reproduces shell
topologies, bond order and virial partitioning in the Wigner regime is found
to live on a one- or two-dimensional manifold whose leading axis correlates
almost perfectly with global size and radial spread. Backflow, in turn, is
high-dimensional and local exactly in the regimes where shell structure is
less rigid and close encounters are more frequent. This supports a physically
meaningful decomposition: a small set of global ``collective coordinates''
captured by $f_{\text{net}}$ and a many-body, predominantly local correction
channel implemented by BackflowNet. The success of this decomposition in
capturing both energies and detailed Wigner-molecule structure suggests that
similar architectures may be effective for other strongly correlated finite
fermion systems.
