\graphicspath{{../results/figures/results/}}

\chapter{Quantum Dots}

We study two-dimensional harmonic quantum dots, a numerically delicate setting due to the Coulomb singularity at short range and the second-order derivatives from the kinetic energy. Both effects can yield ill-conditioned objectives that typically require many epochs and small learning rates. We consider two-electron, six-electron and twelve-electron dots across a wide confinement range ($\omega\!\in\!\{1.0,\,0.5,\,0.1,\,0.01,\, 0.001\}$). 
When $\omega$ becomes sufficiently small, the system approaches the Wigner molecule limit, where the inter-electron distance becomes large compared to the confinement length scale. In this regime, the electrons localize into geometric arrangements that minimize their mutual repulsion while balancing the trap potential.
As $\omega$ decreases, the well broadens and correlations strengthen; nevertheless, the Slater\,+\,smooth-$F_\theta$\,+\,analytic-cusp ansatz remains stable and accurate in practice.
In addition, we provide a detailed analysis of the learned representations, focusing on the correlator manifold and backflow field.

\section{Training protocol (summary).}
Unless otherwise stated, we use residual-based pretraining to obtain a well-conditioned initializer, followed by a Stochastic Reconfiguration (SR) tail to tighten the last digits of the energy. The SR phase uses $\mathcal{O}(10^3\!-\!10^3.5)$ iterations with batches of $\sim\!3{\times}10^3$ samples per iteration.
\subsection{Energies}

Table~\ref{tab:Residual} reports two-electron energies from \emph{residual-only}
training (PINN and PINN+BF). For $N{=}2$, backflow is not required to reach DMC
accuracy, but it tends to reduce variance and smooth convergence.

\begin{table}[h!]
\centering
\caption{Ground-state energies (Hartree) for two electrons in a harmonic trap
using \emph{residual-only} training. DMC references as in [Ref.]. Parentheses
denote $1\sigma$ on the last digits.}
\label{tab:Residual}
\begin{tabular}{c c c c}
\hline
$\omega$ & PINN+BF & PINN & DMC (Ref.) \\
\hline
1.00  & 2.99998(3)   & 2.999940(5)  & 3.00000(1) \\
0.50  & 1.65975(2)   & 1.65979(3)   & 1.65977(1) \\
0.10  & 0.440795(3)  & 0.440833(1)  & 0.44079(1) \\
0.01  & 0.0740724(6) & 0.0741679(6) & 0.073839(2) \\
\hline
\end{tabular}
\end{table}

Table~\ref{tab:energies} compares diffusion Monte Carlo (DMC) references with
our \emph{final} PINN+BF energies after residual pretraining and an SR tail.
From the ultra–shallow two-electron case ($\omega{=}0.01$) to the many-electron
systems, absolute deviations are small; for $N{=}2$ they lie within DMC
uncertainties, and for $N\!\in\!\{6,12\}$ the relative deviations are
$\mathcal{O}(10^{-2}\,\%)$–$\mathcal{O}(10^{-1}\,\%)$ (see also
Fig.~\ref{fig:rel_error_vs_omega}).

\begin{table}[htbp]
  \centering
  \caption{Residual-based pretraining + SR tail. Reference DMC and our PINN+BF
  ground-state energies (Hartree) for selected $(N,\omega)$. Parentheses denote
  $1\sigma$ on the last digit(s). The final column is the relative deviation
  from DMC (\%).}
  \label{tab:energies}
  \begin{threeparttable}
    \begin{tabular}{c|c|ccc}
      \toprule
      $N$ & $\omega$ & Reference (DMC) & PINN+BF & \%\,err \\
      \midrule
      2 & 0.001 & --- & $0.0137948(8)$ & --- \\
        & 0.01 & $0.073839(2)$ & $0.073838(1)$ & $-0.0014$ \\
        & 0.10 & $0.44079(1)$  & $0.44079(1)$  & $+0.0000$ \\
        & 0.50 & $1.65977(1)$  & $1.65976(2)$  & $-0.0006$ \\
        & 1.00 & $3.00000$     & $2.99999(1)$  & $-0.0003$ \\
      \midrule
      6 & 0.001 & ---           & $0.140913(1)$ & --- \\
        & 0.01 & ---            & $0.69125(2)$  & --- \\
        & 0.10 & $3.55385(5)$   & $3.5549(1)$   & $+0.0295$ \\
        & 0.50 & $11.78484(6)$  & $11.7895(4)$  & $+0.0395$ \\
        & 1.00 & $20.15932(8)$  & $20.1610(6)$  & $+0.0083$ \\
      \midrule
      12 & 0.001 & ---          & $0.515823(3)$ & --- \\
         & 0.01 & ---           & $2.48620(5)$  & --- \\
         & 0.10 & $12.26984(8)$ & $12.2731(2)$  & $+0.0266$ \\
         & 0.50 & $39.1596(1)$  & $39.1786(8)$  & $+0.0485$ \\
         & 1.00 & $65.7001(1)$  & $65.717(1)$   & $+0.0257$ \\
      \bottomrule
    \end{tabular}
  \end{threeparttable}
\end{table}

\begin{figure}[htbp]
  \centering
  \includegraphics[width=0.8\textwidth]{rel_error_vs_omega_allN}
  \caption{Relative deviation of PINN+BF ground-state energies from DMC
  references as a function of trap frequency $\omega$ for the cases where
  DMC data are available. For $N{=}2$ the deviations are at or below the DMC
  error bars ($\sim 10^{-3}\,\%$), while for $N{=}6$ and $N{=}12$ the errors
  remain in the range $2.5\times 10^{-2}$–$5\times 10^{-2}\,\%$ across
  $\omega$, with no sign of deterioration in the shallow- or tight-trap
  limits.}
  \label{fig:rel_error_vs_omega}
\end{figure}

\paragraph{Summary.}
A compact Slater–Jastrow PINN with soft-core pair features and an explicit
analytic cusp achieves DMC-level accuracy for $N{=}2$ across $\omega$, and for
$N\!\in\!\{6,12\}$ reaches small, systematic relative deviations (roughly
$0.01$–$0.08\%$ across the grid; Fig.~\ref{fig:rel_error_vs_omega}). In terms
of sample efficiency, the SR tail uses on the order of $10^3$ iterations with
$\sim\!3{\times}10^3$ samples each; this is \emph{orders of magnitude fewer}
training samples than some earlier neural approaches to quantum dots (see
\S\ref{sec:related} for a quantitative comparison). We attribute the stability
to conditioning choices (trap-unit scaling, soft-core features with
$ds/dr\!\to\!0$ at coalescence, and an explicit cusp), which keep both
gradients and Laplacians numerically tame from shallow to tight traps.
Convergence traces and ablations (activation, initialization, SR settings,
backflow on/off) are reported in Appendix~X.

\section{What the networks learn}
\label{sec:repr-analysis}

We analyze the learned \emph{correlator} $f_{\text{net}}$ (the residual head fed by
$\overline{\phi}\,\|\,\overline{\psi}\,\|\,\mathbf g$) and the \emph{BackflowNet} by probing their
representations on $|\Psi|^2$ samples. The goal is not to reproduce every diagnostic, but to
identify a few robust patterns:
(i) the intrinsic dimensionality of the correlator manifold;
(ii) how strongly the head aligns with its principal axes and how those axes map to physical
observables; and
(iii) how much extra structure backflow actually adds, especially in the Wigner regime.

\subsection{A low-dimensional correlator manifold}

Across all systems, including the deepest Wigner cases, the correlator $Z$ remains
strikingly low-dimensional. We measure an entropy effective rank
$r_{\rm eff}(Z)=\exp(-\sum_i p_i\log p_i)$ from the spectrum of singular values
and the cosine $\mathrm{corr}(\text{head},\text{head@PC1})$ between the original
head readout and its projection onto the first principal component (PC1).

\begin{table}[h]
\centering
\caption{\textbf{Correlator $Z$ geometry in representative regimes.}
Entropy effective rank $r_{\rm eff}(Z)$, alignment of the head with PC1, and PC1
block power across branches. All entries are for the no-backflow network.}
\label{tab:fnet-geometry-short}
\small
\setlength{\tabcolsep}{6.5pt}
\begin{tabular}{ccccccc}
\toprule
$N$ & $\omega$ & $r_{\rm eff}(Z)$ & corr(head,PC1) & $\phi$ & $\psi$ & extras \\
\midrule
2  & $10^{-3}$ & 1.00 & 0.998 & 0.000 & 1.000 & 0.000 \\
2  & $10^{-2}$ & 1.00 & 0.998 & 0.000 & 0.999 & 0.001 \\
\midrule
6  & $10^{-3}$ & 1.54 & 0.968 & 0.009 & 0.128 & 0.863 \\
6  & $10^{-2}$ & 1.25 & 0.987 & 0.060 & 0.470 & 0.469 \\
\midrule
12 & $10^{-3}$ & 1.55 & 0.967 & 0.522 & 0.124 & 0.354 \\
12 & $10^{-2}$ & 1.86 & 0.956 & 0.531 & 0.060 & 0.409 \\
\bottomrule
\end{tabular}
\end{table}

For the two-electron dot at $\omega=10^{-3}$ the correlator is essentially
one-dimensional: $r_{\rm eff}(Z)\approx 1.00$, the head--PC1 correlation is
$0.9976$, and PC1 block power is almost purely in the pair branch
($\psi\approx 0.99998$, $\phi,\text{extras}\approx 0$).
The slightly tighter 2e case at $\omega=10^{-2}$ is similar.
This shows that for 2e the network has effectively discovered a \emph{single
pair coordinate} that controls the energy.

In the strongly correlated 6e and 12e Wigner cases ($\omega=10^{-3}$),
the correlator spectrum still has $r_{\rm eff}(Z)\simeq 1.5$ and the head
is almost one-dimensional: $\mathrm{corr}(\text{head},\text{head@PC1})\approx
0.97$ in both systems.
The PC1 block power shows a clear division of labour:
for $N{=}6$ PC1 lives almost entirely in the ``extras'' branch (0.86),
while for $N{=}12$ it is dominated by the global/per-particle branch
$\phi$ (0.52--0.53). In other words, the large dot is summarized by a
single global axis, while the medium dot prefers a more pair-like/global
mixture.

PC ablations confirm that a handful of PCs capture almost all of the head.
For the deepest Wigner cases we obtain, for example:
\begin{itemize}
  \item $(N,\omega)=(2,10^{-3})$:
    rel-MAE with only PC1 is $0.045$, which drops to $0.0068$ with $k{=}4$
    PCs and to $2.0\times 10^{-5}$ with $k{=}12$;
  \item $(N,\omega)=(6,10^{-3})$:
    $0.20 \to 0.18 \to 0.18 \to 4.1\times 10^{-3}$ for $k{=}1,2,4,8$;
  \item $(N,\omega)=(12,10^{-3})$:
    $0.18 \to 0.16 \to 0.028 \to 5.6\times 10^{-4}$ for $k{=}1,2,4,8$.
\end{itemize}
Thus, despite the complicated Wigner-molecule structure in real space, the
correlator lives on a very low-dimensional manifold, from 2e up to 12e.

\subsection{Head PCs track global size and fluctuations}

We probe how these latent directions relate to coarse observables by regressing
simple linear probes from the head PCs to global size and radial variance.
For the systems that are deep in the Wigner regime we obtain
\begin{align*}
(N,\omega)=(2,10^{-3}) :\quad
  & R^2(r_{\rm mean}) \approx 0.9999,\quad R^2(r_{\rm var}) = 1.0,\\
  & R^2(\Pr(r<0.25)) = 1.0,\quad R^2(\text{shell\_contrast}) \approx 0.02,\\[2pt]
(N,\omega)=(6,10^{-3}) :\quad 
  & R^2(r_{\rm mean}) \approx 0.99,\quad R^2(r_{\rm var}) \approx 0.64,\\[2pt]
(N,\omega)=(12,10^{-3}) :\quad
  & R^2(r_{\rm mean}) \approx 0.996,\quad R^2(r_{\rm var}) \approx 0.83.
\end{align*}
At 2e, a single pair-dominated axis almost perfectly parameterizes the
entire radial distribution (mean, variance, and even the small-$r$ tail
through $\Pr(r<0.25)$), while at 6e and 12e the same axis (plus one or two
more PCs) captures global size and fluctuations across the Wigner crossover.
Shell contrast remains only weakly linearly encoded ($R^2\sim 0.02$--0.03),
consistent with the need for angular registration to resolve crystalline
order.

\subsection{Backflow adds very little in the deep Wigner regime}

We next compare the no-backflow and backflow networks on matched
configuration sets. For each system we report the mean energy difference
$\Delta E = E_{\rm BF} - E_{\rm noBF}$ on BF samples, its standard error,
and how the correlator manifold changes.

For the weakest traps we find:
\begin{align*}
(N,\omega)=(2,10^{-3}):\quad &
  E_{\rm noBF} \approx 0.0138197 \pm 6.7\times10^{-6}\ {\rm Ha},\\[-2pt]
  & E_{\rm BF} \approx 0.0137998 \pm 5.6\times10^{-6}\ {\rm Ha},\\
  & \Delta E \approx 1.8\times10^{-7} \pm 1.1\times10^{-7}\ {\rm Ha},\\[4pt]
(N,\omega)=(6,10^{-3}):\quad &
  \Delta E \approx 2\times 10^{-6} \pm 10^{-6}\ {\rm Ha},\\[2pt]
(N,\omega)=(12,10^{-3}):\quad &
  \Delta E \approx -4\times 10^{-6} \pm 10^{-6}\ {\rm Ha},
\end{align*}
i.e.\ the backflow correction is at the level of a few micro-Hartree and
statistically compatible with zero in all three cases.

Geometrically, the leading correlator axis is essentially unchanged.
For $(N,\omega)=(2,10^{-3})$ the PC1 alignment between no-BF and BF is
exactly $1.0$ (to numerical precision) and $r_{\rm eff}(\Delta Z)\approx
1.00$. For $(6,10^{-3})$ the PC1 cosine is $0.999977$ with
$r_{\rm eff}(\Delta Z)\approx 1.54$, and for $(12,10^{-3})$ the values
are $0.999971$ and $\approx 1.55$. Backflow therefore behaves as a very
small, low-rank reweighting of the existing correlator manifold when the
system is deep in the Wigner regime, consistent with the energy analysis
in Sec.~\ref{sec:wigner-molecule} that shows the state already close to
the classical virial balance.

At slightly stronger confinement, backflow is more active:
for $(N,\omega)=(6,10^{-2})$ we find
$\Delta E \approx 5.4\times 10^{-4} \pm 1.0\times 10^{-4}$\,Ha with
PC1 cosine $0.999891$ and $r_{\rm eff}(\Delta Z)\approx 1.25$.
For $(12,10^{-2})$ the energy difference is more noisy
($\Delta E \approx 3.0\times 10^{-3} \pm 4.3\times 10^{-3}$\,Ha),
but the manifold picture is similar: PC1 cosine $0.999989$ and
$r_{\rm eff}(\Delta Z)\approx 1.85$. In this intermediate regime the
correlator still changes only along $1$–$2$ directions, but the
correction is large enough to matter energetically.

Near-field conditioning of the energy change shows the same pattern.
For $(N,\omega)=(2,10^{-3})$ the lowest $1\%$, $5\%$, and $10\%$ of
$r_{\min}$ configurations account for $\sim1.01$, $1.04$, and $1.03$
times their fair share of $\Delta E$, i.e.\ essentially uniform.
For $(6,10^{-3})$ and $(12,10^{-3})$ the shares are similarly close
to unity. In contrast, for $(N,\omega)=(12,10^{-2})$ the lowest $5\%$
and $10\%$ bins carry ${\sim}6\times$ and ${\sim}4.7\times$ their
fair share of $\Delta E$, indicating a strong near-field focus at
intermediate confinement.

\subsection{BackflowNet: high-dimensional and local}

While the correlator $Z$ is low-rank, the backflow vector field
$\Delta x$ is genuinely higher-dimensional once the system has more
than two electrons. Entropy ranks for representative systems are
\[
r_{\rm eff}(\Delta x) \approx
\begin{cases}
3.62 & (N,\omega)=(2,10^{-3}),\\
10.7 & (N,\omega)=(6,10^{-3}),\\
10.1 & (N,\omega)=(6,10^{-2}),\\
22.0 & (N,\omega)=(12,10^{-3}),\\
12.1 & (N,\omega)=(12,10^{-2}),
\end{cases}
\]
with relatively flat PCA spectra and gradual error decay as more PCs are
retained for $N\ge 6$. In contrast to $Z$, linear probes from $\Delta x$
PCs to global observables are essentially uninformative: typical values are
$R^2(r_{\rm mean}),R^2(r_{\rm var}) \ll 0.05$, and in some extreme Wigner
cases the probes are numerically ill-conditioned. This supports the view
that $\Delta x$ implements \emph{local nodal adjustments} rather than
global rescalings.

The two-electron Wigner limit at $\omega=10^{-3}$ provides an extreme
illustration. Here the backflow network has effectively collapsed to a
trivial transformation: the displacement spectrum has
$r_{\rm eff}(\Delta x)\approx 3.6$ but PC ablations show zero relative
error for any $k$ (the field is effectively constant on the sampled
manifold), all input-channel ablations (including $r_{ij}^2$) produce
no change, and the near-field shares of $\|\Delta x\|^2$ are exactly
unity. Combined with the micro-Hartree-level $\Delta E$, this confirms
that the optimizer has learned to \emph{not use} backflow at all when
a simple pair-correlator already suffices.

For $N{=}6$ and $N{=}12$ at intermediate confinement, backflow behaves
quite differently: $r_{\rm eff}(\Delta x)$ is large (10–20), PC ablations
show progressive error reduction as more PCs are retained, and near-field
$\Delta E$ shares indicate that corrections are concentrated on small
$|r_{ij}|$ configurations. Input-channel ablations in these cases show
that the scalar distance channel $r_{ij}^2$ is the dominant driver of
$\Delta x$, with absolute coordinates playing a negligible direct role.

\paragraph{Design takeaway.}
Overall, the network organizes the problem into a clean separation of roles:
a low-dimensional correlator manifold $Z$ that tracks global size and radial
fluctuations (and is already nearly optimal in the Wigner regime),
and a high-dimensional backflow field $\Delta x$ that makes small, local,
distance-based corrections when the system is not yet fully classical.
In the two-electron Wigner limit, backflow becomes variationally inert,
and a single pair-dominated latent axis essentially parameterizes the
entire state.


\section{Wigner–molecule crossover}
\label{sec:wigner-molecule}


\begin{figure*}[htbp]
  \centering
  \captionsetup[subfigure]{justification=centering,font=small}

  % ---------- Row 1: ω = 1.0 ----------
  \begin{subfigure}[t]{\threecolw}
    \centering
    \includegraphics[width=\linewidth]{one_body_density_2_omega_1.00000_20251029_231309.pdf}
    \caption{$N{=}2,\;\omega=1.0$}
  \end{subfigure}\hspace{\figgutter}%
  \begin{subfigure}[t]{\threecolw}
    \centering
    \includegraphics[width=\linewidth]{one_body_density_6_omega_1.00000_20251030_093439.pdf}
    \caption{$N{=}6,\;\omega=1.0$}
  \end{subfigure}\hspace{\figgutter}%
  \begin{subfigure}[t]{\threecolw}
    \centering
    \includegraphics[width=\linewidth]{one_body_density_12_omega_1.00000_20251104_194000.pdf}
    \caption{$N{=}12,\;\omega=1.0$}
  \end{subfigure}

  \vspace{0.5em}

  % ---------- Row 2: ω = 0.001 ----------
  \begin{subfigure}[t]{\threecolw}
    \centering
    \includegraphics[width=\linewidth]{one_body_density_2_omega_0.00100_20251106_160757.pdf}
    \caption{$N{=}2,\;\omega=0.001$}
  \end{subfigure}\hspace{\figgutter}%
  \begin{subfigure}[t]{\threecolw}
    \centering
    \includegraphics[width=\linewidth]{one_body_density_6_omega_0.00100_20251106_214930.pdf}
    \caption{$N{=}6,\;\omega=0.001$}
  \end{subfigure}\hspace{\figgutter}%
  \begin{subfigure}[t]{\threecolw}
    \centering
    \includegraphics[width=\linewidth]{one_body_density_12_omega_0.00100_20251109_110913.pdf}
    \caption{$N{=}12,\;\omega=0.001$}
  \end{subfigure}

  \caption{One-body densities for 2, 6 and 12 particles for confiments $\omega=1.0$ (top row) and $\omega=0.001$ (bottom row). As the confinement weakens, the densities broaden and develop pronounced shell structures characteristic of Wigner molecules.}
  \label{fig:onebody_grid}
\end{figure*}


\subsection{What we measure}
For each $(N,\omega)$ we draw $\mathbf X\!\sim|\Psi|^2$ and report
(i) the pair distribution $g(r)$,
(ii) the single-particle radial law $P(r)\!\propto\!r^{d-1}g(r)$ ($d=2$),
and scalars from $P(r)$: $r_{\rm mode}$, $\langle r\rangle$, $\sigma_r$, FWHM, $q_{10},q_{50},q_{90}$.
We condense localization via
\[
\gamma \;=\; \sigma_r/r_{\rm mode},
\]
(smaller $\gamma$ $\Rightarrow$ stiffer localization).
For $N>2$ we also monitor ring bond–orientational order $|\Phi_m|$ on resolved shells 
and a dimensionless Lindemann-type ratio for nearest-neighbour angular spacings 
(see also bond-orientational diagnostics in classical Wigner-like systems~\cite{Mazars_2008}).

\paragraph{Density parameter $r_s$.}
Following Egger \textit{et al.}~\cite{Egger_1999}, we estimate $r_s$ from the first maximum $r^\ast$ of the (spin-summed) pair correlation,
\(
r_s \equiv r^\ast/a_B^\ast.
\)
The Fermi–liquid $\to$ Wigner–molecule crossover occurs already near $r_s\!\simeq\!4$ in parabolic dots; 
our weak–trap cases lie well beyond this threshold and should display clear Wigner signatures~\cite{Egger_1999,Filinov_2001}.

\begin{figure*}[htbp]
  \centering
  \captionsetup[subfigure]{justification=centering}

  \begin{subfigure}[t]{0.48\textwidth}
    \includegraphics[width=\linewidth]{ratio_T_over_Vint_desc_allN}
    \caption{$T/V_{\rm int}$ vs.\ $\omega$ for $N{=}2,6,12$. The horizontal line marks $T/V_{\rm int}{=}0.1$.}
  \end{subfigure}
  \hfill
  \begin{subfigure}[t]{0.48\textwidth}
    \includegraphics[width=\linewidth]{ratio_2Vtrap_over_Vint_desc_allN}
    \caption{$2V_{\rm trap}/V_{\rm int}$ vs.\ $\omega$ for $N{=}2,6,12$. The horizontal line marks the classical virial ratio $2V_{\rm trap}=V_{\rm int}$.}
  \end{subfigure}

  \caption{Energy-based Wigner diagnostics as functions of trap frequency.  
  As $\omega$ decreases the dots become interaction dominated ($T/V_{\rm int}\!\ll1$) and the trap--interaction partition approaches the classical virial form, most clearly for $N{=}12$.}
  \label{fig:wigner_ratios}
\end{figure*}

\paragraph{Energy-based crossover diagnostics.}
As a complementary probe we analyze the expectation values of the kinetic energy $T$,
the Coulomb energy $V_{\rm int}$, and the trap energy $V_{\rm trap}$ as functions of $\omega$.
Log–log fits of the form $E(\omega)\propto\omega^{\alpha}$ on the grid
$\omega\in\{10^{-3},10^{-2},10^{-1},0.5,1\}$ give
\[
\alpha_T \simeq 1.00,\qquad
\alpha_{V_{\rm int}} \simeq 0.66,\qquad
\alpha_{V_{\rm trap}} \simeq 0.81,\,0.73,\,0.72
\]
for $N=2,6,12$, respectively.  Thus $T$ scales essentially linearly with $\omega$, while both
$V_{\rm int}$ and $V_{\rm trap}$ grow sublinearly with exponents that are nearly $N$-independent.
The ratios
\[
\frac{T}{V_{\rm int}} \;\propto\; \omega^{\alpha_T-\alpha_{V_{\rm int}}}
\quad\text{and}\quad
\frac{T}{V_{\rm trap}} \;\propto\; \omega^{\alpha_T-\alpha_{V_{\rm trap}}}
\]
therefore decrease as power laws with exponents $\alpha_T-\alpha_{V_{\rm int}}\!\approx\!0.32\text{--}0.35$
and $\alpha_T-\alpha_{V_{\rm trap}}\!\approx\!0.19\text{--}0.28$.
Across $\omega\in[1,10^{-3}]$ the interaction-to-kinetic ratio
$\Gamma= \langle V_{\rm int}\rangle/\langle T\rangle$ increases from
$\Gamma\simeq0.9\!\to\!8.7$ ($N=2$), $2.4\!\to\!25.4$ ($N=6$), and $3.6\!\to\!39.9$ ($N=12$),
while $T/V_{\rm int}$ falls correspondingly (Fig.~\ref{fig:wigner_ratios}a), e.g.\
$T/V_{\rm int}$ decreases from $\approx 0.28$ to $\approx 0.025$ for $N=12$.

The partitioning between trap and interaction energies becomes quasi-classical at weak confinement.
At $\omega=10^{-3}$ we find
\[
\frac{2\langle V_{\rm trap}\rangle}{\langle V_{\rm int}\rangle}
\simeq 1.25,\;1.12,\;1.02 \quad\text{for}\quad N=2,6,12,
\]
so that the $N=12$ dot satisfies $V_{\rm int}\!\approx\!2V_{\rm trap}$ to within $2\%$,
as expected for nearly classical charges in a harmonic trap (Fig.~\ref{fig:wigner_ratios}b).
Using $V_{\rm trap} = \tfrac12 \omega^2 \sum_i \langle r_i^2\rangle$ we can extract
a root-mean-square radius
$r_{\rm rms} = \sqrt{2V_{\rm trap}/(N\omega^2)}$.
For $(N,\omega)=(2,6,12;1)$ we obtain
$r_{\rm rms}\!\approx\!1.14,1.62,2.03\,a_B^\ast$, which expands to
$\approx 70,128,169\,a_B^\ast$ at $\omega=10^{-3}$.
Fitting $r_{\rm rms}\propto\omega^{\alpha_r}$ yields
$\alpha_r\simeq-0.60,-0.63,-0.64$ for $N=2,6,12$,
i.e.\ $\langle r^2\rangle\propto\omega^{-\beta}$ with
$\beta\simeq1.19,1.27,1.28$.
The cloud thus expands \emph{faster} than the non-interacting
$\omega^{-1}$ expectation, in line with interaction-driven swelling and
shell formation in the Wigner regime.

% ------------------------- N=2 -------------------------
\subsection{Two electrons ($N=2$): monotone approach to the Wigner side}
As $\omega$ decreases, $r_{\rm mode}$ grows and $\gamma=\sigma_r/r_{\rm mode}$ shrinks (Table~\ref{tab:two_e_summary}),
with a power-law $r_{\rm mode}\!\sim C\,\omega^{\alpha}$ giving $\alpha\simeq-0.62$ over $\omega\in\{1,0.5,0.1,0.01\}$,
close to the classical two-body scaling $-2/3$.  

At the weakest trap, $\omega=10^{-3}$, the radial diagnostics read
$r_{\rm mode}\approx64.3$, $\langle r\rangle\approx64.0$,
$\sigma_r\approx18.5$, FWHM$\approx44.0$ and
$\gamma\approx0.29$, with only
$\sim 4.6\times 10^{-3}$ of the single-particle mass inside
$r\le 0.25\,r_{\rm mode}$.
The pair distance distribution has mode $r_{12}\approx122$ and
Lindemann $\gamma_{r_{12}}\approx0.20$, while the relative angle
is sharply peaked near $\pi$ with
$\sigma(\pi-\Delta\phi)\approx0.35$ and a tiny near-origin mass
$p(r_{12}\le0.25\,r_{12}^{\rm mode})\approx2.6\times 10^{-3}$.
Together these show a very dilute, strongly anti-aligned Wigner dimer.

\begin{figure}[htp]
  \centering
  \includegraphics[width=0.8\textwidth]{N2_all_densities}
  \caption{Two-electron radial densities $P(r)$ across $\omega$.}
  \label{fig:2N_radial_densities}
\end{figure}

\subsection{Interacting Two-Dimensional Quantum Dots}

\begin{table}[H]
\centering
\caption{$N=2$ summary across $\omega$ (Bohr).}
\label{tab:two_e_summary}
\begin{tabular}{lcccccccc}
\toprule
$\omega$ 
& $r_{\rm mode}$ 
& $\langle r\rangle$ 
& $\sigma_r$ 
& FWHM 
& $q_{10}$ 
& $q_{50}$ 
& $q_{90}$ 
& $\gamma$ \\
\midrule
1.00 & 1.435 & 1.632 & 0.705 & 1.743 & 0.735 & 1.581 & 2.581 & 0.491 \\
0.50 & 2.303 & 2.475 & 1.007 & 2.466 & 1.209 & 2.409 & 3.829 & 0.437 \\
0.10 & 3.601 & 3.830 & 1.728 & 4.449 & 1.640 & 3.714 & 6.127 & 0.480 \\
0.01 & 14.191 & 14.593 & 5.615 & 13.602 & 7.390 & 14.322 & 22.038 & 0.396 \\
0.001 & 64.292 & 63.953 & 18.479 & 43.977 & 40.392 & 63.814 & 87.714 & 0.287 \\
\bottomrule
\end{tabular}
\end{table}

\paragraph{Takeaway.}
All observables—growth of $r_{\rm mode}$, shrinking $\gamma$, sharper anti-alignment,
and suppressed near-origin mass—evolve smoothly with decreasing $\omega$,
consistent with the expected two-site Wigner dimer limit.

% ------------------------- N=6 -------------------------
\subsection{Six electrons ($N=6$): one ring persists, correlations stiffen}
Across $\omega\in[1,10^{-3}]$, $N=6$ retains a single pronounced shell; 
localization ratios and Lindemann indices decrease steadily (Table~\ref{tab:N6}).  
This is precisely the ``rotating Wigner molecule'' (RWM) scenario—lab-frame densities are ring-like, 
while crystalline order is recovered after angular registration~\cite{Egger_1999,manninen2007metalclustersquantumdots}.

\begin{table}[H]
\centering
\caption{$N=6$ diagnostics vs.\ $\omega$ (Bohr; angles in radians).}
\label{tab:N6}
\begin{tabular}{c|ccccc}
\toprule
$\omega$ & $r_{\rm mode}$ & $\sigma_r$ & $\gamma$ & $\sigma(\text{ring angles})$ & $\gamma_{r_{ij}}$ \\
\midrule
1.00  & 2.0907 & 0.9249  & 0.4424 & 0.7046 & 0.3538 \\
0.50  & 3.0702 & 1.3648  & 0.4445 & 0.6943 & 0.3397 \\
0.10  & 8.1686 & 3.5025  & 0.4288 & 0.6842 & 0.3149 \\
0.01  & 33.1538 & 13.9094 & 0.4195 & 0.6497 & 0.2889 \\
0.001 & 113.5542 & 47.8076 & 0.4210 & 0.6368 & 0.2664 \\
\bottomrule
\end{tabular}
\end{table}

\paragraph{Weak confinement at $\omega=0.01$.}
A near-degeneracy between the $(1,5)$ and $(0,6)$ sectors is observed, with
\[
(1,5)\ \text{fraction}\approx 0.692,\qquad (0,6)\approx 0.308,
\]
and phase–aligned ring order
\[
|\Phi_5|=0.453\!\pm\!0.214,\ \ \text{Lind}=0.329;\qquad
|\Phi_6|=0.466\!\pm\!0.206,\ \ \text{Lind}=0.315.
\]
The global $g(r)$ is exactly reproduced by the expected OO/IO mixture (cosine similarity~$=1.000$).

\begin{figure}[htp]
  \centering
  \includegraphics[width=0.8\textwidth]{N6_all_densities}
  \caption{Six-electron radial densities $P(r)$ across $\omega$.}
  \label{fig:6N_radial_densities}
\end{figure}

\paragraph{Pushing to $\omega=10^{-3}$.}
The ring stiffens further and both symmetry classes persist with high-quality order.
From $K=300{,}000$ frames we obtain
\[
(1,5)\ \text{fraction}\approx 0.801,\qquad (0,6)\approx 0.199,
\]
with phase-aligned ring order
\[
|\Phi_5| = 0.597\!\pm\!0.224,\ \ \text{Lind}\approx0.225;\qquad
|\Phi_6| = 0.663\!\pm\!0.183,\ \ \text{Lind}\approx0.205.
\]
Thus the shell is always present; lab-frame symmetry is restored by rotation,
but in the co-rotating frame the six electrons form a stiff Wigner ring,
predominantly in the $(1,5)$ sector with a significant $(0,6)$ minority.
Classically, $(1,5)$ is favored and $(0,6)$ a proximate competitor; the quantum
mixture at large $r_s$ mirrors this~\cite{schweigert1994spectralpropertieschargedparticles,Kong_2002}.
(In both weak-trap cases the global $g(r)$ is reproduced to numerical precision
by the appropriate II/IO/OO mixtures.)

% ------------------------- N=12 -------------------------
\subsection{Twelve electrons ($N=12$): two shells and commensurability at \boldmath{$\omega=10^{-3}$}}
Already at $\omega=0.01$ the system is well on the Wigner side, with two-shell formation frequent.  
Using a conservative radial-gap threshold $\tau{=}3.0$ we find a two-shell fraction of $0.761$, with inner–ring occupancies
\[
(1,11)=32.639\%,\quad (2,10)=25.688\%,\quad (3,9)=13.716\%,
\]
and bond order / angular fluctuations
\[
\begin{aligned}
&(1,11):\quad |\Phi_{\rm out}|=0.276,\ \ \text{Lind}_{\rm out}=0.495,\\
&(2,10):\quad |\Phi_{\rm in}|=0.661,\ \ \text{Lind}_{\rm in}=0.315;\ \ |\Phi_{\rm out}|=0.300,\ \ \text{Lind}_{\rm out}=0.432,\\
&(3,9):\quad |\Phi_{\rm in}|=0.565,\ \ \text{Lind}_{\rm in}=0.342;\ \ |\Phi_{\rm out}|=0.322,\ \ \text{Lind}_{\rm out}=0.408.
\end{aligned}
\]

Upon lowering the trap to $\omega=10^{-3}$, shelling becomes ubiquitous and a \emph{commensurate} split emerges as modal.
For the full $500{,}000$-frame ensemble the two-shell fraction is
\[
f_{\rm 2shell}(\tau) \approx 0.991,\ 0.966,\ 0.923
\quad\text{for}\quad \tau=2.0,2.5,3.0,
\]
with inner–ring occupancy histograms that are very stable under~$\tau$.
At $\tau{=}3.0$ the two-shell subset ($46.15\%$ of all frames) has
\[
(3,9)=43.29\%,\quad (2,10)=19.89\%,\quad (1,11)=12.54\%,
\]
with the remaining weight spread over $(4,8)$, $(5,7)$ and rarer configurations.
The median outer radius in trap units is ${\approx}7.02$, consistent with the
energy-based estimate of a strongly swollen cloud.

On $(3,9)$ frames the rings show strong bond order with small angular Lindemann:
\[
|\Phi_3|=0.687\!\pm\!0.229,\ \ \text{Lind}_{\rm in}=0.223;\qquad
|\Phi_9|=0.526\!\pm\!0.194,\ \ \text{Lind}_{\rm out}=0.224.
\]
Using a reduced bundle with $K\approx2.5\times 10^5$ frames that satisfy the
two-shell criterion, we construct II/IO/OO histograms and weight them by the
combinatorial mixture $(0.045{:}0.409{:}0.545)$.
The resulting synthetic $g(r)$ has cosine similarity $0.9982$ to the global
$g(r)$, i.e.\ is indistinguishable at our statistical resolution
(see Fig.~\ref{fig:N12_w0001_decomposition}).

\begin{figure}[htbp]
  \centering
  \includegraphics[width=0.85\textwidth]{mixfit_N12_w0.001_nin3.pdf}
  \caption{Shell-resolved reconstruction of $g(r)$ at $\omega=10^{-3}$ for $N{=}12$.  
  The global pair distribution is reproduced by a convex mixture of inner--inner, inner--outer and outer--outer histograms with weights set by the observed $(n_{\rm in},n_{\rm out})$ statistics, confirming that the structure is geometric rather than a sampling artifact.}
  \label{fig:N12_w0001_decomposition}
\end{figure}

This is exactly the Wigner-molecule regime: the \emph{radial} structure freezes
into concentric shells, while residual \emph{azimuthal} dynamics mixes a few
low-lying ring occupancies.
The $(3,9)$ commensurate split at the weakest trap aligns with classical
ground-state analyses for parabolic Coulomb clusters~\cite{schweigert1994spectralpropertieschargedparticles,Kong_2002}; 
the finite weight of $(1,11)$ and $(2,10)$ reflects near-degeneracies and
quantum rotational mixing~\cite{manninen2007metalclustersquantumdots,Filinov_2001}.

\begin{figure}[htp]
  \centering
  \includegraphics[width=0.8\textwidth]{N12_all_densities}
  \caption{Twelve-electron radial densities $P(r)$ across $\omega$.}
  \label{fig:12N_radial_densities}
\end{figure}

\paragraph{Significance.}
Taken together, the $\omega=10^{-3}$ results provide a sharp, quantitative realization of the Wigner-molecule picture in parabolic dots:
(i) shell-resolved reconstruction of $g(r)$ at machine precision,
(ii) strong, commensurate bond order on rings with small angular Lindemann ratios,
(iii) topology fractions that smoothly approach the classical ordering as confinement weakens,
all consistent with~\cite{Egger_1999,Filinov_2001,schweigert1994spectralpropertieschargedparticles,Kong_2002}.
Our sector-resolved $|\Phi_m|$ are extracted \emph{within} a fully variational quantum ansatz and measured conditionally on shell occupancy, 
which (to our knowledge) has not been reported before for $N=12$ parabolic dots.
