\section{Conclusions for quantum dots}
\label{sec:qd-conclusions}

The quantum-dot benchmarks in this chapter serve a dual purpose: they test the
numerical stability and expressivity of the Slater--Jastrow PINN+backflow
ansatz, and they provide a controlled setting in which to study Wigner-molecule
formation and the internal representations learned by the network.

On the \emph{energy} side, the results in Sec.~\ref{sec:energy-discussion}
show that a single, trap-unit–normalized architecture with soft-core pair
features and an explicit cusp achieves DMC-level accuracy for two-electron
dots and maintains sub-$0.1\%$ relative errors for $N\in\{6,12\}$ across
$\omega\in\{0.1,0.5,1.0\}$. The same setup extrapolates smoothly to
ultra-weak traps at $\omega=10^{-3}$, where no DMC data are available, with
internal consistency checks (scaling exponents, virial ratios and structural
diagnostics) supporting the reliability of the resulting variational
predictions. Crucially, these accuracies are obtained with a relatively
modest SR tail on top of a residual-based pretraining stage, emphasizing that
conditioning and inductive bias matter at least as much as raw sample count.

On the \emph{structural} side, Sec.~\ref{sec:wigner-molecule} establishes that
the weak-trap states are genuine Wigner molecules rather than vaguely
``strongly correlated'' dots. As $\omega$ decreases, all three systems
$(N=2,6,12)$ undergo a smooth crossover in which (i) radii and localization
ratios evolve towards large, stiff configurations, (ii) the energy
partitioning approaches the nearly classical virial form
$V_{\rm int}\approx 2V_{\rm trap}$, and (iii) shell structures and bond
orders emerge that match classical Coulomb-cluster topologies. For $N{=}6$,
the dot forms a single rotating Wigner ring with a quantum mixture of
$(1,5)$ and $(0,6)$ sectors; for $N{=}12$, a robust two-shell Wigner molecule
with a commensurate $(3,9)$ split becomes modal. Shell-resolved
reconstructions of $g(r)$ confirm that the full pair structure is explained
by these topologies and their combinatorics, with cosine similarities
$\approx 1$ between reconstructed and global $g(r)$.

On the \emph{representation} side, Sec.~\ref{sec:repr-discussion} shows that
this rich real-space physics is compressed into a remarkably simple latent
geometry. The correlator $f_{\text{net}}$ lives on a low-dimensional manifold
($r_{\rm eff}(Z)\sim 1\text{--}2$) whose leading PCs behave as collective
coordinates for global size and radial fluctuations, while finer angular and
topological information resides in conditional distributions on this manifold.
Backflow, in contrast, realizes a high-dimensional displacement field
$r_{\rm eff}(\Delta x)\sim 10\text{--}20$ whose energetic leverage is
concentrated in near-field configurations at intermediate traps and becomes
negligible deep in the Wigner regime. In the ultra-weak limit, the optimizer
effectively switches off backflow: the static Slater--Jastrow correlator
saturates the physics, and the model reduces to a low-rank Wigner-molecule
ansatz.

Taken together, these findings support a physically meaningful decomposition
of the neural ansatz into (i) a compact, smooth correlator that plays the
role of a learned set of collective coordinates, and (ii) a local,
many-body backflow channel that only activates where nodal refinements are
energetically relevant. The quantum-dot results thus provide both a stringent
validation of the method and a template for applying the same architecture to
more complex finite fermion systems, where similar trade-offs between global
structure, local correlations and representation rank are expected to arise.
